\chapter{Introduction to Artificial Chemistries}\label{introduction-to-achems}

Artificial Chemistries provide an interesting testbed for investigating various evolutionary phenomena. This review \namecref{introduction-to-achems} provides background material to aid in understanding how \glspl{achem} are almost uniquely suited to the exploration of our research questions.

Fundamentally, \glspl{achem} provide a tunable evolutionary system, capable of highly complex behaviour, built around familiar metaphors (in many cases real-world chemistry, and potentially biology). In a molecular \gls{achem}, a set of rules describing how atoms interact gives rise to emergent forms: molecules. At a higher level, these molecules, under the same interaction rules, also interact in patterns (reactions.)

Still higher emergent levels emerge under favourable conditions. Reactions may form cycles, where a sequence of reactions eventually returns to an earlier state. Cycles in particular are interesting as many biological processes are cyclical\footnote{In the broadest sense life can be seen as an autocatalytic process where an entity catalyses the production of one or more descendant entities.}. Replication, resulting in an exact copy of an entity, is a macro-example of a cycle; metabolism is another. Building on the apparent correspondence between higher emergent levels in \gls{achem} evolution and biology, we believe like others (\eg \textcite{Steel2013}) that cycles, of some form, are a necessary building-block for more complicated structures in \glspl{achem}.

Artificial chemistries are regularly employed in three application areas: real-world chemistry simulators; tools for the exploration of artificial life, and models to test various hypotheses of the origin of life. Chemistry emulators and origins-of-life tools aim for fidelity with real-world chemistry, unlike most artificial life models. Real-world fidelity requires either the use of a library of predefined reactions, which conflicts with the goal of unlimited extension, or a chemically plausible method of constructing reactions from first principles. Because of the complexities of real-world chemistry this later method appears to be quite difficult, and the goal of a realistic, computationally practical, artificial chemistry remains open. 

However, the more limited objective of a less realistic, but still unbounded chemistry, has been achieved (see \cref{classification-of-artificial-chemistries} for examples.) One advantage of semi-realistic \glspl{achem}, as pointed out by \textcite[5]{Funes2001}, is that it is easier to evaluate solutions in a domain close to the real world as opposed to a purely symbolic or abstract domain such as a lambda-calculus, or a programmatic environment like Tierra. When contemplating difficult problems such as complexity our intuition can be helpful, but only in situations close enough to our normal experience for it to be relevant. \Glspl{achem} are a particular type of model, familiar from real world chemistry, for the simulation of reaction-based systems. 

A mathematical treatment of \glspl{achem} can be found in \textcite{Benko2009}. \Textcite{Dittrich:2001zr} and \textcite{Suzuki2008a} provide excellent reviews of the field, while an influential taxonomy is given in \textcite{Dittrich:2001zr}, described in \cref{classification-of-artificial-chemistries}. Every \gls{achem} defines a set of constitutive elements (molecules), a set of transformation rules (reactions) and a mechanism for choosing and ordering the sequence of reactions (a reactor algorithm) \parencite{Dittrich:2001zr}. These components and taxonomies are described in more detail in the next \namecref{classification-of-artificial-chemistries}, but first we digress to provide a short review of chemistry in the real, rather than artificial, world.

\section{An introduction to real-world chemistry}\label{sec:real-world-chemistry}

\Glspl{achem} are not necessarily models of real-world physical chemistry but most \glspl{achem} borrow, at least, from the language of real-world chemistry. An understanding of the key concepts in physical chemistry is therefore useful in appreciating \glspl{achem}.

The richness of the physical world is built on a substantial foundation of chemical complexity. A \textit{reaction} transforms \emph{reactants} into \emph{products}, and is often described by the quantities (or stoichiometry) of the reactants and resulting product molecules. The reaction can also be characterised by a description of the dynamics, or kinetics, of the reaction to explain how the reaction proceeds in response to temperature changes or to varying concentrations of reactants. Reactions often require an input of energy, such as from molecular collisions, to proceed; the energy required is called the reaction's activation energy (\(E_a\)), and is specific to the particular reaction. In general there is no accurate mechanism to predict reaction dynamics without experiment.

The \textit{reaction rate} is the change in concentration of a substance over time: \(\frac{-d[A]}{dt} = k[A][B]\) (where the notation \([X]\) means that concentration of \(X\) and \(k\) is the reaction rate constant) leading to Arrenhius' description of the relationship between the activation energy (\(E_a\)), the temperature (\(T\)) and the rate constant (\(k\)): \(k = Ae^{E_a/RT}\) ($R$ is the universal gas constant.) The \textit{reaction order} describes how the reaction rate changes with the concentration of the reactants, usually captured as a first-, second-, or third-degree polynomial expression determined empirically. For example, the reaction rate equation for \(2NO + Cl_2 \rightarrow 2NOCl\) is \(rate = k[NO]^2[Cl_2]\) (experimentally determined), and is second-order with respect to \(NO\), first-order with \(Cl_2\) and overall (the sum of terms), third-order (example taken from \textcite{Kotz2006}.)

Reactions can generally be decomposed to a chain of \emph{elementary steps}, with each step resulting in a single change, such as bond formation or cleavage, to the reacting molecules. Elementary steps are somewhat predictable, practical reactions somewhat less so. Generally in experimental chemistry we know the reactants and products and can sometimes deduce the sequence of elementary steps.

In modelling, we can either represent reactions exactly as an atomic transformation from reactants to products with characteristics that can only be determined experimentally, or we can attempt to construct the reaction from a sequence of elementary steps with the properties of the reaction derived from the properties of the steps involved. This later approach is the only practical one when the reaction is novel, or when we lack experimental data. Several alternate sequences of steps, or \textit{reaction pathways}, may be possible between reactants and products. Each pathway will have a different activation energy, and hence reaction rate.

The kinetics of elementary steps are defined by the \emph{stoichiometry}. In theory you might expect to be able to predict the overall reactions kinetics from the composition of these elementary steps. In practice, the results are close, but not exact, when compared to experiment. However they form a useful abstraction for analysis. The reaction rate of an elementary step is defined by the stoichiometry, where the rate equation is the product of the reactant concentrations and the rate constant. Therefore a step with one reactant has order 1, a bimolecular step (\(A + B\)) has order 2 and so on. The step \(2A + B\) has rate equation \(k[A][A][B] = k[A]^2[B]\) and is of order 3. 

Elementary steps are an abstraction, an aid to analysis, of the underlying molecular dynamics. At the molecular level, reactions can be
modelled as a series of collisions between molecules. The reaction rate is then determined by the percentage of collisions (or in other words,
the concentration) that are energetic enough to overcome the inherent stability of the interacting molecules and cause a change in molecular
structure or shape (in other words, the activation energy).

For an entity to emerge from a network of reactions, the core problem is how to privilege the reactions that are useful in producing products (ones that can enter into other necessary reactions), over the ones that are not. Given a collection of molecules, the set of all possible reactions is determined by the chemistry that governs how molecules interact to form other molecules. The reactions that actually take place though are driven by molecular concentrations and kinetics. Molecules increase in concentration as they are produced in reactions, and become scarcer as they are consumed as reactants. More common molecules are more likely to be chosen as reactants, and less common ones less likely. This can also affect the direction of a reaction--for reactions which may run in either direction, the relative concentrations of molecules on one side of the reaction versus those on the other side determines the direction in which the reaction is most likely to run.

Recall that the rate of a reaction is a function of concentration (at the gross level) or collision rate (at the molecular level) and the
activation energy for the reaction, which at the molecular level is directly related to the energy required to overcome the stability of the
reactants. There are therefore two clear mechanisms to alter the rate for a reaction: either reduce activation energy, or increase
concentrations. A catalyst does precisely the first, and autocatalysis is one method for the second.

A \emph{catalyst} is anything that isn't consumed by the reaction and that affects the rate (or the kinetic equation for the reaction) without
affecting the reaction's equilibrium constant. Catalysts allow the reaction to proceed by an alternate, lower activation energy, pathway.
The reaction equation remains the same, but the dynamics are changed--in the case of biological enzymes the rate can be increased by several
orders of magnitude over the uncatalysed reaction, enabling reactions fundamental to life that would be effectively impossible in an
uncatalysed form. At the molecular level catalysts often function by providing a substrate that preferentially attacks the bonds critical to
the reaction. The platinum within an automotive catalytic converter is a well-known example of a catalytic substrate. Catalysts are often show above the reaction arrow in standard chemical reaction notation, \eg$A\xrightarrow{catalyst} B$.

In \emph{autocatalysis}, introduced by \textcite{Ostwald1890}, rather than reducing the required activation
energy, autocatalysis increases the reactant concentrations: autocatalysis reactions form feedback loops where a compound is both a
reactant and a product. In the standard definition, an autocatalytic reaction is one that is catalysed by its own products, resulting in a
characteristic rate acceleration over time given by the differential equation--
\[\frac{dx_i}{dt} = k(\mathbf{X}) \cdot x^n_i + f(\mathbf{X})\] where $n$ is the order of the reaction, and $f(\mathbf{X})$ the contribution from all other elements of the system \parencite{Plasson2010}. As an example, in the indirect network autocatalysis of glycolysis where the pattern is ATP \(\rightarrow\) n.ATP.

Autocatalysis may be realised by \begin{inparaenum}
	\item either a single reaction \eg$\Sigma x_i + A\rightarrow \Sigma y_j + \Sigma B_k$ (where $n$ of the ${B_0, B_1...B_k}$ products are equivalent to $A$) or a linear chain of reactions through intermediate products, called template autocatalysis, branching chain reactions or autocatalytic sets \parencite{King1978}, or
	\item by a reaction network. Networks may be indirect through a series of intermediary products, or collective where there is no connection between the component cycles other than through catalysis--as seen for example in the replication of viroids where each RNA strand can catalyse the production of the other.
\end{inparaenum}

However, in all cases, the reactions reduce to the defining $A \rightarrow A_0...A_n$ pattern described by Ostwald's differential equation \parencite{Plasson2010}.

\section{Classification of Artificial Chemistries}\label{classification-of-artificial-chemistries}

A straightforward scheme for classifying \glspl{achem} is given in \textcite{Faulconbridge2011}, based on the relationship between a molecule's representation and its properties:

\begin{itemize}
	\item \emph{Symbolic}. Symbols/molecules have no inherent meaning, so no ``implicit reactions'', only preprogrammed ones.	
	\item \emph{Structured}. One or more atoms arranged in a structure (such as a string, tree, or graph) with bonds within atoms to form molecules. This results in unlimited capability, but is computationally expensive.	
	\item \emph{Sub-symbolic}. Emergence of properties from lower-levels (\eg real chemistry, also neural networks). The symbols (for example, atoms) have an internal structure which determines the macro-properties of the chemistry.
\end{itemize}

Another system was presented by \textcite[p.132]{Nellis2012} with \glspl{achem} categorised according to a quite different set of three factors:
\begin{itemize}
	\item \emph{World}. Physical elements such as molecules and atoms.
	\item \emph{Chemistry}. The interactions between the elements.
	\item \emph{Constraints}. Restrictions on how the world and chemistry interact. Nellis gives as an example a binding model that describes which interactions are possible between elements.
\end{itemize}

However, there is an earlier, and more influential, categorisation by \textcite{Dittrich:2001zr} which also coincidentally categorises \glspl{achem} against three factors, and which is commonly used in the literature (\eg \textcite{Lenaerts2009,Gardiner2007}). This taxonomy is described in more detail in the \namecref{dittrich} below.

\section{\textless{}\emph{S},\emph{R},\emph{A}\textgreater{} classification scheme of Dittrich et al.}\label{dittrich}

In \textcite{Dittrich:2001zr}, a broad variety of \glspl{achem} are classified according to the tuple \textless{}\emph{S},\emph{R},\emph{A}\textgreater{} where \emph{S} describes the form of the component molecules in the \gls{achem}, \emph{R} the rules for the interactions between the molecules, or reaction rules, and \emph{A} the mechanism used to select molecules for reactions\footnote{An alternative \textless{}\emph{S},\emph{I}\textgreater{} form is also described, where \emph{I} combines the \emph{R} and \emph{A} components, but is less descriptive and less commonly used in the literature.}. 

\subsection{Set of molecules (\textless{}\emph{S}\textgreater{})}

This element describes how molecules are represented in the \gls{achem} (\eg labelled graphs \parencite{Faulconbridge2011} or binary strings \parencite{Banzhaf94}), and how any implicit properties of molecules may be derived, such as bond energy calculations by Extended Huckel Theory (EHT) (\eg textcite{Benko2003}.)

\subsection{Reaction rules (\textless{}\emph{R}\textgreater{})}

Reaction rules in an artificial chemistry describe how reactants are transformed to reaction products. Reactions may be pre-defined (often the case when simulating real-world chemistry where exact pathways are important) or dynamically determined using molecular properties. In \glspl{achem} enforcing conservation of energy, elements are neither created or destroyed so reactions can be represented solely by rearrangements or bond changes. \Textcite{Tominaga2007} showed for a particular artificial chemistry, that it is computationally universal with only uni-molecular and bimolecular reactions; the reaction rules in most \glspl{achem} therefore only describe these forms of reaction.

\subsubsection{Constructive chemistries}\label{constructive-chemistries}

%\quote{A distinguishing feature of chemistry is that the changes of molecules upon interaction are not limited to quantitative physical properties such as free energy, density, or concentrations, since molecular interactions do not only produce more of what is already there--rather, novel molecules can be generated.}{\parencite{Benko2009}}

The action of the reaction rules are \textit{constructive} \parencite{Fontana1994} if new components may be generated through the action of other components--a form of emergence, explicitly linked to the production of novelties: \quote{construction: to understand how the organizations upon which the process of natural selection is based arise, and to understand how mutation can give rise to organizational, that is: phenotypic, novelty.}{\parencite{Fontana1994}}

A strongly constructive system is one which maintains closure\footnote{Or as stated by \textcite[217]{Fontana1994},``A strongly constructive system that contains agent \emph{A} must cope with the network of its implications. But, then, it also must cope with the implications of the implications. And so on.''}, and in which there is self-consistency and some form of logical structure. Both implicit laws and implicit molecule definitions are required for constructive chemistries:

\begin{itemize}
	\item Implicit reaction laws are molecular structure-based, while explicit laws are independent of molecular structure.
	\item Implicit molecule definitions provide a description for a molecule's construction, while explicit molecule definitions are taken from an fixed set of symbols.
\end{itemize}

Strongly constructive chemistries provide a mechanism for exploration in the \gls{ea} sense--new products can be generated, and new products can participate in reactions. A weakly constructive approach might be to pre-specify all possible reactions; the strongly constructive approach is to generate reactions ``on-the-fly'' from the structures of the interacting molecules. 

Pre-specification is well suited to simulating real chemistry as it allows properties of reactions observed in chemical experiments to be attached to their simulated equivalents. However, it does not allow for arbitrary reactions, and it requires reaction properties to be predetermined--rather difficult for novel or artificial reactions. Clearly, although constructive, this is not strongly constructive (in the sense of \textcite{Fontana1994,Dittrich:2001zr}) as it is not open-ended. However, \textcite{Hartenfeller2011} suggests that it is still useful for applications such as drug discovery.

An artificial chemistry with the ability to create reactions ``on-the-fly'' given a set of possible reactants may discover more than one possible reaction pathway between the same reactants and products. The method used to choose one reaction pathway from the alternatives is an important component of the artificial chemistry, and the mechanism may be tuned or tailored to privilege or preferentially chose particular types of pathways independent to other factors such as temperature or concentration. In Chemistry the choice of reaction pathway is fundamentally linked to those other properties and cannot be treated independently.

\subsection{Reactor algorithm (\textless{}\emph{A}\textgreater{})}

The reactor algorithm provides the mechanism to select (in effect, to order) reactions, or in the words of \textcite[sect. 4.1.3]{Faulconbridge2011}, the ``...algorithm which describes the order of and intervals between reactions, starting from an initial collection of molecules...''

If the reaction rules are analogous to chemistry, the reactor algorithm is analogous to physics--the way molecules move and collide in the reaction vessel determines which reactions are possible (for example, by providing enough kinetic energy to overcome a reaction's activation energy), and the order in which the reactions occur.

\Textcite{Faulconbridge2011} identifies three basic types of \emph{mixing method} or reactor algorithm:
\begin{enumerate}
	\item
	\emph{Well-mixed/aspatial}. Either discrete time (uniform probability distribution for selecting reactions) or continuous time \parencite{Gillespie1976} assuming the reactions are known in advance (which of course is not possible for a strongly constructive \gls{achem}.)
	\item
	\emph{N-dimensional}. A grid with reactions in adjacent cells, or a continuous space where molecules have position and velocity. The main advantage is the ability to simulate spatial affects; the primary disadvantage is performance.
	\item
	\emph{Mixed scale}. Hierarchical spaces, such as aspatial cells within bigger grid, mostly for simulating biology (\eg \textcite{Jeschke2008}). One possible advantage is the potential for parallelisation.
\end{enumerate}

\section{Applications in real-world chemistry}\label{applications-in-real-world-chemistry}

\Glspl{achem} may be applied to backward-chain from a set of desired products to identify a set of currently-available initial molecules, for example in drug discovery (\eg \textcite{Hartenfeller2011}). A second use is in reaction network discovery, where the goal is to describe a closed set of reactions and reactants from some initial reactants and reactions (\eg \textcite{Faulon2001}). Finally, \glspl{achem} can be used in the modelling of biological phenomenon such as enzyme function (\eg \textcite{Flamm2010}.)

The primary requirement in these cases is fidelity with real-world chemistry, which requires either a library of empirically derived reaction definitions and rates, or a model capable of accurately simulating quantum-mechanical processes. The latter approach has been taken by a family of Artificial Chemistries, beginning with \textcite{Benko2003}, built on Extended H\"{u}ckel Theory with parameters taken directly from chemical experiments and later extended (for example in \textcite{Benko2005}) to a general purpose model with parameters derived from theoretical chemistry. The model was used in \textcite{Hogerl2010} for the study of the behaviour and topology of chemical reaction networks, specifically Diels-Alder and Formose reaction networks, and in a series of papers (\eg \textcite{Flamm2010,Ullrich2010}) for the examination of the evolution of metabolic networks in early organisms using a simple model of RNA coding for catalysts.

The core problem in drug-discovery is the selection of a set of reactions to generate a given product. The CASREACT Chemical Reaction database \footnote{\url{https://www.cas.org/content/reactions}} references more than 80.5 million known reactions in published work; clearly it is impractical in a laboratory setting to simply apply each of these in turn to a set of initial molecules in the hope that the goal compound will emerge. \Textcite{Hartenfeller2012} used simulations in reaction-SMARTS and RDKit \parencite{rdkit} to conclude that a much smaller set of 58 reactions, acting on a set of 10,000 to 50,000 ``building block'' molecules, might instead meet the requirements for \textit{de novo} drug discovery. Of the 58 nominated reactions, 29 were ring-forming; the suggestion is that as so many interesting compounds involve molecular rings, the lack of ring forming reactions was a deficiency of previous approaches. With this small reaction set, chosen for the practicality of transferral to the laboratory benchtop, the combinatorics are such that a search mechanism is needed to identify the most promising pathways from the initial molecule set to the destination compound.

Finally, \Glspl{achem} have been used as tools to explore the role of function prediction, for example in the interesting, but necessarily simplified, approach taken by \textcite{Flamm2010,Ullrich2010}. Determining the shape, and hence the function, of an enzyme from its RNA transcript is perhaps the most important problem in current molecular biology: the three-dimensional structure of a molecule--the arrangement of the elements in space--drives many of even the simplest chemical reactions. For example, acid-base reactions result from the shift of charge from one region of a molecule to another, revealing one region while shielding another from activity. Some of the most complex reactions are shape-driven: the function of many enzymes (biological catalysts) derives from their shape, and furthermore this shape is often under regulatory control. \Textcite{Flamm2010,Ullrich2010} attach a catalytic function to a molecule based on its secondary structure (shape) and then investigates the influence of these functions upon the evolution of early metabolic networks. 
	
\section{Origins of life modelling with artificial chemistries}\label{achem-and-ool}

Real-world chemical processes are also important to modelling scenarios for the origin of life or other related areas such as the formation of metabolic networks in the earliest protocells. In most cases though the specific focus is less on the bottom up model constructed from the most basic elements (although Kauffman's autocatalytic protein sets, and Kaneko's protocell toy mode are counter-examples,) and more on task-based models of processes where the particular base level is predetermined by the researcher, such as Ganti's chemoton \parencite{Ganti:2003hl}.

\Textcite{Farmer1986} describe an \gls{ode} model of polymers where bidirectional reactions connect each polymer condensate $c$ from monomer or polymer constituents $a$ and $b$, catalysed by some enzyme $e$, in the presence of water $h$. All reactions are catalysed (by some randomly chosen polymer), principally to reduce computational complexity; this is justified by the order of magnitude difference in the reaction rates between catalysed and uncatalysed reactions.  The model commences with a food set or initial population of monomers and simple polymers in a well-mixed chemostat, no perturbations, and ends when no further new polymers are generated.

Some other examples of artificial chemistries for the investigation of the origin of life include:

\begin{itemize}
\item
\emph{Lattice \gls{achem}} \parencite{Madina2003,Ono2000}. Membrane formation and cell division, assumeing five different types of particles (some hydrophilic and some hydrophobic) that together form an autocatalytic cycle similar to those observed in biological cells.
\item
\emph{SCL}. Three types of particle are employed by the Substrate-Catalyst-Link (or SCL) chemistry of \textcite{Varela:1974qd,Suzuki2008}: the eponymous Substrate ($S$), Link ($L$) and Catalyst ($C$). Cells are formed from links around a catalyst, with a single predefined reaction rule $S + S\xrightarrow{C} L$ and some straightforward constraints on movement of the particles in the matrix (for example, bonded Link particles cannot cross each other.)
\item
\emph{\Textcite{Flamm2010, Ullrich2010}}. The evolution of metabolic networks in early organisms using a simple model of RNA coding for catalysts
\item \emph{\Textcite{Hogerl2010}}. The simulation of chemical reaction networks characteristic of the transition to life, specifically the Diels-Alder and Formose reaction networks.
\item
\emph{\Textcite{Dorin:2006fk}}. Concerned, almost uniquely, with a chemical ecosystem, based on a set of atoms interacting in pre-specified ways to represent biological photosynthesis, respiration and biosynthesis (or growth). The goal is to explore the interactions in an ecosystem made up of a set of organisms pre-built to perform various defined roles.
\item
\emph{\Textcite{Gardiner2007}}. A string-based chemistry to investigate protein metabolism evolution under genetic control. Three types of molecule--protein, gene and service molecule--react in ways determined by the types of interacting molecules. The type and pattern of molecules define the type of interaction.
\item
\emph{\Textcite{Fernando:2008xy,Fernando:2007pf}}. A flow-reactor for the evolution of metabolism in lipid aggregates based on predefined molecular types and reactions.
\end{itemize}

\section{Artificial chemistries and Alife}\label{artificial-chemistries-and-alife}

Artificial Chemistries have also been used in the exploration of open-ended or creative evolution (\eg \cref{tab1}). Squirm3 \parencite{Hutton2002,Hutton2009} adopts fixed molecule types, and pre-defined reactions for replication and gene-sequence transcription, and so although capable of interesting behaviour is not capable of unlimited extension. StringMol \parencite{Hickinbotham2011}, a bacterial inspired microprogram chemistry, demonstrates a rich inheritance mechanism using string-matching as a model for molecular binding, and RBN-World \parencite{Faulconbridge2011} shows that a form of Random Boolean Network, with the addition of a bonding mechanisms to allow for composition and decomposition of RBNs, can be used to build a chemistry capable of almost limitless extension out of non-traditional components.

\begin{table}
	\scriptsize
	\begin{center}
	\caption{A sample of Artificial Chemistries for open-ended evolution. Constructive chemistries are capable of ongoing extension.}
	\label{tab1}
	\begin{tabular}{@{}p{7.5cm}p{6cm}@{}}
		\toprule
		Chemistry                                                          & Constructive? (see \cref{constructive-chemistries})\\ 
		\midrule
		\textcite{Ducharme2012}                                                & Yes                                      \\
		StringMol \parencite{Hickinbotham2012}                             & Yes                                      \\
		RBN-World \parencite{Faulconbridge2011}                            & Yes                                      \\
		\textcite{Lenaerts2009}                                                & Yes--molecular interactions             \\
		NAC \parencite{Suzuki2006}                                         & Yes                                      \\						
		GGL/ToyChem \parencite{Benko2005}                                  & Yes                                      \\
		\midrule
		Substrate-Catalyst-Link (SCL) \parencite{Varela:1974qd,Suzuki2008} & Unknown                                  \\
		\textcite{Fernando:2008xy,Fernando:2007pf}                             & No--atomic reactions                    \\
		\textcite{Gardiner2007}                                                & No--atomic reactions                    \\
		Lattice \gls{achem} \parencite{Ono2000,Madina2003}        		   & No                                       \\
		GGL/ToyChem \parencite{Benko2003}                                  & No--pre-defined reactions only          \\
		Squirm3 \parencite{Hutton2002,Lucht2012}                           & No                                       \\
		ZChem \parencite{Tominaga2004}                                     & No--reactions are atomic with wildcards \\
		\bottomrule
	\end{tabular}
	\end{center}
\end{table}

\section{Conclusion}

In this \namecref{introduction-to-achems}, we have:

\begin{itemize}
	\item Introduced the basic elements of \glspl{achem}.
	\item Established those elements within a commonly-used classification scheme \parencite{Dittrich:2001zr} which uses the three factors--\emph{S} describing the form of the component molecules, \emph{R} the rules for the interactions between the molecules, and \emph{A} the mechanism used to select molecules for reactions--to classify \glspl{achem}.
	\item Provided a reminder of the requirement for constructive reaction rules in open-ended chemical systems.
	\item Reviewed the three dominant areas for the use of \glspl{achem}: real-world chemical modeling, artificial life, and the origin of life.
	\item Suggested, by providing examples of other \glspl{achem} in the exploration of \gls{alife} and the origin of life, the suitability of an \gls{achem} for addressing our research questions.
\end{itemize}

From the brief review in \cref{achem-and-ool} of \glspl{achem} in the exploration of the origin of life it is clear that there is little consensus around the form of the preferred \gls{achem} to adopt. In the next \namecref{ch:toyworld}, we shall move from the general description of \glspl{achem} to the introduction of a new and specific \gls{achem}, ToyWorld.