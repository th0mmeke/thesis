\chapter{ToyWorld implementation}\label{toyworld-implementation}

ToyWorld is custom Python code that uses third-party packages and
libraries for certain functions. The most significant chunks of external
functionality come from RDKit, an open-source toolkit for
ChemInformatics, and PyMunk, a physics toolkit. Additionally, we use the
NetworkX package for Python for network graph manipulations (mainly in
the cycle evaluators) and in \url{Molecule} to determine the individual
molecules within an aggregation, and we use arrays from the Numpy
package for Python to record the molecular populations.

\section{RDKit}\label{rdkit}

RDKit provides a number of useful capabilities including format
conversions to and from SMILES \autocite{smiles}, a standard language
for representing molecules; simple visualisations of molecule structure;
standard sanity checks for molecular structure; and molecular
manipulations. RDKit is also well-tested, and keenly optimised for
performance.

The ToyWorld \url{Molecule} is a subclass of a RDKit \url{AllChem.Mol}.
Molecules in ToyWorld are therefore built from RDKit \url{Atoms} and
\url{Bonds}, although all manipulations and transformations of these
elements are written in our own code.

RDKit also provides a number of handy functions to the ToyWorld Chemical
Model:

\begin{enumerate}
\def\labelenumi{\arabic{enumi}.}
\item
  Format conversions between RDKit molecules and a string representation
  of a molecule (SMILES); conversion from the representation of a
  molecule into a canonical form; and representation as a 2D graphical
  form
\item
  Reference information: for example, the mass of an atom and number of
  outer electrons for an atom
\item
  Molecular structure manipulation: iteration over the atoms or bonds in
  a molecule; addition, modification and deletions of bonds and atoms
\item
  Utility functions: combining two representations, each of one
  molecule, into one representation of two molecules, and vice versa;
  sanitising the representation of a molecule by checking for molecular
  validity
\end{enumerate}

These are so standard that it isn't worth our while to replace them with
anything specific to ToyWorld.

\section{PyMunk}\label{pymunk}

PyMunk is used extensively for 2D Physics calculations, for example:

\begin{enumerate}
\def\labelenumi{\arabic{enumi}.}
\item
  Calculating the future position of a molecule inside a reaction vessel
  based on the molecules velocity and current position
\item
  Collision detection between two (or more) molecules
\item
  Summing the forces acting on a molecule and adjusting the molecule's
  acceleration
\item
  Simple visualisation of the molecules within the reaction vessel as
  shapes on a 2D plane
\end{enumerate}

\begin{algorithm}[ht]
\DontPrintSemicolon
\KwData{Population of Molecules}
\KwResult{Updated population}

$\text{collision list}=\{\}$\;
\While{not finished}{
    \BlankLine
    \tcp{Selection of reactants}
    \While{$\text{collision list}=\{\}$}{
        Move all molecules by time increment $\tau$\;
        $\text{collision list} \leftarrow \text{all collisions}$\;
    }
    reactants $\leftarrow$ select two colliding molecules from collision list\;
    \BlankLine
    \tcp{Construction of reaction}
    Discover all possible reactions that could occur between reactants\;
    Determine the available energy to drive the reaction\;
    \uIf{$\exists$ a reaction option that is possible with the available energy}{
        Determine chemical products of the reaction\;
        Transform chemical products into physical molecules with kinetic and internal energy\;
        \uIf{a change in maximum or minimum molecular velocities}{\label{alg:adjust_tau}
            Adjust time step increment $\tau$\;
        }
    }
    \uElse{
        No reaction possible -- so just bounce the molecules off each other (a collision)\;
    }
}
\caption{The main simulation loop in ToyWorld}
\end{algorithm}

\begin{sidewaystable}
\begin{center}
\tiny
\begin{tabular}{@{}lllp{10cm}@{}}
\toprule
Parameter & Default value & Possible values & Description \\
\midrule
Energy&300&Int&The initial average kinetic energy (KE) for each molecule in the reaction vessel\\
EnergyModel&\url{DefaultEnergyModel}&Subclass of \url{DefaultEnergyModel}&More sophisticated model of energy inflow and outflow from the reaction vessel\\
RadiationRate&0.0&Float&Proportion of kinetic energy lost (apparently by radiation) each unit time interval from the reaction vessel\\
EnergyInput&0.0&Float&Absolute energy added to reaction vessel each unit time interval\\
Reactions&\url{EmergentReactions}&Subclass of \url{Reactions}&Reaction mechanism - EmergentReactions provides a fully constructive reaction chemistry\\
ProductSelectionStrategy&"energy"&String&See Section \ref{product-selection-strategies}\\
Molecule&molecule.Molecule&Subclass of Molecule&Model of a molecule\\
Vessel&\url{ReactionVessel}&Subclass of \url{ReactionVessel}&Model of the container in which the simulation proceeds\\
DipoleForceConstant&0.01&Float&Multiplier for the electro-magnetic force acting between two \url{ChargedMolecule}s\\
\midrule
name&<None>&String&Name of the experiment - primarily used to identify the output data files for the experiment\\
PopulationFilename&None&Filename&Name of the XML file defining the initial set of molecules to be placed in the reaction vessel\\
Iterations&10000&Int&Number of reactions before finishing\\
recover&True&Boolean&If True then do not redo completed repeats if the simulation is restarted, say after a machine reboot\\
repeats&1&Int&Number of replicates to run for this experiment\\
seed&None&Anything&Initial seed for the random number generation at the beginning of the experiment\\
DeltaT&1&Float&Size of a time-step in the simulation\\
StateRecordRate&0.01&Float&Recording interval for snapshots of \url{ReactionVessel} state information (e.g., molecule positions)\\
Visualize&False&Boolean&If \url{True} then display a PyMunk display of molecules, updated each DeltaT\\
ShowForceVectors&False&Boolean&If \url{Visualize} is \url{True} then show lines representing the electro-magnetic forces between \url{ChargedMolecule}s\\
ShowOrientation&True&Boolean&If \url{Visualize} is \url{True} then show a line representing a molecule's orientation\\
\bottomrule
\end{tabular}
\end{center}
\caption{Simulation Parameters in the Experiment Design XML file}
\label{tbl:simulation_parameters}
\end{sidewaystable}

\section{Workflow}\label{workflow}

Each run of an experiment in ToyWorld is usually split into three
stages:

\begin{enumerate}
\def\labelenumi{\arabic{enumi}.}
\item
  Create the overall experiment design in XML - either manually, or by
  modifying an existing design, or by running \url{doe.py} which will
  generate an experiment design from a set of inputs.
\item
  Run the experiments with \url{main.py}, which takes an experiment
  design and executes it with optional logging, saving the data
  generated by each experiment repeat to a separate datafile for later
  analysis.
\item
  Analyse the resulting data either through \url{evaluate.py} or some
  other tool. \url{Evaluate.py} performs the evaluations defined in the
  experiment design, again with options for logging, saving the raw
  analysis output as text files for graphing or further statistical
  examination.
\end{enumerate}

The overall sequence of experiments to run is defined by an experiment
design. The experiments are run in the order they are defined, with all
replicates of an experiment being completed before we move to the
replicates of the next experiment in the sequence.

All replicates of an experiment are run with the same set of parameters
with the exception of a random seed which varies between replicates in a
predictable way. The reactions in the replicate are grouped into
fixed-size blocks and run in sequence with results saved incrementally
between each block. By writing incrementally we attempt to make sure
that a record of at least a partial set of reactions is recoverable
should the experiment come to a premature end.

\section{Evaluation and Analysis module}\label{evaluation-and-analysis-module}

ToyWorld provides a number of Evaluators, both numerical and graphical,
that can be used to analyse the results from each repeat of each
experiment.

The evaluations to perform are listed in the Evaluation section of the
experiment design, for example:

\begin{verbatim}
<Evaluation>
    <Method>evaluators.evaluator_summary.EvaluatorSummary</Method>
        <<Method>evaluators.plot_new_molecule_types.PlotNewMoleculeTypes</Method>
        <Method>evaluators.plot_new_cycle_types.PlotNewCycleTypes</Method>
        <Method partition="False">evaluators.evaluator_spatial_correlation.EvaluatorSpatialCorrelation</Method>
        <Method partition="True">evaluators.evaluator_cycles.EvaluatorActualCycles</Method>
        <Method>evaluators.evaluator_iterations.EvaluatorIterations</Method>
</Evaluation>
\end{verbatim}
