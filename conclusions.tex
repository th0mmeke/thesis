\chapter{Conclusions}\label{thesis-conclusions}

This thesis has covered a number of related topics in \gls{alife}, linked by the theme of creative \gls{oe}. The first has been the connection between evolutionary inheritance and variation, and the necessary conditions for an ongoing system capable of novelty and surprise. This confirmed that for true \gls{oe}, the evolutionary mechanism itself must be capable of evolution. The means for this was explored in the next topic, a model for how inheritance evolves in the presence of variation and selection. The suggested implementation of this model, an ``erroneous copier'' \parencite[p.16]{Zachar2010}, was the final topic, introduced in \cref{part3}. Given the necessary conditions for creative \gls{oe} of an embedded, endogenous evolutionary mechanism, one compatible platform for demonstrating the mechanism is an \gls{achem}. Although a full implementation awaits future work, the ToyWorld \gls{achem} shows emergent behaviour and is demonstrably a plausible first-step towards the final goal.

There are a few significant items for future work. First, the environmental model in \cref{part2} could reasonably be extended to represent a broader variety of environments. Second, it would be interesting to explore an information-based measure of environmental change and perhaps a mechanism for abrupt environmental change (both suggested in \cref{part2-future-work}) to replace the ARIMA parameters currently used in \cref{environmental-model}. Finally, and most significantly, it would be most useful to test the overall hypothesis by the implementation of an evolutionary inheritance and variation mechanism in ToyWorld.

The mechanism implementation could be progressed in two steps. Biological replicases perform this function in living organisms; by starting with a ``shortcut'' implementation of a replicase in the ToyWorld chemistry it would be straightforward to confirm the general principle. However, this assumes two potentially quite difficult preconditions. How straightforward would it be to transpile a replicase (and which replicase?) from real-world chemistry into the ToyWorld chemistry? And second, would such a transpiled replicase function in the same way; that is, is the biological function purely a result of the sequence of atoms into a molecule, or is it dependent upon other factors such as the three-dimensional conformation of the molecule which cannot currently be captured within ToyWorld? Many biological processes are dependent upon side-effects or subtle interactions that are almost certainly missing from our \gls{achem}, and indeed any practical \gls{achem} that is fast enough to be useful. Furthermore, biological replicases are to a great deal contingent upon past evolutionary history and it isn't necessarily straightforward to separate the core elements from the historical ones when doing any transpilation - it may be necessary to translate the whole molecular complex rather than just the minimal core as an unfortunate result.

Therefore any ``shortcut'' based on biology may not prove to be much of a shortcut at all. Therefore, either as an alternative, or as a necessary follow-up if the shortcut is feasible, a native core replicase needs to be constructed from scratch. This is a more challenging problem than faced by previous researchers (\eg \cite{Hickinbotham2011, Nellis2012, Ray1991, Ofria2004, Fontana1992}) who could make use of a \emph{copy} command or its equivalent in their foundational level-0 system. In ToyWorld, and as we have seen, in any system with an evolutionary inheritance and variation mechanism, no such command exists. Instead it must be created (or evolved) \textit{de novo.} In practice we should leverage the evolutionary mechanism itself, based on the results from \cref{part2}, to start with a minimally capable mechanism which would then be improved through evolution--evoevo.

