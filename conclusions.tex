\chapter{Conclusions}\label{ch:thesis-conclusions}

This thesis has explored two related topics in the emergence of evolutionary replicators from artificial chemistries: first, how environmental variability affects heritability in a simple evolutionary model, and second, how the reactions in an artificial chemistry can result in simple replicating entities. The combination of these two topics suggests a pathway towards the formation of full informational replicators in artificial chemistries purely through evolutionary bootstrapping.

We shall now return to our research questions from \cref{research-questions}, and for each question summarize the major results and findings. Then, in order to qualify the scope of the results, we critique some aspects of the work in \cref{sec:limitations}, which naturally leads in to our final \namecref{sec:future-work}, a consideration of future work in \cref{sec:future-work}.

\section{RQ1: In what way does selective pressure drive changes in heredity in a population of evolving replicators?}

This research question served as the primary motivation behind the work documented in \cref{ch:models-of-inheritance}, and in particular for the model relating heritability and fitness described in that \namecref{ch:models-of-inheritance}. By making the usually implicit measures of heritability and fitness explicit for each entity, the model allows us to measure them directly as the population of an otherwise standard evolutionary system evolves. Our first experiment tested the evolution of heritability in a stable environment (without external selective pressure), before we introduced an explicit model of environmental change for the second and final experiment in the \namecref{ch:models-of-inheritance}. 

The changing environmental model incorporates a novel application of an AR-timeseries, combining a regular component and an unpredictable component. At each generation of the evolutionary model, the corresponding value from the timeseries is added to the fitness value of each entity in the population, so efficiently simulating the effects of external environmental change.

\subsection{Main findings}

In summary, the main findings related to RQ1 are:

\begin{enumerate}
\item Heritability is proportional to the predictability of the environment, and is at a minimum in conditions of maximum unpredictability, and at a maximum in stable conditions.
\item The variation in heritability, $\sigma_{heritability}$, is proportional to the degree of environmental variability. As a corollary, the $\sigma_{heritability}$ under changing conditions is greater than that under stable conditions.
\end{enumerate}

These results align with an intuitive understanding of the effects of environmental change: in varying environments, a reservoir of variation provides flexibility, whereas perfect inheritance restricts the variation in a population, and limits the population's ability to respond to change. 

\section{RQ2: Can variable replicators emerge from a molecular artificial chemistry?}

Our approach to this research question is based upon the exploration of a hierarchy of replication proposed by \textcite{Zachar2010}. We began in \cref{ch:toyworld} by describing a modularized \gls{achem} for experimentation, ToyWorld, and conjectured that a count of reaction cycles might serve as a proxy for the number of higher level replicators. We showed that the strategies used to select reactants and products in the \gls{achem} affected the higher level entities (such as reaction cycles) observed in the resulting reaction networks. 

The next two chapters used the ToyWorld \gls{achem} to search for first multipliers and then variable replicators in the reaction graphs that result from a selection of reactant and product selection strategies. We also extended the environmental model from \cref{ch:models-of-inheritance} from one to two dimensions, to include not only the shape or form of the timeseries, but the element of the evolutionary model to receive the change (the target). 

To the single shape of an AR-timeseries from \cref{ch:models-of-inheritance}, we also added a second shape in the form of a bistate series that switches between two alternate values at regular intervals. 

For the target of change, we defined two alternatives, one being kinetic energy (where each molecule's kinetic energy is increased or decreased by the change value) and the other directly modifying the population by adding or removing the number of molecules specified by the change value.

Finally, to detect replicators within the reaction networks, we defined three novel algorithms:
\begin{enumerate}
\item Reaction cycle detection from a set of seed molecules selected by sampling from the molecules in the reaction graph (\cref{alg:identify-cycles}.)
\item Identification of exact multipliers from reaction cycles, based on a specific definition for exact multipliers as two or more copies of the same reaction cycle species, where the reaction cycle species has stoichiometry greater than one, and where each cycle in the multiplier is connected to at least one other multiplier cycle by a molecule that is a product of one cycle and a reactant in the other (\cref{alg:discover-multipliers}.)
\item Identification of variable replicator candidates (that is, without consideration of selection) from reaction cycles, where we define variable replicators as multipliers that can occupy any of a limited set of states without losing their underlying identity (\cref{alg:discover-variable-multipliers}.)
\end{enumerate}

\subsection{Main findings}

The main findings for RQ2 are summarized below.

\begin{enumerate}
\item Exact multipliers do arise in the ToyWorld \gls{achem} (in approximately one-third of all runs), but not in any large number, and when they do, they do not persist for long.
\item Multipliers occur as the result of a non-neutral combination of Product and Reactant selection strategies, and not purely by chance. 
\item The hypothesised relationship between environmental variabilty and the numbers of multipliers remains unproven. 
\item Although ToyWorld produces a number of variable replicator candidates (sequences of repeated reaction cycles), none were observed that meet all of our criteria for variable replication. Specifically, although we observe exact multipliers, and variable replicator candidates, we do not observe candidates that only occupy a restricted or limited set of states, or that exist in multiple runs (see discussion in \cref{sec:limitations})
\item The choice of $S_\mathrm{Reactant}$ has a significant effect on the emergence of reaction cycles in ToyWorld; $S_\mathrm{Product}$ is of lesser effect.
\item A Kinetic Reactant selection strategy is more effective for cycle emergence than a Uniform one. The number of cycles, and length of longest cycle, are maximized under the combination of a Kinetic Reactant selection strategy and a LeastEnergy Product selection strategy.
\item Other parameters to the ToyWorld model have lesser effects on cycle formation; only the relationship of $E_{bonds}$ to the number of cycles produced has any significance.
\end{enumerate}

\section{Limitations}\label{sec:limitations}

Identifying cycles in a reaction network by searching from a sampled set of seed molecules dramatically improves the performance of the cycle detection algorithm, and as shown in \cref{ch:multiplication,ch:variability}, produces sufficient density of cycles to establish the presence of multipliers and variable replicators. However, a significant limitation is that sampling influences the likelihood of detecting all cycles in a multiplier or variable replicator and so the length of a replicator is likely to be underreported. If a cycle in the middle of sequence of cycles is not detected through sampling, the algorithms will identify two shorter replicators instead of one longer one. 

The next topics are closely linked. First, it seems clear from both the experiments in this work, and from our knowledge of early life, that the probability of complex replicators arising within 100,000 generations (around the current limit of practicality for analysis) under the conditions described is extremely low. Some combination of a significant increase in generations and a change of conditions will be required to increase the likelihood of observing a significant step such as the emergence of an informational replicator. At present, the only approach is to conduct many extremely long-duration trials and so leverage probabilities. This is clearly unsatisfactory.

Second, the current work does not provide any guidance as to how that step from a variable to an template-based informational replicator might arise, and yet it is essential if complex replicators are to form. 

Third, our current algorithms rely upon network analysis at the reaction cycle level, and cannot inherently detect any higher-level structure. The distinction between genotype and phenotype in an informational replicator occurs at a different conceptual level to the component reactions, and our level of interest and investigation needs to change accordingly. We need an approach that adapts to different levels of emergence, from cycles to elements built from cycles, to yet more complex elements, and so on. 

Finally, the current approach of graph analysis of complete reaction networks, even dynamic analysis, cannot scale to the network sizes needed. It seems clear that either the performance of the current algorithms must be dramatically improved, perhaps by rewriting in a lower-level higher-performance computer language, or more profitably, the approach to analysing the generated data must change if larger networks are mandated. The graph structure we generate at present forms a single connected component in which every molecule and reaction are contained (with the exception of those molecules that never take part in a reaction). The lack of obvious substructures within this single graph makes it difficult to naturally subset the graph to improve the speed of the analysis.

\section{Future work}\label{sec:future-work}

The previous \namecref{sec:limitations} has identified issues primarily with the scope of the present work. In this \namecref{sec:future-work}, we concentrate on those improvements that could be made within the current scope. 

Although the AR-timeseries generator described in \cref{environmental-model} produces a time series for environmental change with the property of stationarity, the $\delta$ term makes the evolutionary model of fitness non-stationary. However, any change still remains steady and gradual. An extension would be to co-opt the idea of concept drift from time series analysis to induce an abrupt change with probability $p$ at each generation. Each change would therefore form a new `concept`. Instead of the environment changing in a predictable and describable way from one generation to another, the change could not be predictable from the earlier history.

There are some obvious extensions of the model from \cref{base-model} that have been left for future work. First, the model currently assumes only single-parent inheritance, whereas many biological species have two parents. Extending to two parents would be a useful enhancement to increase the model's scope. Second, the model does not include any influence from development (the production of the phenome from the genome). However, it is unclear at this stage what effect development would have on the model as its effects are bundled into the overall \emph{fitness} parameter. Finally, although outside of the overall scope of this work in evolutionary systems, the effect of acquired characteristics would be interesting to explore. Others (\eg \textcite{Gaucherel2012,Paenke:2007ie,Sasaki:2000dq}) have studied the differences between general models based on acquired and non-acquired characteristics, finding a difference between models in changing environments. This could be another area of exploration for the future.

A limitation of the experiment design in \cref{ch:toyworld} is that the values chosen for the high and low values of $E_\mathrm{Bonds}$ make it impossible to determine the cause of the difference observed in \cref{RQ2.3}. There are two alternative explanations: first, the energy required to make or break bonds is simply different between the two factor levels; second, in the low factor level, based on real-world values, the bond make and break energies for even a single bond vary depending on the atoms involved, while in the high factor level these values are consistent for all bonds of the same degree. To distinguish between the two explanations, the average levels at each degree should be the same for each factor; this is a suggestion for a future experiment.

As mentioned in \cref{sec:limitations}, the sampling algorithm for cycles means that the sizes of replicators may be underestimated. The sampling proportion $p$ could certainly be increased, but this is currently impractical for large reaction graphs. Alternatively, repeating the cycle detection with a new set of seed molecules could eliminate any sampling gaps. After identifying an initial set of replicators, the seed molecules for the repeated cycle dection should include the product and reactant molecules from the replicator cycles that are not consumed or produced, respectively, by the replicator.

In \cref{ch:variability} we have assumed that each environmental change affected all entities equally; however, this isn't necessarily the only option. We can identify three levels of scope, or the proportion of entities in the population to receive a particular set of changes, from most homogeneous to least:

\begin{enumerate}
	\item The group of all entities. All entities receive the same set of changes.
	\item A group for each set of ``related'' entities, where the most natural and obvious relation is that between parent and child; this is unambiguous and straightforward in our model where each entity has only one parent. We refer to a group of entities related by inheritance as a \emph{lineage}. A separate set of changes is provided for each lineage.
	\item A single-member group for each entity. Each entity receives a unique set of changes.
\end{enumerate}

The first level is the simplest application of environmental change, and the one adopted throughout this work, while the second represents the common scenario where we expect similar entities to react in similar ways to change, and where similarity is a result of descent: entities that share a common ancestor are more similar to each other than they are to other lineages. 

The third scope level implies that each entity has an independent response to environmental changes. This seems problematic; environmental response is a function of phenotypes, and we would expect related entities to have related phenotypes\footnote{In general, although in biology there can be significant phenotypic differences between related entities.}. Thus instead of single-member groups we would expect lineage-related groups.

\section{Personal reflection}

The work in this thesis has been driven by a belief that the most effective approach to replication was likely to be one in which replicators could emerge from a simpler artificial chemical system without explicit external design or direction. The experiments described in this work have therefore followed a ``big data'' approach, but given limited resources, they have been only to an exploratory scale. Given the probabilities involved, it is clear that ideally we would greatly increase the number of trials, or reactions simulated.

This could be done by rearchitecting the current implementation to use cloud-based computing resources in parallel. Running multiple experiments in parallel, with cycle, multiplier and variable replicator detection happening in real-time from streamed results would effectively remove the current limits on the number of reactions that can be modelled and analysed.



