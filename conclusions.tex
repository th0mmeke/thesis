\chapter{Conclusions}\label{thesis-conclusions}

This thesis has explored two related topics in the emergence of evolutionary replicators from artificial chemistries: first, how environmental variability affects heritability in a simple evolutionary model, and second, how the reactions in an artificial chemistry can result in simple replicating entities. The combination of these two topics suggests a pathway towards the formation of full informational replicators in artificial chemistries purely through evolutionary bootstrapping.

\section{What is the nature of the relationship between heredity and selective pressure for a \emph{theoretical} replicator?}

Model relating heritability and fitness

\emph{AR-timeseries}, described by three parameters $\Theta$, $\sigma_e$ and $\delta$. $\Theta$ is the AR coefficient, $e_t$ is a random, normally distributed, error component around a mean of $0$, where $e_t\stackrel{iid}{\sim}N(0,\sigma^{2}_e)$, and $\delta$ is a fixed bias value. An AR-timeseries combines a regular component (described by $\Theta$ and $\delta$) and an unpredictable component ($\sigma_e$.) Unlike the bistate case, the AR-timeseries is neither constant nor completely regular.

\subsection{Main findings}

\begin{enumerate}
\item Heritability is proportional to the predictability of the environment: at a minimum in conditions of maximum unpredictability, and at a maximum in stable conditions.
\item Variation in heritability, $\sigma_{heritability}$, is proportional to the degree of environmental variability. As a corollary, the $\sigma_{heritability}$ under changing conditions will be greater than that under stable conditions.
\end{enumerate}

\section{What elements of replication can be \emph{demonstrated} in a molecular artificial chemistry?}

Evolutionary model with two main elements, reactant selection and product selection.

We extended the environmental model from RQ1 from one to two dimensions, now not only shape but target. We also added a second shape or form: \emph{Bistate} selective pressure, where the shape of change follows a pattern that switches between two alternate states at regular intervals. 

As for the target of change, we begin with two defined alternatives:

\begin{itemize}
	\item \emph{KE}, where each molecule's kinetic energy is increased by the change value of $\delta$: $KE_{new} = KE+\delta$. As seen in \cref{ch:toyworld}, the population composition is influenced by the kinetic energy available for reactions. Changing the KE of the reaction vessel should indirectly result in a change in reactor population, and hence influence the types of reactions and reaction cycles that develop.
	\item \emph{Population}, in which the $\delta$ value is used to adjust the population size. If $\delta$ is negative, then $|\delta|$ molecules chosen with uniform probability are removed from the population; if $\delta$ is positive, then $\delta$ molecules chosen again with uniform probability from the initial population are added to the current population. In summary, environmental changes will directly affect the composition of the reactor population, once again influencing the relative proportions of reaction cycle species.
\end{itemize}

Sampling algorithm for cycle detection in a reaction network.
Algorithm to identify exact multipliers from reaction cycles, where we developed a specific definition for exact multipliers as \textit{Two or more} copies of the \textit{same reaction cycle species}, where the reaction cycle species has \textit{stoichiometry greater than one}, and where \textit{each cycle in the multiplier is connected to at least one other multiplier cycle} by a molecule that is a product of one cycle and a reactant in the other.
Identify variable replicator candidates (without consideration of selection) from reaction cycles, where our interpretation of a variable replicator is a multiplier that can occupy any of a limited set of states without losing its underlying identity.

\subsection{Main findings}

\begin{enumerate}
	
\item The choice of $S_\mathrm{Reactant}$ is critical to the behaviour of this \gls{achem}; $S_\mathrm{Product}$ on the other hand appears to have a lesser effect on the emergence of cycles in our experiments.

\item Furthermore, $S_\mathrm{Reactant} = \mathrm{Kinetic}$ is more effective for cycle emergence than $S_\mathrm{Reactant} = \mathrm{Uniform}$. The number of cycles, and length of longest cycle, are both maximized with the combination of a Kinetic Reactant selection strategy and a LeastEnergy Product selection strategy.

\item Other parameters to the model, such as $E_{Vessel}$, the initial kinetic energy of the molecular population, and $E_{bonds}$, the bonding model, had lesser effects on cycle formation, with the only significant relationship being that of $E_{bonds}$ on the number of cycles produced.

\item Exact multipliers do arise in the ToyWorld \gls{achem}, but not in any great numbers, and when they do, they do not persist for long. We have shown though that they do occur as the result of a non-neutral combination of Product and Reactant selection strategies, and not purely by chance. It is also apparent that our earlier supposition, that the number of reaction cycles would be a good proxy or predictor for multiplier or replicator activity, is not supported by the evidence.

\item Although the \gls{achem} produced exact multipliers in approximately one-third of all runs, a relationship between environmental variabilty and the numbers of multipliers was unproven. Similarly, although the \gls{achem} generated a number of variable replicator candidates--limited sets of repeated reaction cycles--none of the candidates were other than endemic to a single run. As the condition for a full variable replicator is that each state should be equivalent under selection, the lack of alternative forms in other runs prevents us from properly testing the condition. However, we would have expected a variable replicator to oscillate between a set of states while our candidates never repeated the same state, and so on balance we feel that the results cannot be other than inconclusive.

\end{enumerate}

\section{Limitations}\label{sec:limitations}

Identifying cycles in a reaction network by searching from a sampled set of seed molecules dramatically improves the performance of the cycle detection algorithm, and as shown in \cref{ch:multiplication,ch:variability}, produces sufficient density of cycles to establish the presence of multipliers and variable replicators. However, a significant limitation is that sampling influences the likelihood of detecting all cycles in a multiplier or variable replicator and so the length of a replicator is likely to be underreported. If a cycle in the middle of sequence of cycles is not detected through sampling, the algorithms will identify two shorter replicators instead of one longer one. 

The next topics are closely linked. First, it seems clear from both the experiments in this work, and from our knowledge of early life, that the probability of complex replicators arising within 100,000 generations under the conditions described is extremely low. Some combination of a significant increase in generations and a change of conditions will be required to increase the likelihood of observing a significant step such as the emergence of an informational replicator. At present, the only approach is to conduct many extremely long-duration trials and so leverage probabilities. This is clearly unsatisfactory.

Second, the current work does not provide any guidance as to how that step from a variable to an template-based informational replicator might arise, and yet it is essential if complex replicators are to form. 

Third, our current algorithms rely upon network analysis at the reaction cycle level, and cannot inherently detect any higher-level structure. The distinction between genotype and phenotype in an informational replicator occurs at a different conceptual level to the component reactions, and our level of interest and investigation needs to change accordingly. We need an approach that adapts to different levels of emergence, from cycles to elements built from cycles, to yet more complex elements, and so on. 

Finally, the current approach of graph analysis of complete reaction networks, even dynamic analysis, cannot scale to the network sizes needed. It seems clear that either the performance of the current algorithms must be dramatically improved, perhaps by rewriting in a lower-level higher-performance computer language, or more profitably, the approach to analysing the generated data must change if larger networks are mandated. The graph structure we generate at present forms a single connected component in which every molecule and reaction are contained. The lack of obvious substructures within this single graph makes it difficult to naturally subset the graph for speed of analysis.

\section{Future work}\label{sec:future-work}

The previous \namecref{sec:limitations} has identified issues primarily with the scope of the present work. In this \namecref{sec:future-work}, we concentrate on improvements that could be made within the current scope. 

Although the AR-timeseries generator described in \cref{environmental-model} produces a time series for environmental change with the property of stationarity, the $\delta$ term makes the evolutionary model of fitness non-stationary. However, any change still remains steady and gradual. An extension would be to co-opt the idea of concept drift from time series analysis to induce an abrupt change with probability $p$ at each generation. Each change would therefore form a new `concept`. Instead of the environment changing in a predictable and describable way from one generation to another, the change could not be predictable from the earlier history.

There are some obvious extensions of the model from \cref{base-model} that have been left for future work. First, the model currently assumes only single-parent inheritance, whereas many biological species have two parents. Extending to two parents would be a useful enhancement to increase the model's scope. Second, the model does not include any influence from development (the production of the phenome from the genome). However, it is unclear at this stage what effect development would have on the model as its effects are bundled into the overall \emph{fitness} parameter. Finally, although outside of the overall scope of this work in evolutionary systems, the effect of acquired characteristics would be interesting to explore. Others (\eg \textcite{Gaucherel2012,Paenke:2007ie,Sasaki:2000dq}) have studied the differences between general models based on acquired and non-acquired characteristics, finding a difference between models in changing environments. This would be another area of exploration for the future.

The most significant limitation of the experiment design in \cref{ch:toyworld} is that the values chosen for the high and low values of $E_\mathrm{Bonds}$ make it impossible to determine the cause of the difference observed in \cref{RQ2.3}. There are two alternative explanations: first, the energy required to make or break bonds is simply different between the two factor levels; second, in the low factor level, based on real-world values, the bond make and break energies for even a single bond vary depending on the atoms involved, while in the high factor level these values are consistent for all bonds of the same degree. To distinguish between the two explanations, the average levels at each degree should be the same for each factor; this is a suggestion for a future experiment.

As mentioned in \cref{sec:limitations}, the sampling algorithm for cycles means that the sizes of replicators may be underestimated. The sampling proportion $p$ could certainly be increased, but this is currently impractical for large reaction graphs. Alternatively, repeating the cycle detection with a new set of seed molecules could eliminate any sampling gaps. After identifying an initial set of replicators, the seed molecules for the repeated cycle dection should include the product and reactant molecules from the replicator cycles that are not consumed or produced, respectively, by the replicator.

In \cref{ch:variability} we have assumed that each environmental change affected all entities equally; however, this isn't necessarily the only option. We can identify three levels of scope, or the proportion of entities in the population to receive a particular set of changes, from most homogeneous to least:

\begin{enumerate}
	\item The group of all entities. All entities receive the same set of changes.
	\item A group for each set of ``related'' entities, where the most natural and obvious relation is that between parent and child; this is unambiguous and straightforward in our model where each entity has only one parent. We refer to a group of entities related by inheritance as a \emph{lineage}. A separate set of changes is provided for each lineage.
	\item A single-member group for each entity. Each entity receives a unique set of changes.
\end{enumerate}

The first level is the simplest application of environmental change, and the one adopted earlier in this \namecref{ch:variability}, while the second represents the common scenario where we expect similar entities to react in similar ways to change, and where similarity is a result of descent: entities that share a common ancestor are more similar to each other than they are to other lineages. 

The third scope level implies that each entity has an independent response to environmental changes. This seems problematic; environmental response is a function of phenotypes, and we would expect related entities to have related phenotypes\footnote{In general, although in biology there can be significant phenotypic differences between related entities.}. Thus instead of single-member groups we would expect lineage-related groups.

\section{Concluding remarks}

