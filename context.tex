\chapter{Context}\label{context}

\epigraph{%
In the world of human thought generally , and in physical sciences in particular, the most important and most fruitful concepts are those to which it is impossible to attach a well-defined meaning.}%
{\textsc{H.A. Kramers, quoted in \autocite{Bruylants2010}}}

The major division is between living and artificial systems. Although as \autocite{Bruylants2010} comments, any field of origins has problems with
either/ors -- almost certainly there is a stage where an organism, for example, is not non-living, but not yet living.

 \autocite{Pascal2013}: details of early life may be unknown and unknowable, but principles possible. Sudden transition to life like today astronomically unlikely - single RNA strand capable of ribozyme activity probability of 10E-60\ldots{} (for 100 monomers). The more likely alternative is that there was a series of intermediates, of increasing degrees of ``aliveness''. The corollary is that it is hard therefore to imagine a clear cut transition between non-living and living.

\section{What is life?}\label{what-is-life}

One distinction between living and non-living comes from \autocite{Rasmussen2004} -- non-living systems explore a state-space driven by thermodynamics, and so in a sense through a random ergodic search. Living systems however almost universally employ evolution. Another information-theoretic view, from \autocite{Adami2015}, is that living systems can preserve information on a much longer timescale than non-living things.

			metabolism = how ``living matter evades the decay to equilibrium''
			(Schrodinger 1944) in \autocite{Pascal2013}

			Definition of Life  \autocite{Pascal2013}: Elusive, and probably not useful
			
			
						\autocite{Fernando:2007pf}: Avoids definition of life - instead presents goal as any one of
						\begin{itemize}
							\item
							
							OEE - Bedau 2000
							
							\item
							
							basic autonomy - Ruiz-Mirazo 2004
							
							\item
							
							production of modular replicators with unlimited heredity potential
							Maynard-Smith and Szathmary 1995/Szathmary 2000
							
							\item
							
							course of evolution by which\ldots{} Oparin 1964
							
							\item
							
							coupled cycling of bioelements Morowitz 1968 1971
							
							\item
							
							maximization of energy production by a biosphere Kleidon 2004
							
							\item
							
							minimal unit of life Ganti 2003a
							
							\item
							
							autopoetic unit Maturana and Varela 1992
							
						\end{itemize}
						
Autopoesis

Self-replication

Rosen (M,R) systems

Ganti Chemoton

Autonomy

Or \quote{
	What modifications must be made to this type of
	experiment to allow at least one of the following outcomes:
	‘open-ended evolution’ (Bedau et al., 2000); the origin of
	basic autonomy, i.e. a dissipative system capable of the
	recursive generation of functional constraints (Ruiz-Mirazo
	et al., 2004); a process ultimately capable of the
	production of nucleic acids or other modular replicators
	with unlimited heredity potential (Maynard-Smith and
	Szathmary, 1995; Szathmary, 2000); identification of ‘‘the
	course of evolution by which the determinate order of
	biological metabolism developed out of the chaos of intercrossing
	reactions’’ (Oparin, 1964); the coupled cycling of
	bioelements (Morowitz, 1968, 1971); the maximization of
	entropy production by a biosphere (Kleidon, 2004); the
	minimal unit of life (Ganti, 2003a, b); or an autopoetic unit
	(Maturana and Verela, 1992)?}{\autocite{Fernando:2007pf}}

The arrow of complexity - explored in \autocite{Miconi:2008cy}:
\begin{itemize}
	\item Passive (inevitable result of non-repeating evolution from simple seed)
	\item Active--some drive towards increasing complexity
\end{itemize}

\section{Biological evolution}\label{evolution}

\quote{Evolution is a process that results in heritable changes in a population spread over many generations.}{
	``Sandwalk: strolling with a skeptical biochemist'',
	\url{http://sandwalk.blogspot.co.nz/2012/10/what-is-evolution.html}}

\quote{
	Biological evolution consists of change in the hereditary characteristics of groups of organisms over the course of generations. Groups of organisms, termed populations and species, are formed by the division of ancestral populations or species, and the descendant groups then change independently. Hence, from a long-term perspective, evolution is the descent, with modification, of different lineages from common ancestors.}{
	``Evolution, Science, and Society: Evolutionary Biology and the National Research Agenda'', Working Draft, 28 September 1998, \url{http://www.zoology.ubc.ca/~otto/evolution/Evolwhite.pdf}}

Separate outcome or result from process or mechanism (e.g., adaptation)

Blurring in use of OEE

\begin{itemize}
	\item Either in sense of quote{an indefinitely large number of structures are each capable of replication.}{\autocite{MaynardSmith1999}}, or
	\item Additionally meaning novel, interesting, surprising
\end{itemize}

Ongoing generation of novel forms

Inevitably leads to increasing complexity as without complexity will exhaust new possibilities and cover old ground

Our problem seems tightly linked to life--requirements for OEE (hypothesised) look a lot like requirements for Life (which is only known OEE system, so perhaps not so surprising)

What elements of life are required for OEE, and which are not?

\begin{itemize}
	\item
	      Exact elements (same thing)--e.g., evolution?
	\item
	      Equivalent elements (same concept/function/purpose, different detail) e.g., compartments/membranes?
	\item
	      Not required at all e.g., specific metabolic cycles
\end{itemize}
\subsection{Evolution by Natural Selection (ENS))}\label{ens-evolution-by-natural-selection}


		``Owing to this struggle for life, any variation, however slight and
		from whatever cause proceeding, if it be in any degree profitable to
		an individual of any species, in its infinitely complex relations to
		other organic beings and to external nature, will tend to the
		preservation of that individual, and will generally be inherited by
		its offspring. The offspring, also, will thus have a better chance
		of surviving, for, of the many individuals of any species which are
		periodically born, but a small number can survive. I have called
		this principle, by which each slight variation, if useful, is
		preserved, by the term of Natural Selection, in order to mark its
		relation to man's power of selection (Darwin 1859, p. 61).'' \autocite{Griesemer2005}
		
ENS is an example of OEE
ENS is strongly tied to life, and mostly presupposes living subjects.

	Greisemer 2000 states that the elements of evolution in the Lewontin (variation, fitness differences, heritability of fitness) and Maynard-Smith (variation, multiplication, heritability) formulations are fundamentally different (from \autocite{Vasas2015}).


``Darwin's theory of evolution by natural selection is restricted in scope. One sense in which it is restricted is that it refers to organisms.'' \autocite{Griesemer2005}

Restricted by \autocite{Griesemer2005}:
\begin{itemize}
	\item natural selection, not drift or Lamarckian (biased) inheritance
	\item organisms not defined - so scope vague. Many treat vertebrates as paradigm, despite rarity
	\item organismal level - but many other levels
\end{itemize}


\autocite{Godfrey-Smith2007}:

\begin{itemize}
	\item
	
	Purposes of summaries vary - typically either pedagogy, defending
	evolutionary theory, extensions to other domains (e.g., cultural), and
	have ``intrinsic scientific and philosophical interest as attempts to
	capture some core principles of evolutionary theory in a highly
	concise way.''
	
	\item
	
	Subject to counterexamples, and connection to formal models ``not
	straightforward''
	
	\item
	
	Core requirement for ENS from various summaries is ``combination of
	variation, heredity, and fitness differences.''
	
	\item
	
	Most commonly cited summary is Lewontin1970 - but unusual in that says
	``fitness is heritable'' - typically phenotypic heredity is sufficient
	for trait to evolve. 1980 version more typical
	
	\item
	
	Endler1986 and Ridley1996 discussed
	
	\item
	
	Necessary and sufficient ambiguous in any summary - might mean will
	result in ENS when know what ENS is, or discriminatory - this process
	is ENS. That is, constitutive or causal readings. E.g., conditions for
	``becoming pregnant\ldots{}{[}versus{]} being pregnant''
	
	\item
	
	Usual aim is to ``give conditions that are sufficient ceteris paribus
	for a certain kind of change occurring.'' where the change is assumed
	to be wrt some trait
	
	\item
	
	Problem cases for summaries
	
	
	\begin{itemize}
		\item
		
		Culling - no reproduction. If ENS, then no heredity necessary
		
		\item
		
		Different generation times - only difference is reproductive rate
		
		\item
		
		Biased inheritance
		
		\item
		
		Heritability fails in the fit
		
		\item
		
		Stabilizing selection in an asexual population
		
		\item
		
		Covariance positive with no variation in Z
		
		\item
		
		Accident
		
		\item
		
		Correlated Response
		
	\end{itemize}
\end{itemize}


http://www.protevi.com/john/Morality/evolution4dimensions.pdf--outline of Jablonka, summary of evolutionary ``theory'' as of 2005

\begin{itemize}
	\item
	      Biology based around ENS e.g., \quote{god and natural selection are, after
	      	all, the only workable theories we have of why we exist}{\autocite{Dawkins1982}}
	\item
	      But later biology includes HGT (horizontal gene transmission) which
	      blurs mechanism somewhat, as does importance of neutral-theory,
	      without invalidating ENS
	\item
	      Summaries and formal models attempting to either capture definition
	      (constitutive) or sufficiency (causal) of observed systems
	\item
	      ENS (Evolution by Natural Selection) in biology needs variation,
	      fitness differences, heritability of fitness (Lewontin) or variation,
	      multiplication, heritability (Maynard-Smith)
	\item
	      See \autocite{Griesemer2001} for discussion of differences
	\item
	      Review of various formulations in Godfrey-Smith2007; all from starting
	      point of natural evolution -\textgreater{} summary -\textgreater{}
	      reconcile, address specific difficulties in particular formulations
	      against common problem cases
	\item
	      Core is \quote{``combination of variation, heredity, and fitness
	      	differences.}{\autocite{Godfrey-Smith2007}}
	\item
	      HGT (Horizontal gene transfer)
	\item
	      Thought that biological evolution is Evolution by Natural Selection,
	      but this is not strictly true:
	\item
	      \quote{We take it as given that biology
	      	instantiates ENS {[}Evolution by Natural Selection{]}. That is, ENS
	      	occurs in biological evolution (there is no need to reiterate the
	      	evidence for this). However, we wish to separate the conclusion that
	      	ENS \textit{occurs in} biological evolution from the conclusion that
	      	the algorithm of adaptive biological evolution \textit{is} ENS}{\autocite{Watson2012}}
	      (emphasis in original)
	\item
	      LGT complicates this
	\item
	      Exaptations more common than thought, meaning that evolutionary driver
	      shifting towards neutrality from pure selection \autocite{Barve2013}
	\item
	      Junk DNA/Neutral theory--not a highly selective environment
	      (http://sandwalk.blogspot.co.nz/2008/02/theme-genomes-junk-dna.html)
	\item
	      Random genetic drift
	\item
	      Problems with ENS beyond classical organisms
	\item
	      Previous work in application of ENS to levels of biological hierarchy
	      (Griesemer2005)
	\item
	      issue in identifying ``individuals'' or ``entities'' in ENS
	      formulations--part of the ``unit of selection'' problem
	\item
	      issue in meaning of heredity without genotypes--needs a causal
	      formulation for pre-cellular evolution akin to that in
	      \autocite{Bourrat2015} or presented in \autocite{Griesemer2005} for
	      Weismannian causal relationship between genes and organisms
\end{itemize}

``We take it as given that biology instantiates ENS'' - but that doesn't mean that the algorithm of biology is ENS \autocite{Watson2012}.

\subsection{Evolvability of Evolvability}

Pigliucci2008 - ``Is evolvability evolvable?''

\autocite{Watson2015}:
\begin{itemize}
	\item
	
	Darwinian machine is fundamentally self-referential - products of
	evolution affect process of evolution - lots of examples
	
	
	\begin{itemize}
		\item
		
		Major transitions in evolution not possible without
		self-referentiality - unit of evolution must change between levels
		e.g., from molecules to cells. This was/is accomplished by changing
		the way that individuals interact, from competition to cooperation
		(fitness change) to form the next level. How does selection at one
		level get suppressed, and introduced at the next?
		
		\item
		
		Maynard-Smith/Szathmary
		
		\item
		
		Independent replication before transition, must replicate as part of
		a larger whole afterwards
		
		\item
		
		Fundamental for complexity - complexity associated with levels
		
		\item
		
		Draws analogy with Hinton 2008 Deep Learning
		
		\item
		
		At one level, supervised learning improving adaptation; at next
		higher, unsupervised improving robustness
		
		\item
		
		Modes of inheritance (for groups)
		
		
		\begin{itemize}
			\item
			
			``migrant pools'' - particles disperse horizontally and reform
			into new pools - type 1 group selection - no group inheritance
			
			\item
			
			Group fissioning - vertical inheritance - group inheritance
			(differences are heritable) - type 2
			
		\end{itemize}
	\end{itemize}
\end{itemize}



\autocite{Griesemer2001}
\begin{itemize}
	\item
	
	examination of difference between Lewontin and Maynard Smith's views
	of units of selection/evolution
	
	\item
	
	Maynard-Smith
	
	
	\begin{itemize}
		\item
		
		``if there is a population of entities with multiplication,
		variation and heredity, and if some of the variations alter the
		probability of multiplying, then the population will evolve.
		Further, it will evolve so that the entities come to have
		adaptations....''
		
		\item
		
		M,V,H necessary for units of evolution. F sufficient for evolution
		of adaptations (with proviso in text)
		
		\item
		
		Units of evolution that have F are units of selection. Units of
		selection though do not imply units of evolution as not all MVHF
		necessary.
		
	\end{itemize}
	\item
	
	Lewontin
	
	
	\begin{itemize}
		\item
		
		phentotypic variation, differential fitness, fitness is heritable
		(by generalizing from Darwin 1859 p61) ``Owing to this struggle for
		life...Natural Selection\ldots{}''
		
		\item
		
		``These principles mention quantities that describe quantitative
		roles of causal capacities (Woodward, 1993)''
		
	\end{itemize}
	\item
	
	Darwin's concept of inheritance in 1859 p61 includes heritability (a
	capacity) and inheritance (a process carrying the capacity)
	
	
	\begin{itemize}
		\item	
		Lewontin stresses first, assuming second, and Maynard Smith's
		multiplication is about the second; his heredity is both.
		
	\end{itemize}
\end{itemize}


\subsection{Lamarck}\label{lamarck}

Inheritance of acquired characteristics--e.g., epigenetics, but probably not significant as evolutionary mechanism in biology. But a viable non-biological mechanism

\subsection{Origins of life}

			Oparin 1924 scientists explore the origin of life ``like two parties
			of workers boring from the two opposite ends of a tunnel'' - chemical
			bottom-up and biological top-down approaches in \autocite{Pereto2012}
			
			
			``What we do not know today we shall know tomorrow. A
			whole army of biologists is studying the structure and organization of
			living matter, while a no less number of physicists and chemists are
			daily revealing to us new properties of dead things. Like two parties
			of workers boring from the two opposite ends of a tunnel, they are
			working towards the same goal. The work has already gone a long way
			and very, very soon the last barriers between the living and the dead
			will crumble under the attack of patient work and powerful scientific
			thought.'' (\href{http://www.valencia.edu/~orilife/textos/The\%20Origin\%20of\%20Life.pdf}{\emph{http://www.valencia.edu/\textasciitilde{}orilife/textos/The\%20Origin\%20of\%20Life.pdf}})
			
Similar problem--from chemistry to biology. Transition to biology.

Systems Chemistry is tightly related field.

Analogous to origin of life, but despite parallels, must resist temptation to extend claims to this--the evolution of life was contingent, and because of lack of evidence from early stages, no way anyway to test or confirm.

Work to describe a pathway from plausible conditions on the early abiotic Earth to the first living things. Important to understand that this pathway only describes one possible form of life, and one possible mechanism -- restricted in scope to connecting the two endpoints.

The primary postulates of OOL are not our postulates. We do though have concepts and principles in common - meta-ENS, or the workings of evolutionary processes, of which the mechanisms of OOL are one concrete example (a form of existence proof.)

\begin{itemize}
	\item
	      hypercycles \autocite{Eigen1971}
	\item
	      autocatalysis
	      \begin{itemize}
	      	\item
	      	      present in all three elements (\autocite{Ganti:2003hl}--or
	      	      earlier?) of life
	      	\item
	      	      DNA replicates with enzymatic help
	      	\item
	      	      some metabolites, such as ATP, exclusively autocatalytic
	      	\item
	      	      lipids in a membrane enhance addition of other lipids--Formation of
	      	      autocatalytic cycles/sets (\autocite{Hordijk2004})--percolation increases
	      	      likelihood
	      	\item
	      	      \autocite{Sousa2015} for detection of RAF sets in e coli metabolic
	      	      networks
	      	\item
	      	      Selection amongst autocatalytic networks (Ganti and Wachtershauser referenced in \autocite{Fernando:2005ly})
	      	\item
	      	      \autocite{Fernando:2007pf}--selection/liposomes/chemical avalanches
	      	      based on Wachtershauser
	      	\item
	      	      autocatalytic cores (\autocite{Vasas2012})
	      	\item
	      	      Kauffman's original Reflexively Autocatalytic Polymer Networks
	      	      (RAPN) \autocite{Kauffman1986,Farmer1986} are not capable of
	      	      non-digital evolution. RAPN stabilise into single network without
	      	      variation
	      \end{itemize}
	\item
	      GARD (fixed catalysts) \autocite{Segre1998} not capable of evolution -
	      lack heredity of variation (mutations overwhelm heredity) \autocite{Vasas2010}
	\item
	      Bimolecular rearrangements (\autocite{Fernando:2008xy,Fernando:2007pf})?
	\item
	      template replicators--highly unlikely without intermediate steps
	\item
	      Must be driven far-from-equilibrium (many references e.g.,
	      \autocite{Pascal2015}--continuous supply of energy required (explicit
	      modelling unlike say \autocite{Fontana1994} where energy only
	      implicitly modelled)--and maintained there (by metabolism--e.g., how
	      \quote{living matter evades the decay to equilibrium}{\autocite{Schrodinger1944}})
	\item
	      Eigen threshold for replicators--high mutation rates overwhelm
	      heredity
	\item
	      Biological evolutionary theory
	\item
	      Extension beyond biology
	      	
	      \begin{itemize}
	      	\item
	      	      Previous work on extending evolution--\autocite{Bourrat2015} etc
	      	\item
	      	      Evolution of culture, language, technology
	      \end{itemize}
	\item
	      OOL is focussed on plausible explanations for life-as-we-know-it
	\item
	      And must be constrained by biological givens, i.e.
	\item
	      Starting point consistent with what is known about archaic Earth
	\item
	      Must lead to end-point consistent with earliest known life
	\item
	      In a reasonable time period
	\item
	      single-step astronomically unlikely (single RNA strand probability
	      about 10E-60, based on 100 monomers--\autocite{Pascal2013})
	\item
	      Requires a series of steps--akin to OOL where single-step
	      astronomically unlikely (single RNA strand probability about 10E-60,
	      based on 100 monomers--\autocite{Pascal2013})
	\item
	      Contingent and specific--constraints
	\item
	      starting point compatible with what is known of prebiotic conditions
	      (either on earth or extraterrestrially)
	\item
	      end point of ENS at something that might be LUCA
	\item
	      Must be simplified and abstracted
	\item
	      Choosing an artificial chemistry similar to natural chemistry enables
	      an argument by analogy
	\item
	      Meets known constraints for OEE--and we have OOL as an example of OEE
	      from natural chemistry
	\item
	      Catalysis/autocatalysis possible through emergence in some AChems
	      (\eg \autocite{Virgo2013})
	\item
	      Some fundamental differences remain however between our domain and the
	      natural domain
	\item
	      The sheer size of the Biosphere means we can never duplicate the
	      number of individual evolutionary ``trials'' (selection, mutation and
	      reproduction) events in our model
	\item
	      The Biosphere is underpinned by a phenomenally rich set of physical
	      and chemical laws, implicitly constraining each and every action and
	      interaction
\end{itemize}


\subsection{Synthetic biology}\label{synthetic-biology}

Provides a perspective on living things - one answer to the question of what is required for something to be alive.

ADD IN NOTES

Minimal cells

\begin{itemize}
	\item
	      Minimal cell created either by removing elements from simple cell to
	      produce functioning minimal cell or alternatively by synthesising cell
	      from bottom-up from a set of hypothesised necessary elements.
	\item
	      ``chemical reaction networks coupled to containers''--protocell
	      definition, from Protocells:Back to the Future workshop materials
	      (http://www.unamur.be/en/sci/naxys/pb2f).
	\item
	      Containers useful for protection, for concentration of elements, and
	      for selection, allowing benefits to accrue to originator.
	\item
	      Ganti's observation that contemporary living things always have a
	      metabolic subsystem, a heritable control system, and a boundary system
	      to contain (in \autocite{Szathmary:2006ty}).
\end{itemize}

\section{Compositional evolution}

Random search--Random search/walk will explore possibilities, but without meaning\ldots{}

In life, have HGT e.g., \autocite{Ochman2000}

In general though, compositional evolution does not attempt to understand life, but instead as a guide to achieving similar property in another domain, or in understanding similar processes in another domain \autocite{Arthur2009}

\autocite{Watson2002} discusses compositional evolution and building blocks

\autocite{Pross2011}--Dynamic Kinetic Stability

\autocite{Arthur2009} investigates the evolution of
technology, where evolution is used in the sense of \quote{all objects of
	some class are related by ties of common descent from the collection
	of earlier objects.}{\autocite{Arthur2009}}
\begin{itemize}
		
	\item
	      Evolution in technology occurs by using earlier technologies as
	      building blocks in the composition of new technologies, and these new
	      technologies then become building blocks for use in later
	      technologies, and so on. Arthur calls this ``combinatorial
	      evolution.'' But what is the starting point? How is this regression
	      grounded? Arthur proposes that the capture and harnessing of natural
	      phenomena starts each lineage, and provides new raw components for
	      inclusion in later technologies.
	\item
	      Evolution is related to innovation: in fact, Arthur claims that by
	      understanding the mechanism by which technologies evolve we will
	      understand how innovations arise. In other words, innovations arise
	      as the result of an evolutionary process, rather than de novo from
	      the brain of a designer.
	\item
	      Darwinian evolution, or natural selection, is not appropriate for
	      technology. Arthur quotes from Samuel Butler's essay ``Darwin Among
	      the Machines'' : ``{[}t{]}here is nothing which our infatuated race
	      would desire more than to see a fertile union between two steam
	      engines\ldots{}'' to illustrate the impossibility of slavish
	      adoption of biological models.
	\item
	      However, we can clearly see descent of form, in the example given by
	      Gilfillan in 1935 tracing the development of various elements of the
	      sailing ship: planking, sails, keels, ribbing and fastenings. In
	      each case we see a line of gradual improvements leading to the
	      present day component. But the point is made that this is not
	      evolution in the full sense, as it lacks both universal scope and an
	      underlying mechanism.
	\item
	      The first obstacle to a more general scope is the existence of
	      innovations such as the jet engine, laser, railroad locomotive, or
	      QuickSort computer algorithm (to name Arthur's examples.)
	      Innovations seem to appear without obvious parentage; they do not
	      appear to be the result of gradual changes or adaptations to earlier
	      technologies.
	\item
	      Arthur's answer is to look inside the innovation and to recognise
	      that each is made up of recognisable components or modules; the key
	      lies in the nature of heredity in technology. Technologies are
	      formed by combining modules of earlier technologies. These groupings
	      start as loose assemblages to meet some new function, but over time
	      become fixed into a standard unit (for example, the change in DNA
	      amplification mechanisms from assemblages of laboratory equipment to
	      standard off-the-shelf products.)
	\item
	      \autocite{Bourrat2015} comments that distributive evolution (where
	      distribution of elements changes, as result of selection or drift)
	      cannot result in novelties
	\item
	      Arthur's response is that novelty comes from incorporating new
	      phenomena\ldots{}
\end{itemize}

\section{Artificial Intelligence}\label{ai}
Artificial General Intelligence--one approach to AGI holds that intelligence is an evolutionary adaptation, and therefore that most
promising approach is to follow an evolutionary process

Early review of AL approach to AI in http://www.mitpressjournals.org/doi/pdf/10.1162/artl.1993.1.1\_2.75

Rodney Brooks/Maes--actionist approach to robotics. Similar bottoms-up approach to the one we propose.

Intelligence needs environment

Manipulating symbols that are not grounded in the environment doesn't look like it'll result in emergent intelligence

Intelligence arose from interactions with the world

Understanding of intelligence based on humans--replicating this
difficult. And yet, hard to generalize from this one sample. So rather
than trying to recreate human intelligence, let intelligence emerge from
interactions with world

\begin{itemize}
	\item
	      Similar approach to emergent Alife
	\item
	      The AL approach to AI
\end{itemize}

Similar problems to Alife

\begin{itemize}
	\item
	      ``Life'' and ``Intelligence'' are both hard to define and measure
	\item
	      Both appear emergent--that is, hard to measure progress--either yes
	      or no
	\item
	      Best measures for both may be Turing-like--if it quacks like a
	      duck\ldots{}comparison to existing life or human intelligence
\end{itemize}

\section{Evolutionary Computation}

\begin{itemize}
	\item
	      EAs originally abstracted/inspired by biological ENS
	\item
	      Since specialized, radiated into new areas
	\item
	      Re-unification attempted in works such as \autocite{Paixao2015} but hence not concerned with extension beyond \quote{models in theoretical population genetics and in the theory of evolutionary computation}{\autocite{Paixao2015}}
	\item
	      \quote{Some EDAs can be regarded abstractions of evolutionary processes:
	      	instead of generating new solutions through variation and then
	      	selecting from these, EDAs use a more direct approach to refine the
	      	underlying probability distribution. The perspective of updating a
	      	probability distribution is similar to the Wright--Fisher model.}{\autocite{Paixao2015}}
\end{itemize}

Major points of difference:

\begin{itemize}
	\item
	      Mutation combines copying errors with genetic drift (and probably more). (This, as we will later see, is important for our thesis)
	\item
	      Search through a fixed space, cannot surprise \eg \autocite{Nellis2014}
	\item
	      Fitness mechanism--implicit vs explicit.  ``The
	      difference is that we require a system with the potential for a
	      large degree of intrinsic adaptation for open-ended evolution,
	      rather than a system where the selection of individuals is
	      determined by an externally-defined fitness function'' \autocite{Taylor2001}
	\item
	      top-down for EAs--specific constructs without endogenous evolution (processes used not subject to evolution)
\end{itemize}

Similar interests in desirable properties:

\begin{itemize}
	\item redundancy and degeneracy -- \autocite{Whitacre:2010qy})
	\item novelty (novelty-search - \autocite{Lehman:2008cr})
\end{itemize}
