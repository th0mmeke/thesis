\chapter{Introduction}\label{introduction}

\settowidth{\epigraphwidth}{Wonderful life : the Burgess Shale and the nature of history}
\epigraph{%
Without hesitation or ambiguity, and fully mindful of such paleontological wonders as large dinosaurs andAfrican ape-men, I state that the invertebrates of the Burgess Shale, found high in the Canadian Rockies in YohoNational Park, on the eastern border of British Columbia, are the world's most important animal fossils. Modern multicellular animals make their first uncontested appearance in the fossil record some 570 million years ago--and with a bang, not a protracted crescendo.}%
{\textit{\\Wonderful life : the Burgess Shale and the nature of history}\\\textsc{Stephen Jay Gould}}

The underlying motivation for this work is to understand the requirements for onset of artificial evolution; how robust, interesting, adaptive evolution can be started and undergo self-improvement in the non-living world.

Natural evolution is the foundation for several related bodies of literature, in Computational Intelligence and Artificial Life, as well of course of the study of evolution in natural systems. As many have observed, there are major differences between the original inspiration and the spin-offs. Biological evolutionary systems are complex, contingent, emergent, endogenous and general. \todo{By contrast, as a technology, artificial evolutionary systems are exogenous, problem-specific and designed.}

A convenient shorthand capturing these differences labels the biological original as bottom-up and the derived fields as top-down. If the top-down approach has failed, then perhaps there might be mileage in returning to biology, our sole example of a general self-improving, self-organizing system.

But as biology is contingent, what is the most useful starting point for a recapitulation? Assume first of all that we are in the domain of software, rather than hardware or wetware, for practical reasons. And also assume that biological evolution does not depend upon embodiment - that the outcome can be adequately modelled at a conceptual level without being grounded in either the physical world, or in contingent biology such as enzymes, proteins and genes.

Then our belief is that the evolution of evolution itself can and will emerge under certain conditions, and that an emergent approach will be more successful than the predominant design-driven top-down approach today.

Simple ongoing evolution is not the same as creative or interesting evolution; current state-of-the-art insufficient and unsatisfying. \autocite{Bourrat2015} makes a related distinction between distributive and transformative forms of evolution, where distributive evolution sees simply a change in population distribution without the generation of novelties or new elements characteristic of transformative evolution. The fundamental open problem is how to achieve transformative evolution in artificial systems.

\emph{A copy mechanism with adjustable fidelity can auto-adjust for example to changing environmental conditions.}

\begin{itemize}
	\item
	      Not same as ENS--that explains natural processes; our goal is to achieve something that is as interesting in a different domain
	\item 
	      Not same as EAs--EAs have exogenous fitness, not open-ended (search through a fixed space, cannot surprise (e.g., \autocite{Nellis2014})
	\item  
	      Not same as approach where take ideas from biology and evaluate for different purpose (e.g., in EAs, island populations, GRNs (e.g., L-systems), Lamarkian learning, co-evolution, evolutionary transitions (cooperation and mediation in \autocite{Defaweux:2005fk}\ldots{}). All seem to follow model of currently we use these tools, biology has something we don't have, let's try it\ldots{}
	\item
	      Not the same as most type of Alife, where evolution is not the subject of the research but rather a tool or mechanism towards some other goal (canonically, creating artificial life.)
\end{itemize}

Alternatives to ``evolution''?

\begin{itemize}
	\item Lower bound is random search--not good enough
	\item Upper bound is unknown--but ENS is the current gold standard for self-improving complex systems
	\item A good result would be `interesting' forms
\end{itemize}

\subsection{Life, the gold standard}

\epigraph{%
	In the world of human thought generally, and in physical sciences in particular, the most important and most fruitful concepts are those to which it is impossible to attach a well-defined meaning.}%
{\textsc{\\H.A. Kramers}}

\quote{
	What modifications must be made to this type of
	experiment to allow at least one of the following outcomes:
	‘open-ended evolution’ (Bedau et al., 2000); the origin of
	basic autonomy, i.e. a dissipative system capable of the
	recursive generation of functional constraints (Ruiz-Mirazo
	et al., 2004); a process ultimately capable of the
	production of nucleic acids or other modular replicators
	with unlimited heredity potential (Maynard-Smith and
	Szathmary, 1995; Szathmary, 2000); identification of ‘‘the
	course of evolution by which the determinate order of
	biological metabolism developed out of the chaos of intercrossing
	reactions’’ (Oparin, 1964); the coupled cycling of
	bioelements (Morowitz, 1968, 1971); the maximization of
	entropy production by a biosphere (Kleidon, 2004); the
	minimal unit of life (Ganti, 2003a, b); or an autopoetic unit
	(Maturana and Verela, 1992)?}{\autocite{Fernando:2007pf}}

	Ganti's observation that contemporary living things always have a
	metabolic subsystem, a heritable control system, and a boundary system
	to contain (in \autocite{Szathmary:2006ty}).
	
	One distinction between living and non-living comes from \autocite{Rasmussen2004} -- non-living systems explore a state-space driven by thermodynamics, and so in a sense through a random ergodic search. Living systems however almost universally employ evolution. Another information-theoretic view, from \autocite{Adami2015}, is that living systems can preserve information on a much longer timescale than non-living things.
	
	\autocite{Pascal2013}: details of early life may be unknown and unknowable, but principles possible. Sudden transition to life like today astronomically unlikely - single RNA strand capable of ribozyme activity probability of 10E-60\ldots{} (for 100 monomers). The more likely alternative is that there was a series of intermediates, of increasing degrees of ``aliveness''. The corollary is that it is hard therefore to imagine a clear cut transition between non-living and living.
	
	metabolism = how ``living matter evades the decay to equilibrium''
	(Schrodinger 1944) in \autocite{Pascal2013}
	
	Definition of Life  \autocite{Pascal2013}: Elusive, and probably not useful
	
	\subsection{Origins of life}
	
	Oparin 1924 scientists explore the origin of life ``like two parties
	of workers boring from the two opposite ends of a tunnel'' - chemical
	bottom-up and biological top-down approaches in \autocite{Pereto2012}
	
	``What we do not know today we shall know tomorrow. A
	whole army of biologists is studying the structure and organization of
	living matter, while a no less number of physicists and chemists are
	daily revealing to us new properties of dead things. Like two parties
	of workers boring from the two opposite ends of a tunnel, they are
	working towards the same goal. The work has already gone a long way
	and very, very soon the last barriers between the living and the dead
	will crumble under the attack of patient work and powerful scientific
	thought.'' (\href{http://www.valencia.edu/~orilife/textos/The\%20Origin\%20of\%20Life.pdf}{\emph{http://www.valencia.edu/\textasciitilde{}orilife/textos/The\%20Origin\%20of\%20Life.pdf}})
	
	Similar problem--from chemistry to biology. Transition to biology.
	
	Systems Chemistry is tightly related field.
	
	Analogous to origin of life, but despite parallels, must resist temptation to extend claims to this--the evolution of life was contingent, and because of lack of evidence from early stages, no way anyway to test or confirm.
	
	Work to describe a pathway from plausible conditions on the early abiotic Earth to the first living things. Important to understand that this pathway only describes one possible form of life, and one possible mechanism -- restricted in scope to connecting the two endpoints.
	
	The primary postulates of OOL are not our postulates. We do though have concepts and principles in common - meta-ENS, or the workings of evolutionary processes, of which the mechanisms of OOL are one concrete example (a form of existence proof.)
	
	\begin{itemize}
		\item
		hypercycles \autocite{Eigen1971}
		\item
		autocatalysis
		\begin{itemize}
			\item
			present in all three elements (\autocite{Ganti:2003hl}--or
			earlier?) of life
			\item
			DNA replicates with enzymatic help
			\item
			some metabolites, such as ATP, exclusively autocatalytic
			\item
			lipids in a membrane enhance addition of other lipids--Formation of
			autocatalytic cycles/sets (\autocite{Hordijk2004})--percolation increases
			likelihood
			\item
			\autocite{Sousa2015} for detection of RAF sets in e coli metabolic
			networks
			\item
			Selection amongst autocatalytic networks (Ganti and Wachtershauser referenced in \autocite{Fernando:2005ly})
			\item
			\autocite{Fernando:2007pf}--selection/liposomes/chemical avalanches
			based on Wachtershauser
			\item
			autocatalytic cores (\autocite{Vasas2012})
			\item
			Kauffman's original Reflexively Autocatalytic Polymer Networks
			(RAPN) \autocite{Kauffman1986,Farmer1986} are not capable of
			non-digital evolution. RAPN stabilise into single network without
			variation
		\end{itemize}
		\item
		GARD (fixed catalysts) \autocite{Segre1998} not capable of evolution -
		lack heredity of variation (mutations overwhelm heredity) \autocite{Vasas2010}
		\item
		Bimolecular rearrangements (\autocite{Fernando:2008xy,Fernando:2007pf})?
		\item
		template replicators--highly unlikely without intermediate steps
		\item
		Must be driven far-from-equilibrium (many references e.g.,
		\autocite{Pascal2015}--continuous supply of energy required (explicit
		modelling unlike say \autocite{Fontana1994} where energy only
		implicitly modelled)--and maintained there (by metabolism--e.g., how
		\quote{living matter evades the decay to equilibrium}{\autocite{Schrodinger1944}})
		\item
		Eigen threshold for replicators--high mutation rates overwhelm
		heredity
		\item
		Biological evolutionary theory
		\item
		Extension beyond biology
		
		\begin{itemize}
			\item
			Previous work on extending evolution--\autocite{Bourrat2015} etc
			\item
			Evolution of culture, language, technology
		\end{itemize}
		\item
		OOL is focussed on plausible explanations for life-as-we-know-it
		\item
		And must be constrained by biological givens, i.e.
		\item
		Starting point consistent with what is known about archaic Earth
		\item
		Must lead to end-point consistent with earliest known life
		\item
		In a reasonable time period
		\item
		single-step astronomically unlikely (single RNA strand probability
		about 10E-60, based on 100 monomers--\autocite{Pascal2013})
		\item
		Requires a series of steps--akin to OOL where single-step
		astronomically unlikely (single RNA strand probability about 10E-60,
		based on 100 monomers--\autocite{Pascal2013})
		\item
		Contingent and specific--constraints
		\item
		starting point compatible with what is known of prebiotic conditions
		(either on earth or extraterrestrially)
		\item
		end point of ENS at something that might be LUCA
		\item
		Must be simplified and abstracted
		\item
		Choosing an artificial chemistry similar to natural chemistry enables
		an argument by analogy
		\item
		Meets known constraints for OEE--and we have OOL as an example of OEE
		from natural chemistry
		\item
		Catalysis/autocatalysis possible through emergence in some AChems
		(\eg \autocite{Virgo2013})
		\item
		Some fundamental differences remain however between our domain and the
		natural domain
		\item
		The sheer size of the Biosphere means we can never duplicate the
		number of individual evolutionary ``trials'' (selection, mutation and
		reproduction) events in our model
		\item
		The Biosphere is underpinned by a phenomenally rich set of physical
		and chemical laws, implicitly constraining each and every action and
		interaction
	\end{itemize}
	
	
	\subsection{Origins of Life}\label{origins-of-life}
	
	Darwin ``warm little pond'' Life and letters vol3 1887
	(``\emph{\textbf{But if (and Oh! What a big if!) we could conceive in
			some warm little pond, with all sorts of ammonia and phosphoric salts,
			light, heat, electricity etc, present, that a protein compound was
			chemically formed ready to undergo still more complex
			changes\ldots{}'')}} \autocite{Vasas2012}
	
	
	Of the historic and ahistoric aspects of the Origin-of-life problem,
	historic aspect may never be known \autocite{Pross2013}. Constrained by
	what we do know, but many different pathways, and unless some record
	somewhere (either geological or phylogenetic), actual path essentially
	lost to history. So without evidence for historic aspect, not possible
	to test by falsification, and hence can only be speculative.
	
	Consensus forming that early life formed by chemoautotrophs with energy
	from inorganic redox couples and biomass from CO\textsubscript{2}, and
	that innovations in carbon-fixation created main branches in
	tree-of-life \autocite{Braakman2012}. Initiation of selection marked by
	\gls{ida}, probably from RNA world, followed substantially later by Last
	Universal Common Ancestor (LUCA) \autocite{Yarus2011}, which, it is
	important to note for clarity, was almost certainly not a single cell or
	even species, but rather a construct of evolutionary genetics because of
	the likely predominance of Lateral Gene Transfer (LGT) in archaic
	biology (http://sandwalk.blogspot.ca/2007/03/web-of-life.html).
	Self-replicating RNA enzymes shown in \autocite{Lincoln2009}, forming
	basis of selective system (link to natural selection) (also see
	\autocite{Cheng2010}, \autocite{Powner2009} for formation of RNA in
	prebiotic conditions). Some elements of \gls{ida} thought still with us
	in lineages of informational (for protein synthesis and RNA
	transcription) and operational genes (for some standard cellular
	processes) \autocite{Ragan2009}, for example the ribosome and
	ribonuclease P (RNase P) \autocite{Wilson2009}. Next major transition to
	Protein world (although predominance of RNA transcripts leads to
	suggestions that should be called RNA-Protein world
	\autocite{Altman2013})
	
	Two alternative models for the step from abiotic to \gls{ida}: genetic
	or replicators or RNA-first, and metabolism or protein-first. Both
	metabolism and replication almost certainly required for \gls{ida}. A
	self-sustaining autocatalytic network (in terms of a RAF set
	specifically a ``set of molecules and reactions which is collectively
	autocatalytic in the sense that all molecules help in producing each
	other (through mutual catalysis, and supported by a food set).'')
	generally considered essential \autocite{Pross2013}, but not sufficient
	\autocite{Hordijk2011}. Both competing models--replication first and
	metabolism first--build on that. Autocatalysis expressed by
	self-replication of oligomeric compounds in replication first; by cycles
	and network in metabolism first. In the broadest sense, life can be seen
	as an autocatalytic process where an entity catalyses the production of
	one or more descendant entities.
	
	Metabolism-first privileges function, while replicator-first privileges
	descent.
	
	If life is metabolism plus information, then for metabolism-first where
	lack template-based replication, replication is compositional (composome
	- Vasas)
	
	Common functions required in both protocells and minimal cells, but
	approaches quite different. Protocells must build up from abiotic
	conditions to point where known processes can take over, and to point
	where we have historical evidence--bridge gap between abiotic
	conditions and first hypothetical cells in record
	
	Main issues with replicator-first model: big step from abiotic compounds
	to template-based replication (although ribonucleotides conceivably
	could form in pre-life conditions see \autocite{Powner2009}). Templates
	encode information in biology, so require a encode/decode mechanism as
	well as an information code to represent the product. This is a step
	more complex than simpler duplication.
	
	Main issue with metabolism-first model: shift from composome inheritance
	to template-based; ability of composomes to fulfill heredity requirement
	for natural selection.
	
	Autocatalytic sets are proposed for bootstrapping (from less
	differentiated systems to more complex ones), as a mechanism to increase
	order and so counteract entropy, for stability, and as a unit of
	competition. The catalytic properties of autocatalytic sets result in
	bootstrapping, where the reaction rates in the set are ratcheted-up as
	the product quantities increase. Autocatalytic sets form a complex or
	dynamical system, where the feedback loops and interactions can result
	in forms of temporal (waves and cycles) and spatial order (stable
	structures.) Similarly stability follows from the formation of
	attractors in the complex system that are resistant to pertubation.
	Finally, as the set forms a coherent unit expressing certain properties,
	such as the ability to maintain itself or to create copies of itself, it
	can be seen as a unit of competition where success is measured by
	life-time or by control of resources.
	
	Some abiogenesis results fundamentally assume real-world chemistry and
	conditions, a constraint that doesn't apply to Alife or artificial OEE,
	and so is more restrictive than required. Other abiogenesis work such as
	on properties of autocatalytic sets, is broader in applicability.
	Genetic and catalytic properties of RNA make well suited to creation of
	Alife \autocite{Cheng2010}
	
	Autocatalysis is a source of dynamical behaviour--leads to
	bifurcations, multistability, oscillators, attractors
	\autocite{Plasson2010}
	
	Some properties of RAF sets established by Hordijk and Steel have
	application beyond origin-of-life: linear growth rate in level of
	catalysis compared to size of molecules (n) is sufficient for RAF set to
	form. For a RAF set to appear with high probability (P\textgreater{}0.5)
	in even a model with n=20 (about 1 million molecular types) each
	molecule needs to catalyze 1-2 reactions on average. For the given
	model, need around 65,000 different molecule types (n=15-16) for RAF set
	if use more realistic probability of catalysis of 1:1 million of a
	molecule catalysing any particular reaction. \autocite{Hordijk2011}
	
	Lateral or Horizontal Gene Transfer thought so common in early life that
	no single common ancestor, but genes from multiple lineages combined
	into all lineages today.\autocite{Ragan2009}
	

\subsubsection{Living organisms are supremely well suited to their environments, and can adapt to environmental changes}
\label{living-organisms-are-supremely-well-suited-to-their-environments-and-can-adapt-to-environmental-changes}

Adaptation of organisms to their environments occurs in the main on two
different time-scales.

Evolution by \gls{naturalselection} acts over a period of generations on
populations of individual organisms. Changes are therefore relatively
gradual, and many generations can pass before a change such as a
beneficial mutation becomes ubiquitous in a population (\eg 300-500
generations for targeted modifications in lactose processing in
\emph{E. coli} \autocite{Dekel:2005fk}. In contrast, gene regulatory
effects act during the life cycle of a single individual, either during
development to affect morphology, or during the adult lifespan in
reaction to seasonal or other environmental cues. These
regulation-driven changes are not in themselves heritable, but they can
be assimilated back into the population by influencing the organisms
fitness under natural selection (\eg,
\autocite{Baldwin:1896ly,Dennett:2003ve,Paenke:2009xe,Paenke:2007ve}).

Similar effects can be seen by another adaptive mechanism that operates
on individuals during their lifespan: learning, where behavioural
adaptions can also lead to genetic change
(\eg \autocite{Hinton:1987vy}.)

\subsubsection{Natural selection, acting on populations, is the primary driver for long-term adaptation}
\label{natural-selection-acting-on-populations-is-the-primary-driver-for-long-term-adaptation}

The year 2009 saw the celebration of the 150\textsuperscript{th}
anniversary of the publication of \emph{On the Origin of Species}, the
explanation of evolution by natural selection, with extensive coverage
in scientific and popular media. The terms are therefore fairly well
known to many people, but what exactly does \emph{natural selection}
mean? To quote from \autocite{Futuyama:1979tg}, \gls{naturalselection}
is the ``differential survival and reproduction of genotypes''.

Let's examine each of the components in this idea in turn:

Differential survival. Living organisms are constantly engaged in an
intimate relationship with their environments. Indeed, according to the
theory of \emph{autopoiesis} \autocite{Varela:1974qd}, organisms are
defined by this engagement: to be alive means maintaining oneself
against the surrounding environment. In general the more effectively the
organism is able to do this, the more likely it is to survive. However,
in practice survival for an individual may be affected by random events.
Scholarship winning students can be killed by drunk drivers. Sardines in
shoals flash and turn, yet sharks still manage to grab one or two from
the shoal effectively at random. Averaged over a population however
these chance events balance out; a succession of random trials leads to
a skewed distribution of fitness away from the less able.

\footnote{This introduces two significant differences between \glspl{ea}
	and biology: first, \glspl{ea} conduct a series of discrete trials of fitness,
	rather than a continuous evaluation. Second, fitness in an \gls{ea} is
	measured by an explicit \emph{objective function} whereas in biology
	fitness \emph{emerges dynamically} through continuous interaction with
the environment.}

Reproduction. In this sense, reproduction simply means inheritance. Only
those characteristics that can be passed on from one generation to the
next are relevant. One implication of this is that the only traits of an
organism that matter to natural selection are those that are apparent
while the organism can reproduce. Altruism and kin-selection, where an
individual acts to increase the fitness of a related individual, are
interesting for the light they shed on this implication.

Genotypes. An organism's \gls{genotype} is the heritable information
that defines an individual, of which the great majority is encoded in
DNA (some \emph{epigenetic} information is inherited through
DNA-methylation, maternal protein concentrations and other mechanisms.
However, generally DNA remains the primary source.)

How then does natural selection unfold in practice? Although an organism
is defined by its genotype, its survival is based not on the raw
genotype, but on the expression of the genotype--called the
\gls{phenotype}--that participates in the interaction with the
environment. There is not necessarily a direct one-to-one mapping
between genotype and phenotype; for example, environmental triggers
during development can switch the phenotype in different directions (a
phenomenon called \gls{polyphenism}.)

This indirect mapping enables a number of important mechanisms
significant to the operation of natural selection: first, changes in the
genotype (caused by mutations for example) may build up independent of
phenotype changes--the idea of \emph{neutral mutations}
\autocite{Ohta:1996vn,Ohta:2002ys,Ohta:1973kx}. Examples from studies
of RNA secondary structures (the physical folding of RNA molecules) show
that many closely-related RNA sequences can produce the same RNA folding
\autocite{Fontana:1993zn}. Adjacent changes often have little effect on
structure.

Second, by extension, \autocite{Gavrilets:1997qt} and
\autocite{Gravner:2007yd} have shown that in cases involving many gene
loci under well-defined conditions there is a path between viable
phenotypes that requires only neutral mutations.

Third, behaviours such as learning, rather than purely genetic
mechanisms, can influence the form of this \gls{gpmap} in combination
with natural selection, as illustrated by the Baldwin effect
\autocite{Baldwin:1896ly} and other examples of genetic assimilation
{[}\autocite{Hinton:1987vy};Siegal:2002qn;Waddington:1942jb{]}.

Finally, \glspl{grn} provide another mechanism to modify the \gls{gpmap}
and hence to guide natural selection.

\subsubsection{Novelties often arise from new regulatory connections rather than changes to genes}
\label{novelties-often-arise-from-new-regulatory-connections-rather-than-changes-to-genes}

\autocite{Prudhomme:2007ax} believe that evolutionary novelties more
commonly arise from changes or additions of regulatory
\emph{connections} than from the development of \emph{new} genes or
regulatory elements; that is, from changes to the network topology
rather than from additions to the network elements. The underlying
implication is that novelties are therefore new compositions of
pre-existing elements, rather than being constructed `\emph{de novo}',
and that production of novelties may be relatively rapid. Connection
changes may happen quickly; by comparison, new genes may take many generations.

\section{Goal}
The overall problem we address is to \textit{determine the conditions under which ongoing evolution might occur in a software system}. 

Ongoing evolution means something more than continual change of population frequencies; instead we refer to the goal of a process that continues to produce interesting, novel, and creative results, as does natural biological evolution; in other words, we mean the evolution of evolution. Finally, we constrain our domain to software rather than wetware or hardware; simulations and models, rather than elements embedded in the physical world.

This is clearly a broad and ambitious goal, one more suited to a research programme than a single thesis, and a goal that is additionally the subject of a complex body of related and previous work. We shall return to a more specific and pragmatic problem description in the later section, \ref{research-questions}, after presenting a summary of the most relevant literature.
	
\section{Previous work}
The previous sections have established the context, and incidentally identified the primary areas of relevant literature. From biology, the main threads concern the origins of life, and the theory of evolution. Natural selection provides the primary mechanism for evolution in biology, although it is not the only process at work -- genetic drift and neutral theory \cite{Kimura:1968uq} provide a counterpoint. In artificial systems \\todo{EC etc}, researchers have used biological evolution as an inspiration, gradually over time diverging into a field with the beginnings of its own motivating theory, now only loosely connected with its origins in the natural world. These areas are discussed in more detail in \ref{context} to aid in understanding the context behind the remaining sections of the work.

In artificial systems, some researchers have recently returned to a reappreciation of how biological evolution generates robust, novel, creative outcomes, unlike those seen in current artificial evolution. This has led to a renewed interest in understanding the principles behind biological evolution so that artificial systems can capture some of these admirable properties; the difference now is that the transference is sought at the level of concepts and principles rather than in the historical inspiration of specific biological elements and structures, many of which are contingent and perhaps even arbitrary; certainly complex.

This chapter expands upon previous work that is directly relevant to our problem overall, the conditions for the evolution-of-evolution in software systems. 

The review is organized roughly in alignment with the research goal: first an examination of work that discusses evolution in the abstract; next a summary of the adoptation of these ideas in artificial systems; and finally a review of artificial systems that claim to exhibit ongoing creative evolution. The relevant open problems are summarized in the conclusion to this section.

\subsection{Evolutionary principles}

Pigliucci2008 - ``Is evolvability evolvable?''

``We take it as given that biology instantiates ENS'' - but that doesn't mean that the algorithm of biology is ENS \autocite{Watson2012}.

Although the advantages of a distinction between genotype and phenotype are discussed by many, including \autocite[section 7.2.3]{Taylor1999}, there is no inherent dependency on this in \gls{ens}. Early evolution may have involved holistic or composomal evolution before the advent of digital evolution with a separate genome.

Adaptation in biology appears to precede Natural Selection, so adaptation is possible without NS \autocite{Watson2010}	      	      	

\quote{Evolution is a process that results in heritable changes in a population spread over many generations.}{
	``Sandwalk: strolling with a skeptical biochemist'',
	\url{http://sandwalk.blogspot.co.nz/2012/10/what-is-evolution.html}}

\quote{
	Biological evolution consists of change in the hereditary characteristics of groups of organisms over the course of generations. Groups of organisms, termed populations and species, are formed by the division of ancestral populations or species, and the descendant groups then change independently. Hence, from a long-term perspective, evolution is the descent, with modification, of different lineages from common ancestors.}{
	``Evolution, Science, and Society: Evolutionary Biology and the National Research Agenda'', Working Draft, 28 September 1998, \url{http://www.zoology.ubc.ca/~otto/evolution/Evolwhite.pdf}}

Separate outcome or result from process or mechanism (e.g., adaptation)


\subsubsection{Evolution by Natural Selection (ENS))}\label{ens-evolution-by-natural-selection}

``Owing to this struggle for life, any variation, however slight and
from whatever cause proceeding, if it be in any degree profitable to
an individual of any species, in its infinitely complex relations to
other organic beings and to external nature, will tend to the
preservation of that individual, and will generally be inherited by
its offspring. The offspring, also, will thus have a better chance
of surviving, for, of the many individuals of any species which are
periodically born, but a small number can survive. I have called
this principle, by which each slight variation, if useful, is
preserved, by the term of Natural Selection, in order to mark its
relation to man's power of selection (Darwin 1859, p. 61).'' \autocite{Griesemer2005}

ENS is an example of OEE
ENS is strongly tied to life, and mostly presupposes living subjects.

Greisemer 2000 states that the elements of evolution in the Lewontin (variation, fitness differences, heritability of fitness) and Maynard-Smith (variation, multiplication, heritability) formulations are fundamentally different (from \autocite{Vasas2015}).


``Darwin's theory of evolution by natural selection is restricted in scope. One sense in which it is restricted is that it refers to organisms.'' \autocite{Griesemer2005}

Restricted by \autocite{Griesemer2005}:
\begin{itemize}
	\item natural selection, not drift or Lamarckian (biased) inheritance
	\item organisms not defined - so scope vague. Many treat vertebrates as paradigm, despite rarity
	\item organismal level - but many other levels
\end{itemize}

\autocite{Griesemer2001}

examination of difference between Lewontin and Maynard Smith's views
of units of selection/evolution

Maynard-Smith
\begin{itemize}
	\item
	
	``if there is a population of entities with multiplication,
	variation and heredity, and if some of the variations alter the
	probability of multiplying, then the population will evolve.
	Further, it will evolve so that the entities come to have
	adaptations....''
	
	\item
	
	M,V,H necessary for units of evolution. F sufficient for evolution
	of adaptations (with proviso in text)
	
	\item
	
	Units of evolution that have F are units of selection. Units of
	selection though do not imply units of evolution as not all MVHF
	necessary.
	
\end{itemize}

Lewontin
\begin{itemize}
	\item
	
	phentotypic variation, differential fitness, fitness is heritable
	(by generalizing from Darwin 1859 p61) ``Owing to this struggle for
	life...Natural Selection\ldots{}''
	
	\item
	
	``These principles mention quantities that describe quantitative
	roles of causal capacities (Woodward, 1993)''
	
\end{itemize}

Darwin's concept of inheritance in 1859 p61 includes heritability (a capacity) and inheritance (a process carrying the capacity). Lewontin stresses first, assuming second, and Maynard Smith's multiplication is about the second; his heredity is both. \autocite{Griesemer2001}


\autocite{Godfrey-Smith2007}:

\begin{itemize}
	\item
	
	Purposes of summaries vary - typically either pedagogy, defending
	evolutionary theory, extensions to other domains (e.g., cultural), and
	have ``intrinsic scientific and philosophical interest as attempts to
	capture some core principles of evolutionary theory in a highly
	concise way.''
	
	\item
	
	Subject to counterexamples, and connection to formal models ``not
	straightforward''
	
	\item
	
	Core requirement for ENS from various summaries is ``combination of
	variation, heredity, and fitness differences.''
	
	\item
	
	Most commonly cited summary is Lewontin1970 - but unusual in that says
	``fitness is heritable'' - typically phenotypic heredity is sufficient
	for trait to evolve. 1980 version more typical
	
	\item
	
	Endler1986 and Ridley1996 discussed
	
	\item
	
	Necessary and sufficient ambiguous in any summary - might mean will
	result in ENS when know what ENS is, or discriminatory - this process
	is ENS. That is, constitutive or causal readings. E.g., conditions for
	``becoming pregnant\ldots{}{[}versus{]} being pregnant''
	
	\item
	
	Usual aim is to ``give conditions that are sufficient ceteris paribus
	for a certain kind of change occurring.'' where the change is assumed
	to be wrt some trait
	
	\item
	
	Problem cases for summaries
	
	
	\begin{itemize}
		\item
		
		Culling - no reproduction. If ENS, then no heredity necessary
		
		\item
		
		Different generation times - only difference is reproductive rate
		
		\item
		
		Biased inheritance
		
		\item
		
		Heritability fails in the fit
		
		\item
		
		Stabilizing selection in an asexual population
		
		\item
		
		Covariance positive with no variation in Z
		
		\item
		
		Accident
		
		\item
		
		Correlated Response
		
	\end{itemize}
\end{itemize}

\subsubsection{Compositional evolution}

In life, have HGT e.g., \autocite{Ochman2000}

In general though, compositional evolution does not attempt to understand life, but instead as a guide to achieving similar property in another domain, or in understanding similar processes in another domain \autocite{Arthur2009}

\autocite{Watson2002} discusses compositional evolution and building blocks

\autocite{Pross2011}--Dynamic Kinetic Stability

\autocite{Arthur2009} investigates the evolution of
technology, where evolution is used in the sense of \quote{all objects of
	some class are related by ties of common descent from the collection
	of earlier objects.}{\autocite{Arthur2009}}
\begin{itemize}
	
	\item
	Evolution in technology occurs by using earlier technologies as
	building blocks in the composition of new technologies, and these new
	technologies then become building blocks for use in later
	technologies, and so on. Arthur calls this ``combinatorial
	evolution.'' But what is the starting point? How is this regression
	grounded? Arthur proposes that the capture and harnessing of natural
	phenomena starts each lineage, and provides new raw components for
	inclusion in later technologies.
	\item
	Evolution is related to innovation: in fact, Arthur claims that by
	understanding the mechanism by which technologies evolve we will
	understand how innovations arise. In other words, innovations arise
	as the result of an evolutionary process, rather than de novo from
	the brain of a designer.
	\item
	Darwinian evolution, or natural selection, is not appropriate for
	technology. Arthur quotes from Samuel Butler's essay ``Darwin Among
	the Machines'' : ``{[}t{]}here is nothing which our infatuated race
	would desire more than to see a fertile union between two steam
	engines\ldots{}'' to illustrate the impossibility of slavish
	adoption of biological models.
	\item
	However, we can clearly see descent of form, in the example given by
	Gilfillan in 1935 tracing the development of various elements of the
	sailing ship: planking, sails, keels, ribbing and fastenings. In
	each case we see a line of gradual improvements leading to the
	present day component. But the point is made that this is not
	evolution in the full sense, as it lacks both universal scope and an
	underlying mechanism.
	\item
	The first obstacle to a more general scope is the existence of
	innovations such as the jet engine, laser, railroad locomotive, or
	QuickSort computer algorithm (to name Arthur's examples.)
	Innovations seem to appear without obvious parentage; they do not
	appear to be the result of gradual changes or adaptations to earlier
	technologies.
	\item
	Arthur's answer is to look inside the innovation and to recognise
	that each is made up of recognisable components or modules; the key
	lies in the nature of heredity in technology. Technologies are
	formed by combining modules of earlier technologies. These groupings
	start as loose assemblages to meet some new function, but over time
	become fixed into a standard unit (for example, the change in DNA
	amplification mechanisms from assemblages of laboratory equipment to
	standard off-the-shelf products.)
	\item
	\autocite{Bourrat2015} comments that distributive evolution (where
	distribution of elements changes, as result of selection or drift)
	cannot result in novelties
	\item
	Arthur's response is that novelty comes from incorporating new
	phenomena\ldots{}
\end{itemize}

\subsection{Evolution in artificial systems}

\begin{itemize}
	\item
	EAs originally abstracted/inspired by biological ENS
	\item
	Since specialized, radiated into new areas
	\item
	Re-unification attempted in works such as \autocite{Paixao2015} but hence not concerned with extension beyond \quote{models in theoretical population genetics and in the theory of evolutionary computation}{\autocite{Paixao2015}}
	\item
	\quote{Some EDAs can be regarded abstractions of evolutionary processes:
		instead of generating new solutions through variation and then
		selecting from these, EDAs use a more direct approach to refine the
		underlying probability distribution. The perspective of updating a
		probability distribution is similar to the Wright--Fisher model.}{\autocite{Paixao2015}}
\end{itemize}

Major points of difference:

\begin{itemize}
	\item
	Mutation combines copying errors with genetic drift (and probably more). (This, as we will later see, is important for our thesis)
	\item
	Search through a fixed space, cannot surprise \eg \autocite{Nellis2014}
	\item
	Fitness mechanism--implicit vs explicit.  ``The
	difference is that we require a system with the potential for a
	large degree of intrinsic adaptation for open-ended evolution,
	rather than a system where the selection of individuals is
	determined by an externally-defined fitness function'' \autocite{Taylor2001}
	\item
	top-down for EAs--specific constructs without endogenous evolution (processes used not subject to evolution)
\end{itemize}

Similar interests in desirable properties:

\begin{itemize}
	\item redundancy and degeneracy -- \autocite{Whitacre:2010qy})
	\item novelty (novelty-search - \autocite{Lehman:2008cr})
\end{itemize}


\subsection{Evolution-of-evolution in artificial systems}

Open-ended evolution can be seen as evolution in an open-ended system (\eg Chemistry), where an open-ended system has effectively unrestricted representation: the number of possible types must be much larger than the number of individuals (ideally without any restriction). Without this property all possible types can be generated in a finite time, and the system will either reach stasis or begin to repeat. Not all open-ended systems necessarily support evolution, but in those that do, our intuition suggests that open-ended evolution produces increasing complexity, increasing diversity, accumulation of novelty and continual adaptation \cite{Lehman2012}.

\quote{by open-ended evolutionary capacity we understand the potential of a system to reproduce its basic functional-constitutive dynamics, bringing about an un-limited variety of equivalent systems, of ways of expressing that dynamics, which are not subject to any predetermined upper bound of organizational complexity (even if they are, indeed, to the energetic-material restrictions imposed by a finite environment and by the universal physico-chemical laws)}{\cite{Ruiz-Mirazo2004}}

\begin{itemize}
	\item An open-ended evolutionary system must demonstrate unbounded diversity during its growth phase.
	\item An open-ended evolutionary system must embody selection.
	\item An open-ended evolutionary system must exhibit continuing (``positive'') new adaptive activity.
	\item An open-ended evolutionary system must have an endogenous implementation of niches.
\end{itemize} \cite{Maley1999} (considered ``rather abstract'' by Hutton \parencite[p.341]{Hutton2002}).

\Textcite{Taylor2001,Taylor:1999sc} discuss creativity in \gls{oee} in depth and argues that, for it to be possible, the replicators must \parencite{Hutton2004}:

\begin{enumerate}[label=\roman*] 
	\item Be fully embedded in their arena of competition 
	\item Have rich, unlimited interactions between each other and with their environment 
	\item Initially replicate implicitly, rather than using some encoding of the replication process, and 
	\item Be constructed entirely of `material' components, allowing the possibility of different encodings of information. (\quote{the very stuff from which they are constructed is a valuable resource of matter and energy}{\cite[s3.6]{Taylor2001}})
\end{enumerate}

Bottom-up models for open-ended evolution leverage richness of underlying environment - less information in entity definition, more in environment definition. Similar to biology, where physics and chemistry underpin living organisms, where definition of minimal cell many orders of magnitude simpler than the working out of the chemical and physical rules that it relies upon.

Top-down models assume a knowledge of the necessary elements.

Limited heredity replicators vs unlimited - the first where the number of possible types is less than the number of individuals; the second where it far exceeds\cite{Szathmary:2006ty}

Blurring in use of OEE

\begin{itemize}
	\item Either in sense of quote{an indefinitely large number of structures are each capable of replication.}{\autocite{MaynardSmith1999}}, or
	\item Additionally meaning novel, interesting, surprising
\end{itemize}

Ongoing generation of novel forms

Inevitably leads to increasing complexity as without complexity will exhaust new possibilities and cover old ground


Minimal conditions for evolutionary system capable of open-ended (but not necessarily interesting behaviour)

\begin{itemize}
	\item
	Elements that allow for ongoing evolution--necessary, and starting
	point for novelty
	\item
	Open-ended evolution can be seen as evolution in an open-ended system
	(\eg Chemistry), where an open-ended system has effectively
	unrestricted representation: the number of possible types must be much
	larger than the number of individuals (ideally without any
	restriction). Without this property all possible types can be
	generated in a finite time, and the system will either reach stasis or
	begin to repeat. Not all open-ended systems necessarily support
	evolution, but in those that do, our intuition suggests that
	open-ended evolution produces increasing complexity, increasing
	diversity, accumulation of novelty and continual adaptation
	\autocite{Lehman2012}.
\end{itemize}

\autocite{Taylor2001}:
Competition between individuals for resources - VanValen1973 Red Queen
hypothesis - primary source of intrinsic selection pressure.
Individuals and environment mutually affect each other

Resources must be ``(a) a vital commodity to individuals in the
population; (b) of limited availability; and (c) that individuals
can compete for (at either a global or local level). This resource
can usually be interpreted as energy, space, matter, or a
combination of these.''

``the potential for a large degree of intrinsic adaptation''

Ray made similar arguments in favour of interactions with other individuals (rather than isolated as in EA) 		


Hypothesis - four necessary conditions for OEE (left open if these are also sufficient conditions) \autocite{Soros2014}:

General prereqs: good genetic representation, ``sufficiently large world for every individual to be evaluated'', and a seed or starting point

\begin{itemize}
	\item
	
	A rule should be enforced that individuals must meet some minimal
	criterion (MC) before they can reproduce, and that criterion must be
	nontrivial.
	
	\item
	
	The evolution of new individuals should create novel opportunities
	for satisfying the MC
	
	\item
	
	Decisions about how and where individuals interact with the world
	should be made by the individuals themselves.
	
	\item
	
	The potential size and complexity of the individuals' phenotypes
	should be (in principle) unbounded.
	
\end{itemize}

``openendedness depends fundamentally on the continual production of novelty.'' Standish, in \autocite{Soros2014}

Minimal conditions for OEE \autocite{Vasas2015}:
\begin{itemize}
	\item
	
	very rich combinatorial generative mechanism e.g., organic
	chemistry. Underlies the evolvability of niches (Dorin and Korb
	2011)
	
	\item
	
	unlimited heredity - number of possible heritable types should
	astronomically exceed individuals in population
	(Maynard-Smith:1995lw)
	
	\item
	
	inexhaustible fitness landscape - implies rich, dynamical
	environment
	
	\item
	
	Cannot state in advance possible preadaptations. (not clear why this
	is a condition\ldots{}seems to be saying that evolution is
	algorithmic but results are not - are emergent) Richness part of
	real chemistry, not from representations of chemistry which are
	limited - necessary requirement for OEE in material systems. (But
	real chemistry is also limited...just not as much)
	
\end{itemize}

\quote{
	by open-ended evolutionary capacity we understand the potential of a
	system to reproduce its basic functional-constitutive dynamics, bringing
	about an un-limited variety of equivalent systems, of ways of expressing
	that dynamics, which are not subject to any predetermined upper bound of
	organizational complexity (even if they are, indeed, to the
	energetic-material restrictions imposed by a finite environment and by
	the universal physico-chemical laws.}
{\autocite{Ruiz-Mirazo2004}}

\begin{itemize}
	\item
	Open-ended from \autocite{MaynardSmith1999} definition--\TODO{ size of search space vs population}
	\item
	Heritability a challenge--biological organisms employ digital
	heredity; sophisticated mechanism with controlled error rates, but
	exceedingly unlikely to arise spontaneously
	\item
	Multiplication/heredity for maintenance of population
	\item
	Analog methods possible--\eg:
	
	\begin{itemize}
		\item
		compositional (where new entity contains some elements of original)
		(as seen in ACS ``core'' inheritance e.g., \autocite{Vasas2015, Watson2012}?)
		\item
		\quote{migrant pools}{\autocite{Watson2015}}
		\item
		Group fissioning \autocite{Watson2015}
		\item
		Attractor based \autocite{Szathmary2000}
	\end{itemize}
	\item
	Heredity seen as method to maintain low entropy over much longer time
	than possible with non-''biological'' systems \autocite{Adami2015}
	\item
	Argument that heredity may in fact be a product of evolution rather than a precursor \autocite{Bourrat2015}
	\item
	\autocite{Kauffman:1993kk} argued that self-organization (RAPN) can replace the genome
\end{itemize}

\paragraph{Desirable properties}\label{desirable-properties}

\autocite{Suzuki2003}:

\begin{itemize}
	\item
	The symbols or symbol ingredients be conserved (or quasi-conserved) in each elementary reaction, at least with the aid of a higher-level manager.
	\item
	An unlimited amount of information be coded in a symbol or a sequence of symbols.
	\item
	Particular symbols that specify and activate reactions be present.
	\item
	The translation relation from genotypes to phenotypes be specified as a phenotypic function.
	\item
	The information space be able to be partitioned by semi-permeable membranes, creating cellular compartments in the space.
	\item
	The number of symbols in a cell can be freely changed by symbol transportation, or at least can be changed by a modification in the breeding operation.
	\item
	Cellular compartments mingle with each other by some random process.
	\item
	In-cell or between-cell signals be transmitted in the manner of symbol transportation.
	\item
	Symbols be selectively transferred to specific target positions by particular activator symbols (strongly selective), or at least selectively transferred by symbol interaction rules (weakly selective).
	\item
	There be a possibility of symbols being changed or rearranged by some random process.
\end{itemize}

Desirable properties from \autocite{Faulconbridge2011} for an emergent chemistry:

\begin{itemize}
	\item
	Functional groups.
	\item
	Conservation of energy.
	\item
	\autocite[sec.4.4]{Faulconbridge2011} contains an interesting discussion mapping these desirable properties onto the emergent properties that are then required of an \gls{achem}.
\end{itemize}

Additional properties from \autocite{Hickinbotham2010}:

\begin{itemize}
	\item
	Novelty and innovation, specifically the ability for new molecules introduced to the chemistry to take part in reactions without needing changes to the \gls{achem}.
	\item
	Range of scales.
	\item
	Dynamic environment.
	\item
	Redundancy and degeneracy.
	\item
	Emergent complex properties.
	\item
	Unified molecular representation.
	\item
	Stochasticity.
	\item
	Emergent mutation rates.
\end{itemize}


EvoEvo project taking a similar approach, but from a higher level biological starting point (genotype-phenotype mappings)
http://evoevo.liris.cnrs.fr/about-evoevo-project/

\begin{itemize}
	\item Presupposes microbial evolution, ``at the level of genomes, biological networks and populations.''
	\item Focus on four specific properties of a genotype-phenotype mapping - Variability, Robustness, Evolvability, Open-endedness
	\item Later work to remove biological specificity to provide framework for applying EvoEvo to ICT problems
\end{itemize}

Previous work coming to consensus on conditions--e.g., rich generative mechanism, unlimited heredity, inexhaustible fitness landscape, emergence \autocite{Vasas2015}, and good genetic representation, ``sufficiently large world for every individual to be evaluated'', and a seed or starting point, (plus four specific conditions \autocite{Soros2014})

General difficulties with earlier work:
\begin{itemize}
	\item Somewhat arbitrary choices of elements of description
	\item Genotype/Phenotype, Selection,\ldots{} often based on goal of rationalizing existing descriptions, so not a re-examination
	\item Lack causality--so hard to use as mechanism
	\item Leave options and alternatives for implementer
	\item Sheer number of EA algorithms
\end{itemize}


\subsubsection{Tierra and Avida}

\autocite{Ofria2004}

Avida

``An approach to studying evolution...''

``According to Daniel Dennett, ``...evolution will occur whenever and
wherever three conditions are met: replication, variation (mutation),
and differential fitness (competition)''''


``(However, as Barton and Zuidema {[}3{]} note, general acceptance
will ultimately hinge on whether artificial life researchers embrace
or ignore the large body of population genetics literature.)''


Difference with GAs - natural organisms must replicate themselves to
pass on genetic information - ``final arbiter of fitness'', and
interaction with other organisms and with environment

Steen Rasmussen inspired by computer game core war - competing
segments of simplified assembly code in core memory. With change to
copy command to introduce mutations and hence evolutionary potential,
core world created. But system ``collapsed into a non-living state''
{[}non-living?{]} One possible reason - copying over existing
organisms

Tierra next year (relationship not stated) organisms had to allocate
memory first before using. Initial selective pressure only from rate
of replication. Sequential execution of organism code


Avida summer of 1993 - better metering and measuring, and parallel
code execution


\begin{itemize}
	\item
	
	``In principle, the only assumption made about these
	self-replicating automata in the core Avida software is that their
	initial state can be described by a string of symbols (their genome)
	and that they autonomously produce offspring organisms. However, in
	practice our work has focused on automata with a simple von Neumann
	architecture that operate on an assembly-like language inspired by
	the Tierra system.''
	
	\item
	
	Instruction, read, write, and flow control heads for relative rather
	than absolute addressing - bit like a Turing tape machine
	
	\item
	
	Many instructions grouped into instruction sets. Default set has 26
	instructions
	
	\item
	
	Every program is valid
	
\end{itemize}


Phenotypes - ``The primary mode of environmental interaction is by
inputting numbers from the environment, performing computations on
those numbers, and outputting the results. The organisms receive a
benefit for performing specific computations associated with
resources''

All very configurable, and complicated, but why? What rationale behind
choices? More of a testbed for experiments, e.g., `` in one experiment
we wanted to study a population that could not adapt, but that would
nevertheless accumulate deleterious or neutral mutations through
drift''


``The quest to halt adaptation is only one example of a special
feature in Avida; many more have been explored, and are continuously
being added to the source code. The most successful features are all
fully described in the documentation that comes with the software.''


\autocite{Lenski2003}


Modelled in Avida


``Our experiments demonstrate the validity of the hypothesis, first
articulated by Darwin and supported today by comparative and
experimental evidence, that complex features generally evolve by
modifying existing structures and functions.''



``Some simpler functions were accessible from the ancestor by
relatively few mutations, and these served as a foundation on which
more complex features were built. The foundational role of simpler
functions in the origin of more complex ones was evident in the
overlap of the genetic networks underlying their expression, and the
frequent loss of simpler functions as side-effects of mutations
yielding more complex function''


\autocite{Taylor2001}:


Derived from Thesis (Taylor:1999sc)

Criticism of Tierra
\begin{itemize}
	\item
	
	Not built around any particular theory - ``This weakness is not
	specific to Tierra, but is shared by most, if not all, of the other
	Tierra-like systems which have emerged over the last
	decade\ldots{}''
	
	\item
	
	In Ray's words, ``...this approach involves engineering over the
	early history of life to design complex evolvable organisms, and
	then attempting to create conditions that will set off a spontaneous
	evolutionary process of increasing diversity and complexity of
	organisms''
	
	
	\begin{itemize}
		\item
		
		Problem with `engineering over' is we don't understand the natural
		examples well enough to engineer them
		
		\item
		
		Similar criticism by Pattee1988 - ``simulations that are dependent
		on ad hoc and special-purpose rules and constraints for their
		mimicry cannot be used to support theories of life''
		
	\end{itemize}
\end{itemize}


General points

\begin{itemize}
	
	\item
	
	Phenotype fundamentally ``involves interaction with the environment
	(and that this is the essential distinction between the notions of
	phenotype and genotype - the latter being an informational
	concept)''
	
	\item
	
	Seed (proto-DNA) must itself be an indefinite heredity replicator
	{[}assumes that this is minimal starting point, rather than that
	this itself may evolve{]}
	
	\item
	
	Assume that early stages see A+B implicitly encoded in the
	environment {[}essentially because simpler than explicit mechanism,
	but little justification{]} ``At the early stages of an evolutionary
	process, however, we would not expect there to be mechanisms for
	explicitly decoding the proto-DNA\ldots{}''
	
\end{itemize}


\autocite{Taylor2001}:

Distinction between OEE and ``kinds of evolutionary innovation'' -
e.g., in Ray1977 innovations quite limited - e.g., parasitism emerged
because of system design and initial seeding conditions.

\begin{itemize}
	\item
	
	`fundamentally new' (labelled `creative') means new ways of sensing
	environment and interacting with it
	
	\item
	
	``From the point of view of the evolvability of individuals, the
	more embedded they are, and the less restricted the interactions
	are, then the more potential there is for the very structure of the
	individual to be modified. Recall that this is one aspect of my
	definition of creative evolution. Sections of the individual which
	are not embedded in the arena of competition are `hard-wired' and
	likely to remain unchanged unless specific mechanisms are included
	to allow them to change (and the very fact that specific mechanisms
	are required suggests that they would still only be able to change
	in certain restricted ways).``
	
	\item
	
	Similar to Pattee's semantic closure - ``organisms should be
	constructed `with the parts and laws of an artificial physical
	world'''
	
\end{itemize}

\autocite{Maley1999}

``we would like the evolutionary system, like life, to continue to
produce individuals of increasing complexity and diversity.'' -
although note, following McShea, that much of life is single-celled
and hasn't become much more complex in billions of years.


Focus on diversity (to make progress)


Some suggestions previously that diversity is bounded (at minimum, by
number of molecules available for biosphere, also by energy and
minimal populations and probably other things), and plateaus
(punctuated equilibrium). Indicates two time constants


\begin{itemize}
	\item
	
	fast - expansion to use available resources
	
	\item
	
	slow - innovations to open new adaptive space
	
\end{itemize}


Urmodel1 - neutral landscape, mutation, early stop before all niches
filled (to prevent edge effects)

\begin{itemize}
	\item
	
	32-bit genotype - mutation flips one-bit
	
	\item
	
	Get unbounded diversity, but no selection or heritable effect on
	fitness
	
\end{itemize}

Posits that need selection : ``Requirement 2 An open-ended
evolutionary system must embody selection'' - because ``fails to meet
one of the basic criteria of natural selection: the heritable
variation has no effect on fertility'' {[}ignoring use of fertility as
a fitness-analogue{]}, and from Bedau ``Requirement 3 ...continuing
(`positive') new adaptive activity'' {[}ignore neutral theory, and
accepts Bedau - perhaps to allow use of Anew as measure?{]}


Urmodel2 - natural selection: mutation, ``dissimilarity'' for
competitive advantage (justified by biological example of niche
overlap theory (Levins, 1968)) - no increase in Anew


Urmodel3 - selective sweep (hypothesis): parasites (mutations) and
hosts (fixed genotypes). Fitness on degree of match between parasite
and host bit patterns.


\begin{itemize}
	\item
	
	Claim shows unbounded activity - ``the first known artificial
	evolutionary system demonstrating unbounded evolutionary activity''
	
	\item
	
	Restricted by 32-bit genomes, no death
	
	\item
	
	But probably not a unique or even significant result - ``The only
	trick is to defer the point when the model hits its true asymptotic
	behaviour for long enough that the growth dynamics of the model are
	themselves asymptotic in some sense''
	
\end{itemize}


Urmodel4 - ``the most important aspect of an organism's environment
are the other organisms with which it interacts'' - add coevolution to
Urmodel3 by letting hosts mutate

Two ``distasteful aspects of Urmodel3'' leading to belief that metrics
aren't right


\begin{itemize}
	\item
	
	niches are imposed from outside, not endogenous - this becomes
	Requirement 4
	
	\item
	
	no surprise - claim is because not complex - ``A puddle of inert,
	multicoloured and diverse algae would not be nearly so inspirational
	as the rain forest.'' Again, a biological metaphor.
	
\end{itemize}



\subsection{Discussion}

\section{Approach, and guide to this work}

\TODO{rewrite, properly mark quotations + references!}

The field of artificial life is synonymous with simulation
\autocite[chap.2]{Aicardi2010}. In other forms of science however
practitioners make use of a number of tools, including experiments,
mathematical models, thought experiments, and simulations.
Each method is well suited to some types of
questions, and inappropriate for others. Is the use of simulation
justified for our proposed investigation?

Simulations are becoming central to some disciplines in natural history.
Ecology--\glspl{abm} or \glspl{ibm} (reviewed in
\autocite{DeAngelis2005}; also see
\autocite{Grimm:2006fk,Grimm:2005wd,Grimm:1999kf,Hogeweg:1990jz}).
\glspl{ibm} in Microbiolgy are seen as very close to Alife \autocite{Grimm:2009th}.

The value of simulation over experiments for
\gls{ibm} study lies in a reduction in costs; the difficulties in
cultivation of microbial populations (99\% of known species yet to be
cultivated), and significantly, that they form \quote{complex systems only
	poorly explained by reduction.}{\autocite{Ferrer:2008hv}} Emergence and
dynamic behaviours are important, and yet they are hard to capture with
mathematical models.

Types of investigation

\subsection{Thought experiments}\label{thought-experiments}

non-empirical, clarification, contradictions/dissonance, fast, cheap.

\subsection{Models}\label{models}

\quote{
	It is seldom the case in biology that a model is derived deductively
	from a more fundamental quantitative theory, with the possible exception
	of population genetics which has its foundations in evolutionary
	theory.}{\autocite{Krakauer2011}}

Models can be ``useful stop-gaps'' towards a theory, by providing a
testable body of data for experiments and predictions
\autocite{Krakauer2011}, and may be constructed either bottom-up (such
as in ABM) or top-down, by the application of constraints
\autocite{Krakauer2011}. Many types in \eg ecology--eleven according to
\autocite{Jorgensen2008}--of which fall into two main groups

Mathematical models: complexity, need for abstraction/assumptions
(\eg Fisher's famous equation describing the changes in allele
distribution under selection assumes independent genes--extending this
to realistic cases remains an open problem \autocite{Schuster2011}),
difficulty in handling dynamism/emergence. Cheap, fast. Non-empirical.

Simulations: bottom-up approaches, holistic, variability so diversity
closer to real systems, adaptive behaviour/changing
\autocite{Ferrer:2008hv}. Non-empirical data.

\subsection{Experiments}\label{experiments}

reductionism, conflation (difficulty in removing other factors e.g.,
Heinemann), expense, time (generations). Source of empirical data.

\quote{
	Although this may seem a paradox, all exact science is dominated by the
	idea of approximation.}{The Scientific Outlook, Bertrand Russell}

\subsection{Benefits}\label{benefits}

Unique ability to explore subject = unique object of enquiry e.g., study
of emergence, complex, self-organizing subjects. Biology stands alone in
the importance of emergence \autocite{Bersini:2006ve}, and the
interconnection of levels of analysis, \eg behaviour can influence gene
expression, and genes can affect behaviour \autocite{Krakauer2011}. As
summarized by \autocite{Krakauer2011}, when asking how much of biology
can be predicted bottom-up from the application of basic physical and
chemical laws--``This question is simple to answer: effectively zero.''

Unique method of enquiry, that is, properties that improve on existing
techniques, e.g., by relaxing assumptions

\subsection{The epistemological nature of simulations}\label{the-epistemological-nature-of-simulations}

Simulations seem to fall somewhere in between thought experiments or
abstract models, and experiments. They are also relatively novel; common
use has only come with increased access to digital computers.
Consequently the nature of simulation--what can be claimed as a result
of simulation, and what role may be played legitimately by simulation in
scientific discovery--is a hot topic for philosophers of science. As
might not be unexpected, there are two opposing positions taken, plus a
synthesis that claims the middle ground.

\newthought{Simulations are just calculators}\label{simulations-are-just-calculators}

a computational means to solve analytically intractable equations
\autocite[31]{Winsberg2010}, producing nothing new (just consequences
of what is ``fed in''\autocite{DiPaolo2000}), nothing empirical.

\quote{A simulation is ultimately
	only a high-speed generator of the consequences that some theory assigns
	various antecedent conditions.}{\autocite{Eldridge}, quoting from Dennett 1979 p192}

If a model, then might take many forms--analogy, model, pure exploration \autocite{Webb2009}.

Models are \quote{a purposeful representation. A model needs to have a
	purpose because otherwise there would be no way to decide what to
	include in it. A model's purpose is a filter: the model should not
	include anything not believed essential for explaining the phenomenon
	of interest}{\autocite{Grimm:2009th}}

\autocite{MaynardSmith1974} distinguishes between
``practical'' descriptions of ecological systems, or ``simulations'', and
theoretical ones: ``models''.

Simulations are aimed at answering specific questions, or analysing
particular scenarios. The more accurate the simulation however, and
hence the more valuable the results, the harder it is to generalize to
other cases. It is hard to understand the behaviour of complicated
simulations, and the causes of particular behaviours of interest may be
unclear if there are many variables in play.

Instead, \autocite{MaynardSmith1974} prefers the use of simple models, designed
to illuminate the ``causes of differences of behaviour between different
species or systems'' rather than ``assertions which are true of all
systems or of all species.''

Hughes 1999 would argue that simulations have genuinely ``mimetic''
character, particularly ones that present results graphically as
real-world systems do, that goes beyond plain number-crunching. They use
a variety of methods beyond calculation (such as graphics) to draw
inferences from data. They also incorporate approximations and creative
choices to make the problem tractable, which introduces need for
justification. Simulations need interpretation and justification--they
are not self-contained, their own justifications
\autocite[31]{Winsberg2010}

\newthought{Simulations are themselves an instance of the
	thing}\label{simulations-are-themselves-an-instance-of-the-thing}

That is, the thing is not a shadow but the object. The Animats are
actually alive, and therefore instances of biology.
\footnote{And this way leads us to the claims of Strong Alife--the simulation is actually alive.}
The simulation is a stand-in for the real world, and you can perform
experiments on it as would any other system
\autocite[31]{Winsberg2010}. As \autocite{Adami2002} says, describing
Avida,

\quote{
	These organisms, because they are defined by the sequence of
	instructions that constitute their genome, are not simulated. They are
	physically present in the computer's memory and live there. The world to
	which these creatures adapt, on the other hand, is
	simulated\ldots}{\autocite{Adami2002}}

They certainly have elements of uncertainty and error, like experiments.

TODO Paul Humphreys 1994 says Monte Carlo simulations are experiments; Von
Neumann ``replace a computation from an unquestioned theory by direct
measurement'' (quoted in TODO [28]{Winkler et al 1987}).

Norton and Suppe 2001 argue that they are experiments, when proper conditions met:
``Empirical
data about real phenomena are produced under conditions of experimental
control'', although ``lacking other data, we can never evaluate the
information that these experiments provide'' Hughes 1999 p.142 so no inherent epistemological force to his
argument. Validity solely depends on validation.

\newthought{A third-way, neither experimental or theoretical}\label{a-third-way-neither-experimental-or-theoretical}

\autocite[31]{Winsberg2010} or an Opaque Thought Experiment"
\autocite{DiPaolo2000}. A common view, among others Dowling 1999, 264:
simulation is like theory as about ``manipulating equations'' and
``developing ideas'' but like experiments as ``fiddling with machines'',
``trying ideas out'', ``watching to see what happens.'' Simulation is a
form of Kuhn's theory articulation or ``model building''--making
principles apply to local, concrete systems in the real world. In this
view, \quote{Simulation is a process of knowledge creation}{\autocite[6]{Winsberg2010}}

\subsection{The legitimate role of simulation in scientific
	enquiry}\label{the-legitimate-role-of-simulation-in-scientific-enquiry}

Following this third way, simulations might be seen as a source of new
hypotheses \autocite{Eldridge}. The fundamental question remains
however: what is the exact relationship between world and simulation? If
the simulation exhibits an analogous behaviour, under a set of
assumptions, to the real world, then we can contend that there exist
similar mechanisms in the real world to the assumptions in our model.
TODO Noble 1997, quoted in \autocite{Eldridge}). The example given by
\autocite{Eldridge} is Boids \autocite{Reynolds1987} where flocking behaviour in
birds is very similar to that that results from a set of three simple
rules in a simulation.

However, \autocite{Eldridge} identifies three problems with this
approach: first, logically, similarity does not require congruence
\autocite{Weitzenfeld1984}, second, how is the degree of similarity to
be assessed, and third, the impossibility of proving that a simulation
is an accurate model of a theory. That is, where does attribution lie?
Is the result a result of the underlying theory or a quirk of the
implementation? There may be no way of resolving this absolutely as it
may not even be possible to distinguish between the two
\autocite{DiPaolo2000}--we cannot prove correctness through testing.

TODO On the other hand, Taylor (1989) in \autocite{Webb2009}, argues for
``pure exploration'' or ``exploratory tools'' that do not need
justification, and that may be used to generate ``new questions to ask,
new terms to employ, or different models to construct'' (Taylor 1989,
p122). But in this case you might reasonably argue that any insights are
``insights about a mathematical system'' not necessarily insights into
the real-world (123–124 Taylor 1989).

In practice then, any form of model claiming a significance beyond its
own self must show a correspondence with the thing it claims to be
modelling.

\quote{However, existence proofs clearly do require comparisons
	between model results and empirical data. One cannot evaluate the claim
	that phenomenon X requires condition Y unless one can show that
	phenomenon X is actually produced (with or without Y). And the claim or
	proof will be stronger or weaker depending on how well the simulated X
	matches the real X; for example, demonstrating successful behaviour in
	the same physical situation as the animal.}{\autocite[278]{Webb2009}}
The strength of our belief depends on the degree of similarity.

\section{Previous publications}\label{previous-publications}

A version of Reactant and Product Strategies \cref{reactant-and-product-strategies} was published as \cite{Young2015},
and material from \cref{model-validation} in \cite{Young2013}.

ToyWorld is available under an GNU GPL v2 open source licence from GitHub \cite{toyworld}.

\section{Contributions}\label{contributions}

\begin{enumerate}
	\item
	Open-sourced Artificial Chemistry model
	\item
	Progress towards useful heredity in an artificial system
	\item
	Progress towards OEE in artificial system--evolution compatible with
	OEE without necessarily showing OEE (which is hard to measure and
	prove)
	\item
	Demonstration of formation of ACS in an artificial chemistry (previous
	work with ODEs e.g. \autocite{Hurndall2014} not re-usable in that form)
\end{enumerate}
