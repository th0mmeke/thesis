\documentclass[]{report}

\usepackage{titlesec,enumitem}

\titleclass{\chapter}{straight}

\titleformat{\section}[hang]
{\normalfont\large\bfseries}{\thesection}{1em}{}

\titleformat{\chapter}[hang]
{\normalfont\large\bfseries}{\thechapter}{1em}{}
\titlespacing{\chapter}{0pt}{*4}{*1.5}

\title{}
\author{}

\begin{document}
	
\abstract{
}
\clearpage
\chapter{Introduction}

The underlying motivation for this work is to understand the requirements for onset of artificial evolution; how robust, interesting, adaptive evolution can be started and undergo self-improvement in the non-living world.

Natural evolution is the foundation for several related bodies of literature, in Computational Intelligence and Artificial Life, as well of course of the study of evolution in natural systems. As many have observed, there are major differences between the original inspiration and the spin-offs. Biological evolutionary systems are complex, contingent, emergent, endogenous and general. \\todo{By contrast, as a technology, artificial evolutionary systems are exogenous, problem-specific and designed.}

A convenient shorthand capturing these differences labels the biological original as bottom-up and the derived fields as top-down. If the top-down approach has failed, then perhaps there might be mileage in returning to biology, our sole example of a general self-improving, self-organizing system.

But as biology is contingent, what is the most useful starting point for a recapitulation? Assume first of all that we are in the domain of software, rather than hardware or wetware, for practical reasons. And also assume that biological evolution does not depend upon embodiment - that the outcome can be adequately modelled at a conceptual level without being grounded in either the physical world, or in contingent biology such as enzymes, proteins and genes.

Then our belief is that the evolution of evolution itself can and will emerge under certain conditions, and that an emergent approach will be more successful than the predominant design-driven top-down approach today.

\section{Goal}

The overall problem we address is to \textit{determine the conditions under which ongoing evolution might emerge in a software system}. 

Ongoing evolution means something more than continual change of population frequencies; instead we refer to the goal of a process that produces \\todo{self-tuning, improving etc} interesting, novel, and creative results, as does natural biological evolution. Emergent references the difference between a top-down designed mechanism imposed upon the system, and one that is a natural result of interactions between the system elements (the reason for this will become clear in later discussion concerning the importance of self-referentiality.) Finally, our domain is software rather than wetware or hardware; simulations and models, rather than elements embedded in the physical world.

This is clearly a broad and ambitious goal, one more suited to a research programme than a single thesis, and a goal that is additionally the subject of a complex body of related and previous work. We shall return to a more specific and pragmatic problem description in the later section, \ref{research-questions}, after presenting a summary of the most relevant literature.

\section{Previous work}
The previous sections have established the context, and incidentally identified the primary areas of relevant literature. From biology, the main threads concern the origins of life, and the theory of evolution. Natural selection provides the primary mechanism for evolution in biology, although it is not the only process at work -- genetic drift \cite{XXX} and neutral theory \cite{Kimura} provide a counterpoint. In artificial systems \\todo{EC etc}, researchers have used biological evolution as an inspiration, gradually over time diverging into a field with the beginnings of its own motivating theory, now only loosely connected with its origins in the natural world. These areas are discussed in more detail in \ref{context} to aid in understanding the context behind the remaining sections of the work.

In artificial systems, some researchers have recently returned to a reappreciation of how biological evolution generates robust, novel, creative outcomes, unlike those seen in current artificial evolution. This has led to a renewed interest in understanding the principles behind biological evolution so that artificial systems can capture some of these admirable properties; the difference now is that the transference is sought at the level of concepts and principles rather than in the historical inspiration of specific biological elements and structures, many of which are contingent and perhaps even arbitrary; certainly complex.

This chapter expands upon previous work that is directly relevant to our problem overall, the onset of evolution in software systems. 

The review is organized roughly in alignment with the research goal: first an examination of work that abstracts evolutionary principles; next a summary of what it means for an artificial system to experience ongoing creative evolution; and finally how emergence is relevant to the goal. Some interesting open problems are summarized in the conclusion to this section.

Previous work that relates to a particular section of the thesis is presented in that section, for clarity, and to make clear the distinction beween the work of others and our own contributions.

\subsection{Evolutionary principles}
\subsection{Ongoing evolution in artificial systems}
\subsection{Emergence in artificial evolutionary systems}

\subsection{Discussion}

Two particular open problems are apparent from this review. First, the model of \cite{Bourrat2015} assumes that the problem that evolution is learning is perfectly understandable, and that there is one and only one optimal solution. This is a corollary of the model design where fitness is absolute and unchanging - if fitness represents (as it does) an implicit relationship between an entity and its environment, then in Bourrat's model this relationship is also fixed and unchanging. Evolution is omniscient with full visibility into the world. However, in the real world and in the artificial domains of interest, the relationship between entity and environment is less sure. The environment itself may either change, or be uncertain. This is unexplored by Bourrat.

The second open problem is the demonstration of a self-referential mechanism to provide the fidelity property of Bourrat, or the inheritance and variation properties assumed by other models, that is emergent and embedded in the simulation world. Two important observations can be made about this copying mechanism: first, it must be modifiable by evolution and so must be self-referential, accessible to evolution and therefore must be made of the same things as everything else that is subject to evolution in the system - it is self-referential and constructed of the common elements of the world. It cannot be an exterior independent element. Second it combines inheritance and variation, and in that recollects biological replication mechanisms, such as DNA replicase, which result in a varying and imperfect transferral of information from parent to child. 

\section{Problem and research questions}\label{research-questions}

From this discussion, our overall goal of emergent ongoing evolution can be more pragmatically reduced to the following research questions:

\begin{enumerate}[label=RQ\arabic*:]
	\item Does inheritance still emerge from variation when the fitness-environment relationship is not fixed?
	\item Is there a plausible implementation for a fidelity mechanism in a software system?
\end{enumerate}

\section{Approach, and guide to this work}

The overall structure of this thesis broadly follows the research questions. 

In \ref{part-one} we first confirm the results of previous work regarding the relationship between fidelity and fitness in fixed environments. Then in a new contribution, we extend this relationship to changing environments, and explore the effect of a shifting fitness-enviroment relationship on the behaviour of fidelity. We conclude that fidelity acts to preserve variation in the population, which supports the hypothesis that fidelity can replace variation and inheritance in the description of an evolutionary system across a broad range of environments.

Next, in \ref{part-two}, we move from models and theory to implementation. From a review of earlier work, and following arguments made by other researchers, we posit that an ongoing evolutionary system must be emergent and embedded. We then investigate what this might mean for the implementation of a fidelity mechanism in a software system, and conclude like others that it must be self-referential. From these two strands of argument, we suggest that there are good grounds to adopt an Artificial Chemistry for the implementation of an ongoing evolutionary system, and that biological replicase molecules provide a good model for an embedded, self-referential variable copying mechanism. Replicases are both sources of variation and subject to evolution, and this realization brings us, almost in a full circle, to a sufficient mechanism for ongoing self-improvement--an imperfect copy mechanism constructed out of raw elements in the world.

We then introduce ToyWorld, a new semi-realistic Artificial Chemistry, that is compatible with these arguments. ToyWorld, like other atom/molecule based chemistries, has the significant benefit of being analogous to real-world chemistry with the result that not only is it interpretable in familiar terms, but that it might be simpler to take abstract models of replicases from biology and seed them in ToyWorld than it might be with other less related Artificial Chemistries. We show that ToyWorld is capable of emergent behaviour and suggest that it should be possible to build a replicase equivalent in ToyWorld to seed ongoing evolution.

\chapter{Context}
In this chapter we briefly review work in a variety of related fields for context and background to the main problem; the intention is that this material will by way of preamble ``set the stage'' for the entrance of the principal actors in later chapters.

Our supporting characters come from Biology and Computational Intelligence: they will first attempt to draw a distinction between the living and non-living worlds before concluding unfortunately that the difference is clear where it is unneeded, and opaque where it is most required. Next comes a brief survey of some aspects in Biological Evolution, significant as so much of Artificial Evolution is grounded in Biology, and equally relevant as the explanations of Biological Evolution are so tightly linked to biological contingency and complexity that any parallels must be drawn carefully indeed. Finally, from Computation Intelligence, we first touch on the conceptual similarities between emergentist views of Biological Evolution and Artificial General Intelligence, before concluding with a brief review of the area of Computational Intelligence perhaps most overtly-inspired by natural evolution, Evolutionary Computation.

\part{Evolutionary model}
\chapter{Introduction}\label{part-one}
\section{Assumptions}
\section{Variation (and Inheritance)}
\section{Selection}
\section{Hypothesis: Variation and Inheritance and Selection are sufficient for Evolution}
\section{Hypothesis: Correlated Variation and Selection are sufficient for Inheritance}
\section{Predictions}
\section{Alternative explanations}

\chapter{Simulation model}
\section{Base model}
\section{Initial conditions and settings}

\chapter{Initial screening}
\section{Experimental design}
\section{Factors and levels}
\section{Probability of selection and probability of reproduction}
\section{Number of offspring}
\section{Fidelity correlation}
\section{Derive parameter}
\section{Conclusions}

\chapter{Test under fixed conditions}
\section{Emergence of inheritance from low-start initial conditions}
\section{Confirmation of inheritance under high-start conditions}
\section{Variance of Fidelity reduces under low-start conditions}
\section{Confirmation that variance decreases under high-start conditions}

\chapter{Adaptation to changing conditions}
\section{Introduction}
If we accept that evolution is a mechanism to learn from the environment, then a changing environment provides an interesting challenge. 
\section{Experimental design}
Alongside our existing evolutionary model we now introduce an environmental model. This model describes the change of fitness resulting from enviromental change at each generation of the evolutionary model. Fitness is thus extended from purely a property of an entity to representing the relationship between entity and environment.

The model must describe two particular elements of this relationship: first,the scope of the change, and second, the shape of the change. In our evolutonary model we can only group entities in three non-trivial ways:
\begin{enumerate}
	\item The group of all entities.
	\item A single-member group for each entity.
	\item A group for each set of ``related'' entities, where the most natural and obvious relation is that between parent and child; this is unambiguous and straightforward in our model where each entity has only one parent. We refer to a group of entities related by inheritance as a \emph{lineage}.
\end{enumerate}

We can apply the scope of the change to each of these three groups.

The shape of change is less constrained, and the space of all potential changes at any timestep effectively limitless: in fact, the potential change $\delta$ at timestep $t$ is $\delta_t\in R$. Simply taking a random sample from this space at each step is unlikely to result in enough resolution to test any relevant hypothesis. At the other extreme, taking only a small number of predetermined changes is likely to lead to a sampling fallacy where the choices bias the conclusions.

Instead, we need a way to parameterize the set of interesting enviromental changes so we can instead sample from a constrained but not predetermined parameter space. The range covered by the parameter space should include both predictable and unpredictable changes as the difference between the two is core to our hypothesis.

Our chosen method is to represent environmental change as a parameterized timeseries of particular form, as follows. Environmental change is modelled as an AR(1) or first-order autoregressive timeseries, with each timestep corresponding to one evolutionary generation. Specifically, we can describe the evolutionary change at each timestep as a function of the previous timestep:

$x_t = \Theta x_{t-1} + e_t$

where $x_t$ is the change at timestep $t$, $\Theta$ is the AR coefficient, and $e_t$ is a random, normally distributed, error component around a mean of $0$, where $e_t\stackrel{iid}{\sim}N(0,\sigma^{2}_e)$

This series allows us to represent a broad range of environmental changes:

\begin{itemize}

	\item Each timeseries is completely specified by only two parameters, $\Theta$ and $\sigma_e$.
	\item $\Theta$ in an autoregressive timeseries can be interpreted as specifying stability or smoothness, allowing us to control the predictability of the change.
	\item The timeseries is composed of two independent elements, one predictable (driven by $\Theta$) and the other ($sigma_e$) random and unlearnable. By changing the ratio between the two we can examine the performance of the evolutionary algorithm on a continuum of predictability.
	\item An AR timeseries has the property of stationarity, meaning that the mean of the series is constant through time. However, as we apply the series values as deltas to element fitness, fitness can be non-stationary, or in other words show a long term trend. This allows a simple non-differencing timeseries to describe a steady improvement or worsening in fitness, something we'd like to include in our sample environments.
	\item As a corrollary of stationarity, the range is strongly determined by the initial parameters. This is a useful property as it means that with appropriate parameter choices no scaling of the range is required. 
\end{itemize}

Introduction of abrupt change to the environmental change model. A simple model would introduce an abrupt change with probability $p$ at each generation, where the change would form a new `concept` (in machine learning terminology, appriately as evolution can be seen as a learning system). Instead of the environment changing in a predictable and describable way from one generation to another, the change could not be predictable from the earlier history.

Our hypothesis, that fidelity is related to learnability, can now be refined in terms of the predictability of the environment, where predictability is proportional to the ratio of $\Theta$ to $\sigma_e$.

From the hypothesis we make these predictions:
\begin{enumerate}
	\item Fidelity will be at a minimum in conditions of maximum unpredictability, that is for timeseries where $\Theta=0$ and $\sigma_e>0$.
	\item Fidelity is proportional to $\frac{\Theta}{\sigma_e}$.
\end{enumerate}
	
\section{Results and discussion}
\section{Future work}

\chapter{Elimination of alternative explanations}
\section{Variation alone is sufficient for Inheritance}
\section{Selection alone is sufficient for Inheritance}
\section{Variation and Selection, without property correlation, is sufficient for Inheritance}

\chapter{Conclusions}

\part{ToyWorld}
\chapter{Introduction}\label{part-two}

Artificial Chemistries of discrete atoms provide an interesting testbed for investigating various evolutionary phenomena. Fundamentally, they provide a tuneable evolutionary system, capable of highly complex behaviour, built around familiar metaphors (real-world Chemistry, and potentially Biology). A set of rules describing how atoms interact gives rise to emergent forms -- molecules. At a higher level, these molecules, under the same interaction rules, also interact in patterns -- reactions.

Still higher emergent levels emerge under favourable conditions. Reactions may form cycles, where a sequence eventually returns to an earlier product. Our interest is in identifying the factors that influence the emergence of these higher levels. Cycles in particular are interesting as many biological processes are cyclical. Replication, resulting in an exact copy of an entity, is a macro-example of a cycle; metabolism is another. Building on the apparent correspondence between higher emergent levels in Artificial Chemistry evolution and Biology, we believe like others (e.g., \cite{Steel2013}) that cycles, of some form, are a necessary building-block for more complicated structures in Artificial Chemistries.

By demonstrating the formulation of cycles, we suggest that ToyWorld is capable of emergent behaviour.

\section{Contributions}

\chapter{Artificial Chemistries}
One advantage of molecular Artificial Chemistries as pointed out by \cite[5]{Funes2001} is that it is easier to evaluate solutions in a domain close to the real-world as opposed to a purely symbolic or abstract domain such as a lambda-calculus, or a programmatic environment like Tierra. When contemplating difficult problems such as complexity our intuition can be helpful, but only in situations close enough to our normal experience for it to be relevant.

\section{Introduction}
\section{Classification of Artificial Chemistries}

\chapter{ToyWorld}

\section{Introduction}
ToyWorld, our Artificial Chemistry for the exploration of emergent behaviours, was first introduced in \cite{Young2013}. The elements of the model--Atoms, Molecules, Reactions, a Reaction Vessel--are recognisable from real-world chemistry, but in highly simplified forms.

Although there is no need for ToyWorld to be faithful to standard Chemistry, a degree of familiarity helps, but only in so far as the analogy is consistent. Therefore we endeavour to maintain a basic correspondence wherever possible. However, there is no requirement to provide chemically-realistic results--our model cannot be used to investigate real-world chemical behaviours. There is another reason why familiarity may be important: following our earlier proposal that a replicase-like mechanism would fulfil the requirements for a self-referential copying mechanism, it might be fruitful to base any implementation in ToyWorld on a simplified version of a real biological replicase. Clearly, a simple mapping from real-world chemistry to the rules in our Artificial Chemistry would improve the feasibility of this idea.s

There is also yet sanother reason. The model has many degrees-of-freedom, and these must be constrained by parameter choice before the model can be used in simulation. Some values are important to our thesis, and so are considered true independent variables in our investigation. These are examined fully. The remainder however are those to which the simulation is insensitive, but still must be specified. For these we prefer real-world values rather than arbitrary artificial values. In short, where we consider something important, we investigate. Where we think it less important, we use a consistent set of pre-existing values -- real-world chemistry.

s\section{Atoms and molecules}
\section{Energy Model}
\section{Reactant and product selection strategies}
\section{Reactant selection strategies: selecting reactants for a reaction}
\section{Product selection strategies: determining the products of a reaction}

\chapter{Model Validation}
\section{Introduction}
\section{Results and Discussion}
\section{Conclusions}

\chapter{Reactant and Product Strategies}
\section{Introduction}
\section{Experiment Design}
\section{Results}
\section{Conclusions}

\chapter{Discussion}

\chapter{Introduction}
\section{Requirements for chemistry}
\section{Previous work}
\section{Match to requirements}
\section{Chemistry selection}

\chapter{Discussion and Conclusions}
\section{Future work}

\end{document}
