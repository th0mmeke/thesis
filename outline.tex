\documentclass[]{report}

%opening
\title{}
\author{}

\begin{document}
This work is motivated by the vision of achieving a self-improving, self-organizing, system. There are probably a variety of ways this could be achieved, but other than in very specific domains, it remains an open problem.

Our goal then is to determine the (or perhaps, a) most parsimonious starting point that allows evolution to begin in an artificial system. Evolution is generally thought to require three components--variation, inheritance and selection, where variation and inheritance together relate to the correlation between parent and child, and selection to the relationship between entity and evolutionary success. Previous work has shown that inheritance, instead of being a prerequisite for evolution, can in fact be a product of evolution, emerging from the interaction of variation and selection.

The main result of this thesis to show that not only can inheritance emerge, but that the degree or quality of inheritance is also emergent, influenced by the selection relationship. 

Fidelity provides variation and inheritance. The fidelity property in the model represents a fidelity value that changes in response to evolution - it is a self-referential property of the model - and the model overall therefore models the evolution of evolution.

The corollary of this claim is that this model forms a sufficient condition for the evolution of evolution, and that a system that meets this condition will be capable of ongoing self-improvement. 

Our fidelity property therefore combines two separate concepts: first, it must be modifiable by evolution and so must be self-referential, accessible to evolution and therefore must be made of the same things as the everything else that is subject to evolution in the system - it is self-referential and constructed of the common elements of the world. It cannot be an exterior independent element, divorced from the world. Second it combines inheritance and variation, and in that recollects biological replication mechanisms, such as DNA replicase, which result in a varying and imperfect transferral of information from parent to child. In fact, these biological mechanisms are both sources of variation and subject to evolution, and this realization brings us, almost in a full circle, to a sufficient mechanism for ongoing self-improvement--an imperfect copy mechanism constructed out of raw elements in the world.

In Part 2 we construct a minimally complex model of generalized evolution where each element in the population has two properties only - fitness, representing the quality of the relationship between entity and the (abstract) environment, and fidelity, which represents the degree of similarity between the entity and its parent.

Although minimal in description, this model over time shows complex and non-obvious emergent behaviour.

First, we examine the sensitivity of the parameterized model to levels of the main parameters using a fractional-factorial screening experiment design.

Using these settings we then confirm earlier results showing the emergence of perfect inheritance under fixed or unchanging environments.

We next extend the earlier result from a fixed relationship between entities and the environment to one in which the environment, and hence the relationship, changes. Here we show that the degree of inheritance is related to the degree of environmental change, demonstrating the balance between fitness and robustness we earlier hypothesised.

We show by simulation that the degree of inheritance is tuned by evolution to balance fitness and robustness, maximizing fitness in unchanging environments where there is little penalty to reduced diversity, and maintaining a more diverse population in changing environments where diversity provides robustness to environmental change. This balance emerges unprogrammed from the underlying model.

Next, in Part 3, we make the connection between the properties of this evolutionary inheritance and a plausible example mechanism  - a replication and variation operator under evolutionary control.

We describe the first steps towards a system compatible with this example copy mechanism, and speculate that this direction might lead more directly to transformative evolution that alternative methods based either directly on biology, or on rather arbitrary top-down design decisions.

\chapter{Introduction}

\section{Motivation}
The underlying motivation for this work is to understand the onset of evolution; how robust, interesting, adaptive evolution can be started and undergo self-improvement in the non-living world.

Natural evolution is the foundation for several related bodies of literature, in Computational Intelligence and Artificial Life, as well of course of the study of evolution in natural systems. As many have observed, there are major differences between the original inspiration and the spin-offs. Biological evolutionary systems are complex, contingent, emergent, endogenous and general. As applied to technology, evolutionary systems are exogenous, problem-specific and designed.

A convenient shorthand capturing these differences labels the biological original as bottom-up and the derived fields as top-down. If the top-down approach has failed, then perhaps there might be mileage in returning to biology, our sole example of a general self-improving, self-organizing system.

But as biology is contingent, what is the most useful starting point for a recapitulation? Assume first of all that we are in the domain of software, rather than hardware or wetware, for practical reasons. And also assume that biological evolution does not depend upon embodiment - that the outcome can be adequately modelled at a conceptual level without being grounded in either the physical world, or in contingent biology such as enzymes, proteins and genes.

Then our belief is that the evolution of evolution itself can and will emerge under certain conditions, and that an emergent approach will be more successful than the predominant design-driven top-down approach today.

\section{Problem}
The specific problems we address are the following:
\begin{itemize}
	\item Is there a simple set of conditions that enable evolution to emerge in a non-living, artificial, system?
	\item Can this also lead to the evolution of the evolutionary mechanism itself? Or in other words, can we see a pathway towards a self-improving, open-ended, evolutionary system?
\end{itemize}

\chapter{Context}
In this chapter we briefly review work in a variety of related fields for context and background to the main problem; the intention is that this material will by way of preamble ``set the stage'' for the entrance of the principal actors in later chapters.

Our supporting characters come from Biology and Computational Intelligence: they will first attempt to draw a distinction between the living and non-living worlds before concluding unfortunately that the difference is clear where it is unneeded, and opaque where it is most required. Next comes a brief survey of some aspects in Biological Evolution, significant as so much of Artificial Evolution is grounded in Biology, and equally relevant as the explanations of Biological Evolution are so tightly linked to biological contingency and complexity that any parallels must be drawn carefully indeed. Finally, from Computation Intelligence, we first touch on the conceptual similarities between emergentist views of Biological Evolution and Artificial General Intelligence, before concluding with a brief review of the area of Computational Intelligence perhaps most overtly-inspired by natural evolution, Evolutionary Computation.

\chapter{Previous work}
The previous sections have established the context, and identified the primary areas of relevant literature. From biology, the main threads concern the origins of life, and the theory of evolution. Natural selection provides the primary mechanism for evolution in biology, although it is not the only process at work -- genetic drift and neutral theory provide a counterpoint. In artificial systems, researchers have used biological evolution as an inspiration, gradually over time diverging into a field with the beginnings of its own motivating theory, now only loosely connected with its origins in the natural world.

The research emphasis in artificial systems has more recently returned to a reappreciation of how biological evolution generates robust, novel, creative outcomes, unlike those seen in current artificial evolution. This has led to a renewed interest in understanding the principles behind biological evolution so that artificial systems can capture some of those admirable properties; the difference now is that the transference is sought at the level of concepts and principles rather than in the historical inspiration of specific biological elements and structures, many of which are contingent and perhaps even arbitrary; certainly complex.

This section now expands upon previous work that is directly relevant to our specific problem, the onset of evolution in artificial systems, in two areas:

\begin{itemize}
	\item General evolutionary models
	\item Specific work in artificial systems
\end{itemize}

\chapter{Introduction}
\section{Assumptions}
\section{Variation (and Inheritance)}
\section{Selection}
\section{Hypothesis: Variation and Inheritance and Selection are sufficient for Evolution}
\section{Hypothesis: Correlated Variation and Selection are sufficient for Inheritance}
\section{Predictions}
\section{Alternative explanations}

\chapter{Simulation model}
\section{Base model}
\section{Initial conditions and settings}

\chapter{Initial screening}
\section{Experimental design}
\section{Factors and levels}
\section{Probability of selection and probability of reproduction}
\section{Number of offspring}
\section{Fidelity correlation}
\section{Derive parameter}
\section{Conclusions}

\chapter{Test under fixed conditions}
\section{Emergence of inheritance from low-start initial conditions}
\section{Confirmation of inheritance under high-start conditions}
\section{Variance of Fidelity reduces under low-start conditions}
\section{Confirmation that variance decreases under high-start conditions}

\chapter{Test under changing conditions}
\section{Experimental test}
\section{Experimental design}
\section{Results and discussion}
\section{Future work}

\chapter{Elimination of alternative explanations}
\section{Variation alone is sufficient for Inheritance}
\section{Selection alone is sufficient for Inheritance}
\section{Variation and Selection, without property correlation, is sufficient for Inheritance}

\chapter{Conclusions}

\chapter{Introduction}
\section{Contributions}

\chapter{Artificial Chemistries}
\section{Introduction}
\section{Classification of Artificial Chemistries}

\chapter{ToyWorld}
\section{Introduction}
\section{Atoms and molecules}
\section{Energy Model}
\section{Reactant and product selection strategies}
\section{Reactant selection strategies: selecting reactants for a reaction}
\section{Product selection strategies: determining the products of a reaction}

\chapter{Model Validation}
\section{Introduction}
\section{Results and Discussion}
\section{Conclusions}

\chapter{Reactant and Product Strategies}
\section{Introduction}
\section{Experiment Design}
\section{Results}
\section{Conclusions}

\chapter{Discussion}

\chapter{Introduction}
\section{Requirements for chemistry}
\section{Previous work}
\section{Match to requirements}
\section{Chemistry selection}

\chapter{Discussion and Conclusions}
\section{Future work}

\end{document}
