\chapter{Previous work}

The previous sections have established the context, and identified the primary areas of relevant literature. From biology, the main threads concern the origins of life, and the theory of evolution. Natural selection provides the primary mechanism for evolution in biology, although it is not the only process at work -- genetic drift and neutral theory provide a counterpoint. In artificial systems, researchers have used biological evolution as an inspiration, gradually over time diverging into a field with the beginnings of its own motivating theory, now only loosely connected with its origins in the natural world.

The research emphasis in artificial systems has more recently returned to a reappreciation of how biological evolution generates robust, novel, creative outcomes, unlike those seen in current artificial evolution. This has led to a renewed interest in understanding the principles behind biological evolution so that artificial systems can capture some of those admirable properties; the difference now is that the transference is sought at the level of concepts and principles rather than in the historical inspiration of specific biological elements and structures, many of which are contingent and perhaps even arbitrary; certainly complex.

This section now expands upon previous work that is directly relevant to our specific problem, the onset of evolution in artificial systems, in three areas:

\begin{itemize}
	\item Generalisations from biological and artificial evolution
	\item General evolutionary models
	\item Specific work in artificial systems
\end{itemize}




\section{Generalisations from biological and artificial evolution}

\autocite{Paixao2015}:

\begin{itemize}
	\item
	
	attempt to unify evolutionary computation and population genetics -
	based on common conception of evolutionary process
	
	\item
	
	Related work in Population Genetics
	
	
	\begin{itemize}
		\item
		
		PG evolution described by dynamics of allele or genotype frequencies
		
		\item
		
		Lewontin 1964 equation for canonical evolutionary process - genotype
		frequencies, based on common ``genetic operator'' for rate of
		generating a genotype x from parents y and z (combines mutation,
		recombination etc)
		
		\item
		
		This and other models biologically focussed - where ``selection
		assumes a particular form''
		
		\item
		
		Another approach to PG quantitative genetics - phenotypic view of
		trait evolution. Useful in animal breeding. Doesn't address
		mechanism (genetics)
		
		\item
		
		Price equation - very general, useful for that reason in comparing
		models
		
	\end{itemize}
	\item
	
	Any evolutionary process describes ``population undergoing changes
	over time based on some set of transformations''
	
	\item
	
	A transformation can be decomposed into a collection of (stochastic)
	operators. Operator can be described as a probability distribution of
	potential outcomes; and evolutionary process as a trajectory through a
	space of distributions.
	
	\item
	
	Process described both as a sequence of population transformations,
	and as distribution transformations
	
	\item
	
	Individuals described with both genotype and phenotype; gp mapping
	between them. Some operators act on genotype others on phenotype
	
	\item
	
	Various operators defined - for selection (uniform, proportional,
	tournament, truncation, cut, replace); variation (mutation) (uniform,
	single-point), variation (recombination) (one-point crossover, k-point
	crossover, uniform crossover, unbiased variation)
	
	\item
	
	Most models satisfy five given mathematical properties
	
	
	\begin{itemize}
		\item
		
		V1 - uniformity preserving - no change to distribution if uniformly
		distributed
		
		\item
		
		M1 - mutation acts on individuals
		
		\item
		
		M2 - mutation can generate whole search space ie an ergodic operator
		(defining characteristic of mutations)
		
		\item
		
		R1 - recombination preserves allele frequencies (in expectation)
		
		\item
		
		S1 - selection doesn't change individuals
		
	\end{itemize}
	\item
	
	Demonstrates how existing ``classical models in theoretical population
	genetics and in the theory of evolutionary computation'' can be mapped
	into this framework. Most of PG models can be represented
	(unsurprising as most variants of classical models that have been
	demonstrated in framework); some topic-specific EC models could not be
	- but not ones that have a relationship with PG and so of little
	interest to biologists. GP models omitted for reasons of balance
	between simplicity and inclusiveness
	
	\item
	
	Leaves MOEAs for later work
	
	\item
	
	``In an Estimation Distribution Algorithm (EDA), the algorithm tries
	to determine the distribution of the solution features, e.g.
	probability of having a 1-bit at a particular position, at the
	optimum. Some EDAs can be regarded abstractions of evolutionary
	processes: instead of generating new solutions through variation and
	then selecting from these, EDAs use a more direct approach to refine
	the underlying probability distribution. The perspective of updating a
	probability distribution is similar to the Wright--Fisher model.''
	
	
	\begin{itemize}
		\item
		
		Close similarity between simplest EDA (the Univariate Marginal
		Distribution Algorithm) and Linkage equilibrium models in population
		genetics as pointed out in Chastain2014 (``We demonstrate that in
		the regime of weak selection, the standard equations of population
		genetics describing natural selection in the presence of sex become
		identical to those of a repeated game between genes played according
		to multiplicative weight updates (MWUA), an algorithm known in
		computer science to be surprisingly powerful and versatile. MWUA
		maximizes a tradeoff between cumulative performance and entropy,
		which suggests a new view on the maintenance of diversity in
		evolution.'')
		
		\item
		
		AdaBoost is a form of MWUA
		(\href{https://www.cs.princeton.edu/~arora/pubs/MWsurvey.pdf}{\emph{https://www.cs.princeton.edu/\textasciitilde{}arora/pubs/MWsurvey.pdf}})
		
	\end{itemize}
\end{itemize}

\autocite{Barton2014}

\begin{itemize}
	\item
	
	Experts in MWUA equivalent to alleles in NS
	
	\item
	
	In selection, ``frequency of each type {[}allele{]} is simply
	multiplied by its relative fitness; which corresponds precisely to
	MWUA''
	
	\item
	
	MWUA ``maximizes the sum of two quantities: the expected total fitness
	of the chosen allele, summed over past generations, plus the entropy,
	which is a measure of the allelic diversity.''
	
	\item
	
	Fitness differences increase over time, and so population converges
	towards type with best performance; entropy term acts to slow down
	this convergence
	
	\item
	
	Problems
	
	
	\begin{itemize}
		\item
		
		because MWUA is deterministic, and some interesting problems e.g.,
		recombination as a mechanism for reintroducing gene combinations
		lost through drift in finite populations, are essentially
		stochastic, MWUA cannot address them
		
		\item
		
		Value of sex - MWUA applies equally to sexual and asexual
		populations so hard to gain insights as to role of sex
		
	\end{itemize}
\end{itemize}

\autocite{Chastain2014}:

\begin{itemize}
	\item
	
	Assumes weak selection - from paper: differences in fitness between
	genotypes are small relative to the recombination rate and so
	evolution proceeds near linkage equilibrium - probability of
	occurrence of a certain genotype is the product of the probabilities
	of each of its alleles
	
	
	\begin{itemize}
		\item
		
		{[}Weak selection where two phenotypes have similar fitness, and so
		one only slightly preferred. Only relevant (claim elsewhere -
		wikipedia) that only relevant in Moran process (fixed population as
		births-deaths paired - no birth without a death) as in growing
		population both can proliferate and weak selection results in
		effectively no selection{]}
		
		\item
		
		{[}Also assumes sexual reproduction - ``in the presence of sexual
		reproduction'', and does not address mutation{]}
		
	\end{itemize}
	\item
	
	Shows that ``equations of population genetic dynamics are
	mathematically equivalent to positing that each locus selects a
	probability distribution on alleles according to a particular
	rule...known as the multiplicative weight updates algorithm (MWUA)''
	
	
	\begin{itemize}
		\item
		
		Uses Nagylaki's theorem for allele frequencies given weak selection
		in presence of sex
		
		\item
		
		NS is ``tantamount to each locus choosing at each generation its
		allele frequencies in the population so as to maximize the sum of
		the expected cumulative differential fitness over the alleles, plus
		the distribution's entropy.''
		
	\end{itemize}
	\item
	
	Hints that MWUA enhances entropy of alleles' distribution, so helping
	to maintain genetic diversity under NS - a ``tradeoff between
	increasing entropy and increasing (cumulative) fitness.''
	
\end{itemize}

\section{General evolutionary models}

\autocite{VonNeumann1966} as reviewed in \autocite{Taylor:1999sc} (Lack of environmental emphasis)
Von Neumann's architecture for how ``complicated machines could evolve
from simple machines''

\begin{itemize}
	\item
	
	Fundamental distinction between a description of a machine and the
	machine itself.
	
		      	      Von Neumann architecture includes genotype-phenotype distinction
		      	      (machine and description)
		      	      
		      	      
		      	      \begin{itemize}
		      	      	\item
		      	      	
		      	      	Advantages of G/P distinction discussed by many, including
		      	      	Taylor:1999sc section 7.2.3
		      	      	
		      	      \end{itemize}
		      	      
	\item
	A - Constructive machine takes description and builds instance of machine
	\item	
	B - Copying machine makes a copy of a description
	
	\item
	
	C - Control machine to sequence other two machines - copy first,
	then construct, then link resulting machine to description
	
	\item
	
	Taylor states ``I would suggest that the reproducing programs in
	Tierra and similar systems can also sensibly be analysed in terms of
	von Neumann's architecture.''
	
	
	\begin{itemize}
		\item
		
		Although some of A, B, C are implicit in world rather than in
		organism
		
	\end{itemize}
\end{itemize}

\autocite{Waddington2008} as reviewed in \autocite{Taylor:1999sc}--Originally published in ``Towards a Theoretical Biology, Vol. 2'' in 1969
Waddington a process for open-ended evolution from \autocite{Taylor})

\begin{itemize}
	\item
	
	Perhaps a starting point for further work
	
	\item
	
	Also Genotype (G) and Phenotype (Q*) based, where Q* associated with
	an environment (E from Ej)
	
	\item
	
	For OEE need: Ej infinite numbered set, and sufficient Qs for Q*s
	for all those Ejs
	
	
	\begin{itemize}
		\item
		
		Q*s are part of Ej satisfies condition one {[}recursive?{]}
		
		\item
		
		Second is an emergent one - ``it is not sufficient to create new
		mutations which merely insert new parameters into existing
		programmes; they must actually be able to rewrite the programme''
		- key distinction between OEE and creative evolution
		
	\end{itemize}
\end{itemize}

		
		
		\hypertarget{hogeweg1998---on-searching-generic-properties}{\subsection{Hogeweg1998
				- ``On searching generic
				properties\ldots{}''}\label{hogeweg1998---on-searching-generic-properties}}
		
		\begin{itemize}
			\item
			
			``The simplest way of defining an evolutionary process is to define
			some set of predefined interactions between replicators and subject
			one (or a few) of the parameters of the system to mutations (selection
			automatically ensues from the dynamics of the system).''
			
			\item
			
			``\ldots{}the processes associated with the major transitions are an
			automatic consequence of mutation and selection, due to the generation
			of higher levels of selection due to spatial self-organization.''
			
			
			\begin{itemize}
				\item
				
				iff the interactions between the replicators are defined locally
				(meaning spatially) (from earlier work)
				
				\item
				
				because local interactions will form ``higher-level structures
				(e.g., spiral waves, turbulence, path like structures of different
				sizes) which constitute different levels of selection.''
				
				\item
				
				stress is on new levels of selection, rather than other elements of
				transitions
				
				\item
				
				But still ``they do not give us `novel' entities, as biotic
				evolution undoubtedly has''
				
				
				\begin{itemize}
					\item
					
					Perhaps because lack one of the elements in Maynard-Smith:1995lw -
					``transition from limited inheritance to universal inheritance''
					
					\item
					
					Claim see other three in earlier work:
					
					
					\begin{itemize}
						\item
						
						Symbiogenesis - ``properties of local interacting, evolutionary
						systems ...embody a process reminiscent of `Symbiogenesis' in
						that self-sufficiency is (partly) given up in favor of the
						larger scale entities.''
						
						\item
						
						Conflicts among levels of selection - claims interactions
						between meso-scale and micro-scale entities are inherent (from
						observation of spiral wave experiments)
						
						\item
						
						Division of labour - different elements of spatial structure
						reproduce differently, hence germ-like and soma-like...
						
					\end{itemize}
				\end{itemize}
			\end{itemize}
			\item
			
			Necessary/essential for OEE - ability to redefine interactions,
			genetic representations, and fitness of the replicators
			
		\end{itemize}


\autocite{Bourrat2015}

\begin{enumerate}
	\item
	Fitness independent of inheritance potential--as explored in
	\autocite{Bourrat2015}--bias applied only to bias value of offspring, not fitness
	\item
	fitness dependent on inheritance--more likely. A mechanism that doesn't copy well unlikely to preserve information leading to high
	fitness\ldots{}--the parent's fidelity influences the offspring's fidelity, and to offspring's fitness
\end{enumerate}


\autocite{Bourrat2015} introduces a check-for-overcrowding step--is this necessary? Under endogenous selection shouldn't overcrowding also be endogenous?


	Difference between Evolution, Natural Selection and ENS
		
	\begin{itemize}
		\item
		
		All from a biological reading; examples are regarding genes, traits,
		phenotypes and drift and \ldots{}
		
		\item
		
		And philosophical reading - epistemology regarding causality and
		ability to reason - of earlier work e.g., `` So far this chapter has
		shown that there is an alternative to the concept of fitness as a
		propensity. I have argued for a concept of fitness as relational
		properties between two or more individual entities forming a
		population. I have shown, using causal graphs that distinguishing
		between extrinsic and intrinsic-variable properties on the one hand
		and intrinsic-invariable properties of entities on the other hand
		could be the basis for distinguishing natural selection from
		drift.`` {[}p65{]}
		
		\item
		
		However, does lead to improved understanding of selection,
		reproduction, survival etc from epistemology
		
	\end{itemize}

	``The goal of the process perspective is to delimit what the minimal
	requirements for a population to exhibit nothing but natural selection
	(what I will call pure ENS) are. Thus, the goal here is to assess
	whether the evolutionary change observed in a population with
	imperfect transmission of traits across generations is compatible with
	pure ENS. One way to carry out this evaluation is to consider a
	population (albeit a highly idealised one) of entities in which all
	the evolutionary forces except natural selection have been stripped
	down: that is, one without mutation, migration and/or drift. In such a
	case, any remaining evolutionary change in this population will be ENS
	in its pure form. I call this class of populations, which is a
	subclass of minimal Darwinian populations, pure Darwinian
	populations.'' {[}p83{]}

	``With this in mind, the strategy of endogenising can be understood
	has explaining variables which have previously been taken for granted
	in a model (such as reproduction and inheritance), by reference to
	other, more fundamental variables present in the model.'' {[}p129{]}

	
	Approach to demonstrate the imperfect inheritance is not compatible
	with NS is to list three conditions for a population to evolve solely
	by NS, and then show that at least one of those conditions is
	incompatible with imperfect inheritance (as it happens, no production
	of new variants) {[}p96{]} - argument by contradiction

	
	Fundamental issue is that previous arguments proceed from biology
	-\textgreater{} explanation by division, and then arguments over what
	the divisions are. Bouratt attempts to bring clarity to the divisions
	in order to come up with a consistent explanation for the biology. Our
	goal is different - we're interested in the result not the explanation

	Necessary steps for paradigmatic ENS p136 (slightly different from
	conditions on p129 - wording, and conditions c and d swapped)
	
	
\begin{itemize}
	\item
	
	a ``New variation is introduced in the population over time''
	
	\item
	
	b ``The population is able to maintain its size or, if the size of
	the population is not limited, to increase (multiplication)''
	
	\item
	
	c ``Advantageous phenotypes are able to be transmitted during
	reproduction, making population-level changes against which new
	mutations can occur, possible.''
	
	\item
	
	d ``Reproduction is pervasive, therefore each entity in the
	population is in principle able to reproduce''
	
\end{itemize}

	
Modelling
	
\begin{itemize}
	\item
	
	Biased and unbiased \emph{inheritance}
	
	\begin{itemize}
		\item		
		Unbiased - trait is uncorrelated with parent. Implemented as
		random choice between lower and upper bound {[}p153{]}			
		\item	
		Biased - trait has correlation with parent, at some level. Some
		prediction of traits possible - parent can ``pass it on''. Not
		clear how the offspring traits are calculated though based on
		correlation {[}only description on p141 - later p173 expressly
		says this is not the `transmission bias` of the second term in the
		Price equation{]}. Looks like a narrowing of the gap between upper
		and lower bounds

	\end{itemize}
	\item
	
	Ability to \emph{procreate}	
	
	\begin{itemize}
		\item
		
		Persistors - unable to reproduce, selection only in ``weak'' sense
		of granite grains for hardness
		
		\item
		
		Procreators can reproduce but without inheritance of any property
		(including ability to procreate) except ``fact of coming into
		existence and membership of that class'' (class is class of
		parents defined by ``those properties that do not vary in the
		population''...{[}acknowledged as loose, but has benefit that no
		varying traits included{]}) p137. Procreator's offspring is
		persistors
		
	\end{itemize}
	\item
	
	Classes in models
	
	
	\begin{itemize}
		\item
		
		Minimal reproducers - indefinite procreation - where procreation
		can be transmitted from parent to offspring (with some low degree
		of fidelity)
		
		\item
		
		Unreliable reproducers (low bias for ability to procreate- ability
		to procreate randomly chosen between 0 and parent's ability),
		reliable reproducers (high bias - ability to procreate is same as
		parent's ability)
		
		\item
		
		Replicator - all traits (including procreation) can be inherited
		
	\end{itemize}
	\item
	
	All models assume property has occurred (so exogenous) and then
	investigates implications e.g., mutation or reproducer at
	onset\ldots{} {[}Models p149{]}
		
\end{itemize}

Models

\begin{itemize}
	\item
	
	Model 1
	
	
	\begin{itemize}
		\item
		
		5000 persistors, with random survival rate (viability) between
		0-0.99 (chance of surviving at each time step). Selection only.
		Essentially null hypothesis.
		
		\item
		
		All eventually die
		
		\item
		
		Viability reflects life-span; other events are proportional to
		life-span so procreators with long life-spans will produce more
		children, everything else being equal (note that standard
		meaning for fitness is viability*fertility)
		
	\end{itemize}
	\item
	
	Model 2
	
	
	\begin{itemize}
		\item
		
		4999 persistors and 1 procreator (survival rate, and fertility
		rate - offspring per unit time). Offspring traits uncorrelated
		with parent. Selection -\textgreater{} Reproduction
		-\textgreater{} Check-for-overcrowding (\emph{what is the
			ceiling on population size?})
		
		\item
		
		For procreator: Viability=0.99, fertility=10: all extinct a
		little later than in Model 1 (unsurprising)
		
	\end{itemize}
	\item
	
	Model 3
	
	
	\begin{itemize}
		\item
		
		Model 2 except single procreator replaced by single minimal
		reproducer - viability and fertility of offspring are unbiased,
		ability to procreate is unbiased also (so each offspring may or
		may not be able to procreate; ones that cannot are persistors).
		Viability selection is only form of selection in model
		
		\item
		
		Ability to procreate random between 0 and 0.20 (=probability
		offspring can procreate - looks as if same for each offspring,
		and same each generation - but unclear)
		
		\item
		
		Population size drops then increases to maximum size; about 10\%
		of minimal reproducers
		
		\item
		
		Proportion of high fitness (high viability=0.99) entities
		doesn't increase beyond about 0.05, so \emph{no cumulative
			adaptation}
		
		\item
		
		\emph{Surely also dependent on initial conditions - low rate of
			procreation will lead to extinction, high rate to proportion
			related to rate - calculable?}
		
	\end{itemize}
	\item
	
	Model 4
	
	
	\begin{itemize}
		\item
		
		Model 3 + (perfect) biased inheritance of viability (survival
		rate)
		
		\item
		
		Proportion of high fitness entities rapidly increases to near
		1.0 (near as some random decreases) - \emph{cumulative
			adaptation?}
		
		
		\begin{itemize}
			\item
			
			\emph{Mechanism is that high viability entities live longer
				and so produce more offspring/low viability die sooner and so
				produce less - fertility random, but viability heritable}
			
		\end{itemize}
		\item
		
		\emph{Why have persistors in initial population? Just die off -
			affecting proportions\ldots{}}
		
		\item
		
		\emph{Essentially just says that inheritance of viability +
			selection on viability =\textgreater{} adaptation towards high
			viability}
		
	\end{itemize}
	\item
	
	Model 5 - variable ability to procreate
	
	
	\begin{itemize}
		\item
		
		Model 3 + mutation stage
		(mutation-\textgreater{}selection-\textgreater{}reproduction-\textgreater{}check-for-overcrowding)
		
		\item
		
		At each mutation stage, 0.01 (also stated as 10\^{}-3...) rate
		for increase/decrease of ability to transmit ability to
		procreate (max 0.20 - reproducers), and degree of bias (that is,
		relationship to parent's ability) in ability to procreate (from
		unbiased to biased)
		
		
		\begin{itemize}
			\item
			
			in a sense, ability to procreate is a relationship between
			siblings (low ability of parent means low proportion of
			siblings can procreate, high ability means high proportion of
			siblings)
			
			\item
			
			while bias is the relationship between parent and offspring -
			low bias means offspring's ability is only weakly correlated
			with parents ability.
			
		\end{itemize}
		\item
		
		Viability (fitness) random 0-0.99; Ability to Procreate =
		initial random between 0 and 0.2, change +/- 0.01 equal
		probability; Bias initially 0, change +/- 0.01: max for both is
		1.0
		
		
		\begin{itemize}
			\item
			
			Both \emph{increase} over time in population -\textgreater{}
			1.0 - \emph{why? why does model not result in some infidelity
				or source of variation? }
			
			\item
			
			Asymmetry in direction of change of inheritance of ability to
			procreate - reductions lead to extinction of that line,
			increases lead to increased population\ldots{}
			
		\end{itemize}
		\item
		
		Conclusion: initial population of unreliable reproducers (low
		proportion of procreating offspring, no bias) results in
		reliable reproducers - all offspring are procreators
		
	\end{itemize}
	\item
	
	Model 6
	
	
	\begin{itemize}
		\item
		
		Population of reproducers with perfect bias in ability and
		fidelity of reproduction (end state of Model 5) + mutation on
		viability, and fidelity of transmittal of viability to offspring
		
		\item
		
		5000 individuals, viabilities random between 0 and 0.99;
		fertility between 0 and 10.
			\item
			
			Fidelity of viability \emph{increases} to 1.0

	\end{itemize}
\end{itemize}
		
Results
		
		
\begin{itemize}
	\item
	
	Minimal reproduction (not procreation) introduced through a)
	allows b), requiring only unbiased inheritance
	
	\item
	
	Unbiased inheritance of procreation and traits insufficient for c)
	
	\item
	
	Biased inheritance of trait (e.g., viability) satisfies c)
	(``Minimally-reproductive cumulative ENS''), but with unbiased
	inheritance of procreation doesn't meet d)
	
	\item
	
	Biased inheritance of trait + procreation meets c) and d)
	
\end{itemize}
		
	
Issues
	
	
\begin{itemize}
	\item
	
	\emph{From models with particular assumptions -\textgreater{} run
		(not known how many runs - other than references to ``typical run'')
		-\textgreater{} analyse output. }
	
	\item
	
	\emph{Not known how sensitive models are to parameters (e.g,
		survival rates, fertility rates\ldots{}or inheritance of viability
		rate - in Model 4 rate is 1=perfect inheritance. Other arbitrary
		features: check for overcrowding stage in Model 2 onwards,
		introduced without justification p141)}
	
	\item
	
	\emph{Why do factors in Models 4/5/6 increase to 1.0? What happens
		if start at 1.0? Would they drop? Results seem inconsistent with a
		hypothesis of useful variation\ldots{}}
	
	\item
	
	\emph{Heredity and fitness (viability) are treated as independent
		traits. But mechanism for heredity is the thing that copies the
		information that generates an offspring's traits - so not
		independent. This might explain trait values trending to 1.0 rather
		than a lesser value consistent with a hypothesis of heredity as
		source of variation}
	
\end{itemize}
	
\autocite{Watson2010}:

\begin{itemize}
	\item
	
	Adaptation in biology appears to precede Natural Selection, so
	adaptation is possible without NS
	
	\item
	
	Fundamental open question - mechanisms for adaptation
	
	\item
	
	Explored through variety of different projects, self-organized
	adaptive systems without NS - ``We present an abstract model and
	simulation of this process and discuss how it relates to a number of
	different domains: the evolution of evolvability in gene regulation
	networks {[}12{]}, the evolution of new units of selection {[}10{]}
	via symbiosis {[}15{]} and 'social niche construction' {[}8,9{]},
	games on adaptive networks {[}2{]}, distributed optimisation in
	multi-agent complex adaptive systems {[}13,14{]} and multi-scale
	optimisation algorithms {[}6,7{]}.``
	
\end{itemize}

\autocite{Saunders1994} Evolution without Natural Selection (DaisyWorld):

\begin{itemize}
	\item
	
	Lack of proof that NS was mechanism of natural evolution {[}still
	discussion e.g., Masatoshi Nei stressing mutation as driver;
	acceptance e.g, Mayr, that Darwin couldn't prove NS as mechanism for
	adaptation{]}
	
	\item
	
	Lovelock proposed Daisyworld as alternative explanation (rather than
	selection) for regulation as seen in Gaia hypothesis, and in organisms
	
	\item
	
	Two feedback loops lead to regulation, ``As a result, regulation, one
	of the most fundamental and necessary properties of organisms, appears
	without being selected for. What is more, it appears as a property not
	of the daisies, on which natural selection may have acted, but of the
	planet, on which, as Dawkins rightly points out, it could not.''
	
	\item
	
	{[}individuals modify environment{]}
	
	\item
	
	Daisies adapt planet (temperature for maximum growth) to suit
	themselves, rather than themselves to planet
	
	\item
	
	Little benefit to adaptation by daisies to planet. In fact, ``the
	ability to withstand a greater variability is not the result of
	Darwinian adaptation. On the contrary, it exists because of the
	absence of Darwinian adaptation.''
	
	\item
	
	``What is especially interesting for biology is that the problems
	arise when we try to optimize simultaneously two connected features of
	a structure. They will therefore not be revealed by a research
	strategy which seeks to account for organisms by decomposing them into
	individual traits which, so it is assumed, are acted on separately by
	natural selection.'' -- an argument for a holistic rather than reductionist view.
	
\end{itemize}

\autocite{Maley1999}
\begin{itemize}
	\item
	
	``we would like the evolutionary system, like life, to continue to
	produce individuals of increasing complexity and diversity.'' -
	although note, following McShea, that much of life is single-celled
	and hasn't become much more complex in billions of years.
	
	\item
	
	Focus on diversity (to make progress)
	
	\item
	
	Some suggestions previously that diversity is bounded (at minimum, by
	number of molecules available for biosphere, also by energy and
	minimal populations and probably other things), and plateaus
	(punctuated equilibrium). Indicates two time constants
	
	
	\begin{itemize}
		\item
		
		fast - expansion to use available resources
		
		\item
		
		slow - innovations to open new adaptive space
		
	\end{itemize}
	\item
	
	Urmodel1 - neutral landscape, mutation, early stop before all niches
	filled (to prevent edge effects)
	
	
	\begin{itemize}
		\item
		
		32-bit genotype - mutation flips one-bit
		
		\item
		
		Get unbounded diversity, but no selection or heritable effect on
		fitness
		
	\end{itemize}
	\item
	
	Posits that need selection : ``Requirement 2 An open-ended
	evolutionary system must embody selection'' - because ``fails to meet
	one of the basic criteria of natural selection: the heritable
	variation has no effect on fertility'' {[}ignoring use of fertility as
	a fitness-analogue{]}, and from Bedau ``Requirement 3 ...continuing
	(`positive') new adaptive activity'' {[}ignore neutral theory, and
	accepts Bedau - perhaps to allow use of Anew as measure?{]}
	
	\item
	
	Urmodel2 - natural selection: mutation, ``dissimilarity'' for
	competitive advantage (justified by biological example of niche
	overlap theory (Levins, 1968)) - no increase in Anew
	
	\item
	
	Urmodel3 - selective sweep (hypothesis): parasites (mutations) and
	hosts (fixed genotypes). Fitness on degree of match between parasite
	and host bit patterns.
	
	
	\begin{itemize}
		\item
		
		Claim shows unbounded activity - ``the first known artificial
		evolutionary system demonstrating unbounded evolutionary activity''
		
		\item
		
		{[}Restricted by 32-bit genomes, no death{]}
		
		\item
		
		But probably not a unique or even significant result - ``The only
		trick is to defer the point when the model hits its true asymptotic
		behaviour for long enough that the growth dynamics of the model are
		themselves asymptotic in some sense''
		
	\end{itemize}
	\item
	
	Urmodel4 - ``the most important aspect of an organism's environment
	are the other organisms with which it interacts'' - add coevolution to
	Urmodel3 by letting hosts mutate
	
	\item
	
	Two ``distasteful aspects of Urmodel3'' leading to belief that metrics
	aren't right
	
	
	\begin{itemize}
		\item
		
		niches are imposed from outside, not endogenous - this becomes
		Requirement 4
		
		\item
		
		no surprise - claim is because not complex - ``A puddle of inert,
		multicoloured and diverse algae would not be nearly so inspirational
		as the rain forest.'' Again, a biological metaphor.
		
	\end{itemize}
\end{itemize}

\section{Specific work in artificial systems}

Minimal conditions for evolutionary system capable of open-ended (but not necessarily interesting behaviour)

\begin{itemize}
	\item
	      Elements that allow for ongoing evolution--necessary, and starting
	      point for novelty
	\item
	      Open-ended evolution can be seen as evolution in an open-ended system
	      (\eg Chemistry), where an open-ended system has effectively
	      unrestricted representation: the number of possible types must be much
	      larger than the number of individuals (ideally without any
	      restriction). Without this property all possible types can be
	      generated in a finite time, and the system will either reach stasis or
	      begin to repeat. Not all open-ended systems necessarily support
	      evolution, but in those that do, our intuition suggests that
	      open-ended evolution produces increasing complexity, increasing
	      diversity, accumulation of novelty and continual adaptation
	      \autocite{Lehman2012}.
\end{itemize}

\autocite{Taylor2001}:
	Competition between individuals for resources - VanValen1973 Red Queen
	hypothesis - primary source of intrinsic selection pressure.
	Individuals and environment mutually affect each other
	\begin{itemize}

		\item	
		Resources must be ``(a) a vital commodity to individuals in the
		population; (b) of limited availability; and (c) that individuals
		can compete for (at either a global or local level). This resource
		can usually be interpreted as energy, space, matter, or a
		combination of these.''
		\item
		``the potential for a large degree of intrinsic adaptation''
		\item	
		Ray made similar arguments in favour of interactions with other individuals (rather than isolated as in EA) 		
	\end{itemize}
	
	
Hypothesis - four necessary conditions for OEE (left open if
sufficient) \autocite{Soros2014}:
			
			General prereqs: good genetic representation, ``sufficiently large
			world for every individual to be evaluated'', and a seed or starting
			point
			
\begin{itemize}
	\item
	
	A rule should be enforced that individuals must meet some minimal
	criterion (MC) before they can reproduce, and that criterion must be
	nontrivial.
	
	\item
	
	The evolution of new individuals should create novel opportunities
	for satisfying the MC
	
	\item
	
	Decisions about how and where individuals interact with the world
	should be made by the individuals themselves.
	
	\item
	
	The potential size and complexity of the individuals' phenotypes
	should be (in principle) unbounded.
	
\end{itemize}

			``openendedness depends fundamentally on the continual production of novelty.'' Standish, in \autocite{Soros2014}
					
			Minimal conditions for OEE \autocite{Vasas2015}:
			\begin{itemize}
				\item
				
				very rich combinatorial generative mechanism e.g., organic
				chemistry. Underlies the evolvability of niches (Dorin and Korb
				2011)
				
				\item
				
				unlimited heredity - number of possible heritable types should
				astronomically exceed individuals in population
				(Maynard-Smith:1995lw)
				
				\item
				
				inexhaustible fitness landscape - implies rich, dynamical
				environment
				
				\item
				
				Cannot state in advance possible preadaptations. (not clear why this
				is a condition\ldots{}seems to be saying that evolution is
				algorithmic but results are not - are emergent) Richness part of
				real chemistry, not from representations of chemistry which are
				limited - necessary requirement for OEE in material systems. (But
				real chemistry is also limited...just not as much)
				
			\end{itemize}
			
\quote{
	by open-ended evolutionary capacity we understand the potential of a
	system to reproduce its basic functional-constitutive dynamics, bringing
	about an un-limited variety of equivalent systems, of ways of expressing
	that dynamics, which are not subject to any predetermined upper bound of
	organizational complexity (even if they are, indeed, to the
	energetic-material restrictions imposed by a finite environment and by
the universal physico-chemical laws.}
{\autocite{Ruiz-Mirazo2004}}

\begin{itemize}
	\item
	      Open-ended from \autocite{MaynardSmith1999} definition--\TODO{ size of search space vs population}
	\item
	      Heritability a challenge--biological organisms employ digital
	      heredity; sophisticated mechanism with controlled error rates, but
	      exceedingly unlikely to arise spontaneously
	\item
	      Multiplication/heredity for maintenance of population
	\item
	      Analog methods possible--\eg:
	      	
	      \begin{itemize}
	      	\item
	      	      compositional (where new entity contains some elements of original)
	      	      (as seen in ACS ``core'' inheritance e.g., \autocite{Vasas2015, Watson2012}?)
	      	\item
	      	      \quote{migrant pools}{\autocite{Watson2015}}
	      	\item
	      	      Group fissioning \autocite{Watson2015}
	      	\item
	      	      Attractor based \autocite{Szathmary2000}
	      \end{itemize}
	\item
	      Heredity seen as method to maintain low entropy over much longer time
	      than possible with non-''biological'' systems \autocite{Adami2015}
	\item
	      Argument that heredity may in fact be a product of evolution rather than a precursor \autocite{Bourrat2015}
	\item
	      \autocite{Kauffman:1993kk} argued that self-organization (RAPN) can replace the genome
\end{itemize}

General difficulties with earlier work:
\begin{itemize}
	\item Somewhat arbitrary choices of elements of description
	\item Genotype/Phenotype, Selection,\ldots{} often based on goal of rationalizing existing descriptions, so not a re-examination
	\item Lack causality--so hard to use as mechanism
	\item Leave options and alternatives for implementer
	\item Sheer number of EA algorithms
\end{itemize}

\autocite{Godfrey-Smith2007}
\begin{itemize}
	\item Biological summaries
	\item Purpose of summaries as opposed to Formal models
	\item Identify the major elements in biology
\end{itemize}

Previous work coming to consensus on conditions--e.g., rich generative mechanism, unlimited heredity, inexhaustible fitness landscape, emergence \autocite{Vasas2015}, and good genetic representation, ``sufficiently large world for every individual to be evaluated'', and a seed or starting point, (plus four specific conditions \autocite{Soros2014})


``Synthesis and simulation of living systems'' or contemporary artificial life as \quote{an interdisciplinary study of life and life-like processes, whose two most important qualities are that it focuses on the essential rather than the contingent features of living systems and that it attempts to understand living systems by artificially synthesizing simple forms of them.}{\autocite{Bedau:2007ga}}

``life-as-it-could-be'' rather than ``life-as-we-know-it'' \autocite{Langton1989}

Implicit rather than explicit fitness

\begin{itemize}
	\item
	      The mapping between representation and fitness must be implicit
	\item
	      a property that arises from the representation itself rather than from
	      an external measure
	\item
	      difficult to imagine how to pre-specify a mapping that remains
	      relevant in an open-ended system
\end{itemize}

Life a subset of Alife

Life is the only \emph{provided} example that we have

Life has non-trivial emergence

Alife doesn't have to, but it is a guide

Hard vs Soft vs Wet forms of Alife

Purposes

\begin{itemize}
	\item
	      Insights into life
	\item
	      Swimming--\autocite{Terzopoulos1994}
	\item
	      Foundations for other fields e.g., AI
	\item
	      In own right--life ``de novo''
	\item
	      Philosophy--what is living?
\end{itemize}

Themes \autocite{Aguilar2014}

\begin{itemize}
	\item
	      Properties of living systems--Origins of life, autonomy, self-organization, adaptation (evolution, development, and learning)
	\item
	      Life at different scales--Ecology, artificial societies, behaviour, computational biology, artificial chemistries
	\item
	      Understanding, uses and descriptions of the living--information, living technology, art and philosophy
\end{itemize}

History

\begin{itemize}
	\item Prehistory
	\item Various automata
	\item
	      Philosophy about automata
	\item
	      1818 Mary Shelley ``Frankenstein; or, The Modern Prometheus''
	\item
	      Uptick in mentions of ``Artificial Life'' as quoted in \autocite{Aguilar2014}
	\item
	      Modern field
	\item
	      1951 Von Neumann--first formal model
	\item
	      1984 Christopher Langton
	\item
	      1987 ``Official'' birth of field--first ``Workshop on the Synthesis
	      and Simulation of Living Systems'' in Sante Fe, NM, by Langton
	\item
	      Conway Game of Life
	\item
	      Cellular Automata
	\item
	      Tierra, Avida
	\item
	      Dawkin's Biomorphs
	\item
	      Bedau Challenges
	\item
	      Overlap with Brooks's robotics, AI, etc
\end{itemize}

Some systems claim capable of OEE (e.g., Channon, Avida) but not necessarily creative

EvoEvo project taking a similar approach, but from a higher level biological starting point (genotype-phenotype mappings)
http://evoevo.liris.cnrs.fr/about-evoevo-project/

\begin{itemize}
	\item Presupposes microbial evolution, ``at the level of genomes, biological networks and populations.''
	\item Focus on four specific properties of a genotype-phenotype mapping - Variability, Robustness, Evolvability, Open-endedness
	\item Later work to remove biological specificity to provide framework for applying EvoEvo to ICT problems
\end{itemize}

Finally, in Artificial Life, Artificial Chemistries have been used in the exploration of open-ended or creative evolution. Squirm3 \parencite{Hutton2002,Hutton2009,Lucht2012} adopts fixed molecule types, and pre-defined reactions for replication and gene-sequence transcription, and so although capable of interesting behaviour is not capable of unlimited extension. Stringmol \parencite{Hickinbotham2011} - a bacterial inspired microprogram chemistry - though does demonstrate a rich heredity for open-ended evolution using string-matching to model binding between sequences, and RBN-World \parencite{Faulconbridge2011} shows that a form of Random Boolean Network, with the addition of a bonding mechanisms to allow for composition and decomposition of RBNs, can be used to build a chemistry capable of almost limitless extension out of non-traditional components.

Open-ended evolution can be seen as evolution in an open-ended system (\eg Chemistry), where an open-ended system has effectively unrestricted representation: the number of possible types must be much larger than the number of individuals (ideally without any restriction). Without this property all possible types can be generated in a finite time, and the system will either reach stasis or begin to repeat. Not all open-ended systems necessarily support evolution, but in those that do, our intuition suggests that open-ended evolution produces increasing complexity, increasing diversity, accumulation of novelty and continual adaptation \cite{Lehman2012}.

\quote{by open-ended evolutionary capacity we understand the potential of a system to reproduce its basic functional-constitutive dynamics, bringing about an un-limited variety of equivalent systems, of ways of expressing that dynamics, which are not subject to any predetermined upper bound of organizational complexity (even if they are, indeed, to the energetic-material restrictions imposed by a finite environment and by the universal physico-chemical laws)}{\cite{Ruiz-Mirazo2004}}

\begin{itemize}
	\item An open-ended evolutionary system must demonstrate unbounded diversity during its growth phase.
	\item An open-ended evolutionary system must embody selection.
	\item An open-ended evolutionary system must exhibit continuing (``positive'') new adaptive activity.
	\item An open-ended evolutionary system must have an endogenous implementation of niches.
\end{itemize} \cite{Maley:1999bs} (considered ``rather abstract'' by Hutton \parencite[p.341]{Hutton2002}).

\Textcite{Taylor2001,Taylor:1999sc} discuss creativity in \gls{oee} in depth and argues that, for it to be possible, the replicators must \parencite{Hutton2004}:\begin{enumerate}[label=\roman*] \item Be fully embedded in their arena of competition \item Have rich, unlimited interactions between each other and with their environment \item Initially replicate implicitly, rather than using some encoding of the replication process, and \item Be constructed entirely of `material' components, allowing the possibility of different encodings of information. (\quote{the very stuff from which they are constructed is a valuable resource of matter and energy}{\cite[s3.6]{Taylor2001}})
\end{enumerate}

Bottom-up models for open-ended evolution leverage richness of underlying environment - less information in entity definition, more in environment definition. Similar to biology, where physics and chemistry underpin living organisms, where definition of minimal cell many orders of magnitude simpler than the working out of the chemical and physical rules that it relies upon.

Top-down models assume a knowledge of the necessary elements.

Limited heredity replicators vs unlimited - the first where the number of possible types is less than the number of individuals; the second where it far exceeds\cite{Szathmary:2006ty}

\begin{table}
	\scriptsize
	\caption{A sample of Artificial Chemistries for open-ended evolution}
	\label{tab1}
	\begin{tabular}{@{}p{4cm}p{4.5cm}p{4.5cm}@{}}
		\hline\noalign{\smallskip}
		Chemistry                                                          & Energy Model?                                                      & Constructive?                            \\ 
		\\ \noalign{\smallskip}
		\hline
		\noalign{\smallskip}
		\cite{Ducharme2012}                                                & Yes                                                                & Yes                                      \\
		StringMol \parencite{Hickinbotham2012}                             & No, global energy only, conservation of mass as proposed extension & Yes                                      \\
		Squirm3 \parencite{Hutton2002,Lucht2012}                           & No                                                                 & No                                       \\
		RBN-World \parencite{Faulconbridge2011}                            & Unknown                                                            & Yes                                      \\
		\cite{Lenaerts2009}                                                & No                                                                 & Yes - molecular interactions             \\
		ZChem \parencite{Tominaga2009}                                     & Conservation of mass                                               & No - reactions are atomic with wildcards \\
		Substrate-Catalyst-Link (SCL) \parencite{Varela:1974qd,Suzuki2008} & No                                                                 & Unknown                                  \\
		\cite{Fernando:2008xy,Fernando:2007pf}                             & Yes, and thermodynamics govern reactions                           & No - atomic reactions                    \\
		\cite{Gardiner2007}                                                & No                                                                 & No - atomic reactions                    \\
		NAC \parencite{Suzuki2006}                                         & Unknown                                                            & Yes                                      \\						
		GGL/ToyChem \parencite{Benko2005}                                  & Mass conservation only                                             & Yes                                      \\
		Lattice Artificial Chemistry \parencite{Ono2000,Madina2003}        & No                                                                 & No                                       \\
		GGL/ToyChem \parencite{Benko2003}                                  & Mass conservation only                                             & No - pre-defined reactions only          \\
		\hline
	\end{tabular}
\end{table}



Geb \cite{Channon:iw,Channon:2001ly} is claimed to be open-ended: artificial organisms controlled by neural networks created by a developmental process from a bit-string genotype. Individuals interact with the world through five predefined types of interaction generated by the neural network - reproduce (crossover and mutation of production rules), fight, turn anti-clockwise or clockwise, move forward.

Ducharme et al \parencite{Ducharme2012}. The approach taken is to model the energy changes associated with reactions. The chemistry is spatial; atoms are arranged on a 2-dimensional grid and have velocity. When two atoms pass within a particular distance, they interact. The possible types of interactions are prespecified, with the type chosen being driven by the atomic composition and energies of the interacting atoms. Reactions are therefore between atoms rather than molecules; a molecule in this chemistry is a combination of atoms arranged in a particular structure, re-examined after each reaction to form a stable configuration based on expectations from real-world chemistry. Although computational costs are not reported, it seems plausible that the calculation of intersections on a 2-dimensional grid will be expensive for large molecular populations. Another cost comes from the re-arrangement of molecules into energy-efficient configurations. This spatial structuring enables the model to restrict atomic interactions to those atoms that are accessible on a molecule, but at the cost of additional modelling complexity. 


\subsection{Tierra and Avida}

\autocite{Ofria2004}

Avida
\begin{itemize}
	\item
	
	``An approach to studying evolution...''
	
	\item
	
	``According to Daniel Dennett, ``...evolution will occur whenever and
	wherever three conditions are met: replication, variation (mutation),
	and differential fitness (competition)''''
	
	\item
	
	``(However, as Barton and Zuidema {[}3{]} note, general acceptance
	will ultimately hinge on whether artificial life researchers embrace
	or ignore the large body of population genetics literature.)''
	
	\item
	
	Difference with GAs - natural organisms must replicate themselves to
	pass on genetic information - ``final arbiter of fitness'', and
	interaction with other organisms and with environment
	
	\item
	
	Steen Rasmussen inspired by computer game core war - competing
	segments of simplified assembly code in core memory. With change to
	copy command to introduce mutations and hence evolutionary potential,
	core world created. But system ``collapsed into a non-living state''
	{[}non-living?{]} One possible reason - copying over existing
	organisms
	
	\item
	
	Tierra next year (relationship not stated) organisms had to allocate
	memory first before using. Initial selective pressure only from rate
	of replication. Sequential execution of organism code
	
	\item
	
	Avida summer of 1993 - better metering and measuring, and parallel
	code execution
	
	
	\begin{itemize}
		\item
		
		``In principle, the only assumption made about these
		self-replicating automata in the core Avida software is that their
		initial state can be described by a string of symbols (their genome)
		and that they autonomously produce offspring organisms. However, in
		practice our work has focused on automata with a simple von Neumann
		architecture that operate on an assembly-like language inspired by
		the Tierra system.''
		
		\item
		
		Instruction, read, write, and flow control heads for relative rather
		than absolute addressing - bit like a Turing tape machine
		
		\item
		
		Many instructions grouped into instruction sets. Default set has 26
		instructions
		
		\item
		
		Every program is valid
		
	\end{itemize}
	\item
	
	Phenotypes - ``The primary mode of environmental interaction is by
	inputting numbers from the environment, performing computations on
	those numbers, and outputting the results. The organisms receive a
	benefit for performing specific computations associated with
	resources''
	
	\item
	
	All very configurable, and complicated, but why? What rationale behind
	choices? More of a testbed for experiments, e.g., `` in one experiment
	we wanted to study a population that could not adapt, but that would
	nevertheless accumulate deleterious or neutral mutations through
	drift''
	
	\item
	
	``The quest to halt adaptation is only one example of a special
	feature in Avida; many more have been explored, and are continuously
	being added to the source code. The most successful features are all
	fully described in the documentation that comes with the software.''
	
\end{itemize}

\autocite{Lenski2003}

\begin{itemize}
	\item
	
	Modelled in Avida
	
	\item
	
	``Our experiments demonstrate the validity of the hypothesis, first
	articulated by Darwin and supported today by comparative and
	experimental evidence, that complex features generally evolve by
	modifying existing structures and functions.''
	
	\item
	
	``Some simpler functions were accessible from the ancestor by
	relatively few mutations, and these served as a foundation on which
	more complex features were built. The foundational role of simpler
	functions in the origin of more complex ones was evident in the
	overlap of the genetic networks underlying their expression, and the
	frequent loss of simpler functions as side-effects of mutations
	yielding more complex function''
	
\end{itemize}

\autocite{Taylor2001}:

\begin{itemize}
	\item 	
	Derived from Thesis (Taylor:1999sc)
	\item	
	Criticism of Tierra
	\begin{itemize}
		\item
		
		Not built around any particular theory - ``This weakness is not
		specific to Tierra, but is shared by most, if not all, of the other
		Tierra-like systems which have emerged over the last
		decade\ldots{}''
		
		\item
		
		In Ray's words, ``...this approach involves engineering over the
		early history of life to design complex evolvable organisms, and
		then attempting to create conditions that will set off a spontaneous
		evolutionary process of increasing diversity and complexity of
		organisms''
		
		
		\begin{itemize}
			\item
			
			Problem with `engineering over' is we don't understand the natural
			examples well enough to engineer them
			
			\item
			
			Similar criticism by Pattee1988 - ``simulations that are dependent
			on ad hoc and special-purpose rules and constraints for their
			mimicry cannot be used to support theories of life''
			
		\end{itemize}
	\end{itemize}
	\item
	
	General points
		
	\begin{itemize}
		\item
		
		
		\item
		
		Phenotype fundamentally ``involves interaction with the environment
		(and that this is the essential distinction between the notions of
		phenotype and genotype - the latter being an informational
		concept)''
		
		\item
		
		Seed (proto-DNA) must itself be an indefinite heredity replicator
		{[}assumes that this is minimal starting point, rather than that
		this itself may evolve{]}
		
		\item
		
		Assume that early stages see A+B implicitly encoded in the
		environment {[}essentially because simpler than explicit mechanism,
		but little justification{]} ``At the early stages of an evolutionary
		process, however, we would not expect there to be mechanisms for
		explicitly decoding the proto-DNA\ldots{}''
		
	\end{itemize}
	\item
	
	Nils Barricelli: ````It may appear that the properties one would have
	to assign to a population of self-reproducing elements in order to
	obtain Darwinian evolution are of a spectacular simplicity. The
	elements would only have to: (1) Be self-reproducing and (2) Undergo
	hereditary changes (mutations) in order to permit evolution by a
	process based on the survival of the fittest'' (Barricelli, 1962,
	pp.70--71)
	
	\item
	
	Materiality - better than formal system as get some features `for
	free' {[}more generally, minimal constraint on interactions as only
	one class of elements rather than separate non-interacting ones{]}
	
\end{itemize}
		
\subsection{Rewriting or String-based}
		
\autocite{Antonakopoulos:2011th}
		
		\begin{itemize}
			\item
			
			One developmental mapping for several structures or species e.g., CAs,
			and boolean networks - sparsely connected networks
			
			\item
			
			Using L-systems - one for cells/nodes, and the other for connectivity
			rules
			
			\item
			
			Developmental process ``capable of expressing developmental actions
			e.g., growth and differentiation'' and ``able to express a large
			variety of topologies within each architecture\ldots{}''
			
		\end{itemize}
		
\autocite{Dittrich1998}:

		\begin{itemize}
			\item
			
			``study the phenomena of life, not by simulating life as it is (weak
			AL) but by instantiating life as it could be (strong AL)'' {[}Langton,
			C. G. (1989). Artificial life{]}
			
			\item
			
			``An important property of most strong AL systems is that they contain
			the ability for self-reference. For instance, Ray's Tierra organisms
			are able to read, copy, and modify their own code {[}27{]}. In
			Fontana's algorithmic chemistry every object is a character string
			able to process other objects by using the lambda-calculus that maps
			the character string into an (active) function {[}13{]}. The dualism
			inherent in those systems can be traced back to Godel {[}15{]} who
			defined a mapping of mathematical statements into natural numbers ``
			that allowed self-reference, to Turing's universal machine {[}31{]},
			and to von Neumann's stored program computer {[}7{]}.''
			
			\item
			
			``Here, we will concentrate on dynamic phenomena, especially on the
			emergence of prebiotic evolution''
			
			\item
			
			``The term self-evolution should refer to an evolutionary process
			within a population system where the components responsible for the
			evolutionary behavior are (only) the individuals of the population
			system itself. Every variation is carried out by the individuals.
			Selection pressure is generated implicitly through interaction among
			the individuals and not by external agents. The system must not
			contain a component that can be identified as a fitness function or
			global operators performing selection or variation (e.g.,
			crossover).''
			
			\item
			
			Introduces S,R,A classification scheme for artificial chemistry -
			here, S=`` binary strings with a constant length of 32 bits'',
			R=s1+s2-\textgreater{}s3, and A=''simulates a well-stirred tank
			reactor with mass-action kinetics, which assures that the probability
			of a collision is proportional to the product of the concentration of
			the colliding objects'' from earlier work (Fontana, Kauffman etc)
			
			\item
			
			A=``1. Select two objects s1,s2 from the soup randomly, without
			removing them. 2. If there exists a reaction s1 + s2 to s3 and the
			filter condition f (s1,s2,s3) holds, replace a randomly selected
			object of the soup by s3.'', s1 and s2 are not consumed, rather they
			act as catalysts. Chosen as this shown capable of hypercyclic
			organisation
			
			\item
			
			AND reaction and an automata reaction to generate s3 from s1, s2.
			Automata is a deterministic FSA, running s1 (4-bit fixed instructions)
			on s2.
			
			\item
			
			Measures (all macroscopic)
			
			
			\begin{itemize}
				\item
				
				Diversity
				
				\item
				
				Distance Distribution Complexity (from Kim)
				
				\item
				
				Productivity
				
				\item
				
				Innovativity (innovation = new string)
				
			\end{itemize}
			\item
			
			Experiments - initialised always with random (32-bit) strings
			
			\item
			
			Compared to ODE approach in Bagley1992 (AlifeV version)
			
		\end{itemize}
		
		\hypertarget{fontana1992}{\subsection{Fontana1992}\label{fontana1992}}
		
		\begin{itemize}
			\item
			
			AlChemy
			
		\end{itemize}
		
\autocite{Fenizio2000}:
		
		\begin{itemize}
			\item
			
			Original AlChemy reactions of form A+B-\textgreater{}C where C
			replaces an existing element (X)
			
			\item
			
			This system generates A+B-\textgreater{}C1+C2...CN where C is a
			multiset of size N. Done by modifying the original K rule to detach x2
			and eliminate both original elements (like reactants in chemistry)
			
			\item
			
			Uses combinators rather than lambdas
			
			\item
			
			To prevent from stopping (out of elements) added modification where
			randomly add/remove some elements
			
			\item
			
			Combinator first combines (appends) elements, each element other than
			first bracketed. Then each 1-term combinator applied to string, where
			it makes specific changes e.g., K x1x2s0-\textgreater{}x1s0 (s0 is
			remaining substring, may be null). Apply until no further reductions
			possible (that is, in normal form). Two combinators are equivalent if
			can be reduced to same combinator (and previously noted that order is
			not important - same results regardless of order).
			
			\item
			
			Free pool of atoms for conservation of ``mass''
			
		\end{itemize}
		
\autocite{Fenizio2001}:
		
		\begin{itemize}
			
			\item
			
			Focus on ``identity as an entity separated from its environment'',
			that is, membrane formation. Graph used to model spatial structures
			
			\item
			
			``an artificial chemistry (AC) is embedded in a graph, with each
			molecule being a vertex of the graph and possible interactions being
			allowed only along the edges of the graph''
			
			\item
			
			``Molecules are built from a substrate of elements called atoms. There
			are seven types of atoms (I, K, W, R, B, C, S), each with a different
			function. The total number of atoms in the reactor is kept constant
			during a run. Free atoms (not bounded in molecules) are separately
			stored and form a global pool.''
			
			\item
			
			Provides definition of autopoeisis in terms of graphs
			
			\item
			
			Experiment to show spontaneous formation of autopoietic cells
			
			\item
			
			Rules for combinations of two molecules are predetermined - reaction
			mechanisms are described in Fenizio2000
			
		\end{itemize}
		

\subsection{Boolean Networks and Graphs}

		
\autocite{Nellis2012}
		\begin{itemize}
			
			\item
			
			``Our aim is to improve novelty-generation algorithms by making their
			biological models richer''
			
			\item
			
			Novelty-generation as goal/theme for meta-evolution
			
			\item
			
			No measure for novelty - not even sure is possible. But informal
			definition that says more novelty as result of more embodiment (this
			seems circular?) p87
			
			\item
			
			Embodiment as mechanism. Provides overview in 3.2
			
			\item
			
			AChems categorized as world/chemistry/constraints (p132), with
			constraints being energy model/binding model. RBN-World mentioned as a
			world where elements are RBN networks
			
			\item
			
			States that binding model needs to be rich - empirical evidence
			presented (StringMol) only
			
			\item
			
			Copying is a phenomenon, expressed through mechanisms at different
			levels of the AChem
			
			\item
			
			Explored through embodied copying mechanism built in GraphMol, with
			antecedent in StringMol (Hickingbotham2011)
			
			\item
			
			StringMol includes an embodied copying mechanism -
			start/at-end/char-copy/next. Some elements (at-end) use Smith-Waterman
			matching algorithm, which opens ability for evolution to modify the
			function of at-end (Nellis2012p143). But others, e.g., next, are not
			embodied. GraphMol makes next embodied
			
			\item
			
			Design is graph based (obviously..). ``The world defined by GraphMol
			contains chemicals (represented as graphs) that bind to each other via
			multiple binding sites, and then run simple computer programs (encoded
			in the graphs) that modify the binding of these chemicals.''. Why? No
			explicit rationale presented. Presumably StringMol starting point
			meant programs, copying, then graphs?
			
			\item
			
			Operates (like StringMol) at level of proteins/enzymes e.g.,
			replicases, DNA
			
			\item
			
			Runtimes in weeks to months
			
		\end{itemize}
		
		
\autocite{Nellis2014}
		
		\begin{itemize}
			\item
			
			Summary of PhD thesis - computational novelty through embodiment
			
			\item
			
			Comparing StringMol and GraphMol

			\item
			
			Emergent systems easy to code but produce surprising results. Cannot
			predict results from rules (or in fact easily predict rules from
			desired results. One-way function, akin to encryption)
			
			\item
			
			Standard GAs search through a fixed space - cannot surprise, limited
			range
			
			\item
			
			Need something emergent
			
			\item
			
			Mechanism of evolution must be itself evolvable; individuals and
			environment interact to give new ways of producing new individuals.
			Dynamics required for novelty-generation.
			
			\item
			
			Quick defined embodiment in terms of two dynamical systems mutually
			affecting each other - no need for a physical world. System modifies
			environment. Doesn't address if system is constructed from
			environment. Autopoiesis does say though that system built from
			environment. But autopoiesis talks about maintenance not evolution
			
			\item
			
			Functions such as template copying must be embodied mechanisms in the
			world - so can be affected and evolved
			
			\item
			
			Stringmol and Graphmol have embodied template copying, done in
			different ways. Different computational models result in different
			properties - ``Stringmol exhibits macro-mutation and two chemical
			copying; GraphMol exhibits two types of quasispecies, cooperative and
			parasitic. These two systems use the same domain (emergent evolution)
			and metamodel (machines copying strings), but different computational
			models.''
			
		\end{itemize}
		
		\autocite{Faulconbridge2010, Faulconbridge2011}:
		
		RBN-World \cite{Faulconbridge2011} chemistry where the entities are described as a form of Random Boolean Network, with the addition of a bonding mechanisms to allow for composition and decomposition of RBNs. A number of parameters affect the behaviour of the chemistry, and so a series of experiments sampled from the parameter-space, and then used a GA, to search for interesting variants as measured by non-catalysed ``loops'' (ideal measures of auto-catalytic sets and Hypercycles too rare for use as a measure) (\cite[§8]{Faulconbridge2011}). 
		
		\begin{itemize}

			\item
			
			RBNWorld	
			
			\begin{itemize}
				\item
				
				Entities are described as a form of Random Boolean Network, with the
				addition of a bonding mechanisms to allow for composition and
				decomposition of RBNs. A number of parameters affect the behaviour
				of the chemistry, and so a series of experiments sampled from the
				parameter-space, and then used a GA, to search for interesting
				variants as measured by non-catalysed loops (ideal measures of
				auto-catalytic sets and Hypercycles too rare for use as a measure)
				(chap 8)
				
				\item
				
				Gillespie-like" - random reaction, random time - not correlated in
				any way with reaction energies or rates (Faulconbridge2011§8.4.3.1)
				
				\item
				
				bRBNs used as atomic elements, properties determined by the network
				
				\item
				
				Larger structures are formed by ``bonding'' two independent bRBNs at
				each bRBNs bonding node. ``All reactions are between two reactants;
				it is assumed that more complicated reactions can be expressed as a
				series of two-reactant reactions with intermediate structures.``
				Record kept of composition so that decomposition can be easily done.
				
				\item
				
				Many design choices - bonding mechanism etc. Examination of
				alternatives done by searching with EA
				
				\item
				
				Each bRBN is a RBN, made up of a number of nodes, each with an
				initial state (true/false) assigned randomly and with a input/output
				matrix assigned randomly. Finally k(=2) inputs are established per
				node. Synchronous state update. All based on Kauffman1969
				(interestingly, although noted as ``original'' so later work known)
				
				\item
				
				bonding method uses ''cycle length as the bonding property and
				equality as the bonding criterion....bonds only exist between bRBNs
				that have the same cycle length.'' in initial examples at least n=5
				and b(k?)=2, and alternatives examined using EA. After initial bond
				formation recalculate cycle lengths, and check again for equality -
				might result in decomposition.
				
			\end{itemize}
			\item
			
			Three types of chemistries
			
			
			\begin{itemize}
				\item
				
				Symbolic - symbols/molecules have no inherent meaning, so no
				``implicit reactions''
				
				\item
				
				Structured - one or more atoms arranged in a structure (string,
				tree, graph..) with bonds within molecules - unlimited, but
				consequently computationally expensive
				
				\item
				
				Sub-symbolic - emergence of properties from lower-levels (ala real
				chemistry, also neural networks). Symbols (atoms) have internal
				structure which gives properties.
				
			\end{itemize}
			\item
			
			Describes two generic strategies for the selection of reactants,
			spatial and aspatial, where the primary difference is whether
			molecular position is a factor in reactant selection
			
			\item
			
			Measures are Synthesis, Self-Synthesis, Decomposition, Substitution,
			and Catalysis (Faulconbridge2011§7), and non-catalysed `loops' (ideal
			measures of auto-catalytic sets and Hypercycles too rare for use as a
			measure) (Faulconbridge2011§8)
			
			\item
			
			Random sampling + EA (Fitness function (§8.4.3.2) is based on
			non-catalysed loops (§4.3.9.4))
			
			\item
			
			"identifies three types of \textbackslash{}emph\{mixing method\} or
			reactor algorithm
			
			\item
			
			Contains an interesting discussion mapping these desirable properties
			onto the emergent properties that are then required of an
			\textbackslash{}gls\{achem\}
			
			\item
			
			Some advantages claimed over Hutton (emergence and computational
			intractability p190)
			
			\item
			
			``As the choice to use RBNs as the sub- symbolic representation in
			RBN-World was based on limited information. As a discrete dynamical
			system that is computationally tractable yet also spans a wide range
			of behaviours, RBNs met the appropriate criteria. It is not expected
			that RBNs are the best representation however; others may be more
			suitable for particular emergent properties.``
			
		\end{itemize}
		
\subsection{Atoms/Molecules}
		
		\hypertarget{lucht2012---size-selection-and-adaptive-evolution-in-an-artificial-chemistry}{\subsection{Lucht2012
				- ``Size Selection and Adaptive Evolution in an Artificial
				Chemistry``}\label{lucht2012---size-selection-and-adaptive-evolution-in-an-artificial-chemistry}}
		
		\begin{itemize}
			\item
			
			http://www.cis.utas.edu.au/users/mwlucht/BTL.html.
			
			\item
			
			Challenge to community - "to develop a system in which, starting with
			a soup of free atoms and a simple ``bootstrap'' chemistry, a cell-like
			creature similar to the one in H-41 evolves."
			
			\item
			
			Based on Hutton2007 Squirm3 chemistry, using Hutton's floods
			
			\item
			
			Squirm3 - fixed molecule types, and pre-defined reactions for
			replication and gene-sequence transcription, and so although capable
			of interesting behaviour is not capable of unlimited extension
			
			\item
			
			Added reaction types to address Squirm3's ``global-extinction problem
			and showing how quasi-universal enzymes can evolve
			
		\end{itemize}
		
		\hypertarget{hutton2007---evolvable-self-reproducing-cells-in-a-two-dimensional-artificial-chemistry}{\subsection{Hutton2007
				- ``Evolvable Self-Reproducing Cells in a Two-Dimensional Artificial
				Chemistry ``
			}\label{hutton2007---evolvable-self-reproducing-cells-in-a-two-dimensional-artificial-chemistry}}
		
		\begin{itemize}
			\item
			
			http://www.sq3.org.uk
			
			\item
			
			Enzymes can act as catalysts so G affects P; but enzyme evolution too
			improbable for OEE
			
			\item
			
			2D grid of squares (lattice), spring force for membrane
			
			\item
			
			Enzymes can now affect all reactions except enzyme production; in
			practice too slow
			
			\item
			
			Production rules hardcoded into AChem
			
			\item
			
			Only with raw materials in environment, niche construction through
			adaptation to availability of raw materials
			
			\item
			
			AChem hand-built with reactions instead of enzymes
			
			\item
			
			Prespecified reactions - see Faulconbridge2011 p.49
			
			\item
			
			Hutton2007 and Hutton2002
			
		\end{itemize}
		
		\begin{itemize}
			\item
			
			Artificial system capable of life-like \textbackslash{}gls\{oee\}
			(creativity)
			
			\item
			
			Based on hypothesis (materiality, interactions, embedding) of
			\textbackslash{}textcite\{Taylor2001\}, better expressed in
			\textbackslash{}textcite{[}p.341{]}\{Hutton2002\}; membrane to allow
			individuals to benefit from innovations by protecting internal
			reactions from others
			
			\item
			
			\textbackslash{}Gls\{achem\} from
			\textbackslash{}textcite\{Hutton2002\} used to construct all elements
			in world - material and embedded.
			
			\item
			
			Atoms in \textbackslash{}gls\{achem\} of different types and states;
			reaction rules; otherwise no energy modelling; only physics modelling
			is the concept of location (either integer-coordinates or real
			number-coordinates) and impossibility of colocation. Floods are used
			to recycle raw materials.
			
			\item
			
			Individuals (each bounded by a membrane) with the capability for
			division and mutation. Raw materials (atoms) are only required for
			division
			
			\item
			
			Interactions between individuals are limited to effects on shared
			environment (niche construction without direct interaction), so one
			element of hypothesis untested; restrictions on open-endedness:
			evolution of new enzymes unobserved as extremely unlikely, and
			genotype-phenotype mapping hard-coded (unevolvable)
			
		\end{itemize}
		
	
\subsection{Compositional}
		
		\autocite{Vasas2015, Vasas2012, Vasas2012a}
		
		\begin{itemize}
			
			\item
			If inheritance is statistical only then ok; but literature
			stresses digitally encoded although might just be cognitive bias
			
			\item
			
			Unit of selection very important to define now in abiogenesis
			
			\item
			
			Can evolution happen when information transfer is non-digital?
			Specifically where there is a parent-offspring correlation in
			molecular composition?
			
			\item
			
			Based on GARD Segre et al 1998
			
			
			\begin{itemize}
				\item
				
				Compositional inheritance
				
				\item
				
				claim made that GARD is capable of darwinian evolution, but
				population analysis showed not in response to directional selection
				(Vasas2010)
				
				\item
				
				Eigen threshold applies - mutation rates (see issue of replication -
				big differences between parent and child) overwhelm selection
				
			\end{itemize}
			\item
			
			Ganti and Eigen showed that distinct, organizationally different
			alternative autocatalytic networks in same environment might compete
			and fittest would prevail (e.g., autocatalytic networks as units of
			selection)
			
			
			\begin{itemize}
				\item
				
				Check number of network components (=autocatalytic networks). In
				GARD, end up with just one big component
				
			\end{itemize}
			\item
			
			GARD does not result in selectable replicating entities - there is no
			replication as certain highly catalytic molecules determine the
			properties of the compotype, and these are not inherited equally -
			instead a child may or may not inherit one of these molecules and so
			its properties may be similar to or very different from its parent
			
			\item
			
			Kauffman 1986 Reflexively Autocatalytic Polymer Networks (RAPN) however are
			evolvable. First, likelihood of such networks higher than expected
			(Hordijk and Steel) and second, in Vasas2012 putting these networks
			into compartments (so not well-stirred) then can do directional
			selection
			
			\item
			
			Differences between GARD and RAPN:
				
			\begin{itemize}
				\item
				
				kinetics of growth - RAPN has ligation, GARD does not. Ligation
				allows new components to be formed with new properties
				
				\item
				
				search for adaptations - GARD has fixed catalysis, Kauffman does not
				- mutant polymers can arise, be incorporated into the set, and
				influence its fitness
				
			\end{itemize}
			\item
			
		\end{itemize}
