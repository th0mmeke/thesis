\chapter{Previous work}

The previous sections have established the context, and identified the primary areas of relevant literature. From biology, the main threads concern the origins of life, and the theory of evolution. Natural selection provides the primary mechanism for evolution in biology, although it is not the only process at work -- genetic drift and neutral theory provide a counterpoint. In artificial systems, researchers have used biological evolution as an inspiration, gradually over time diverging into a field with the beginnings of its own motivating theory, now only loosely connected with its origins in the natural world.

The research emphasis in artificial systems has more recently returned to a reappreciation of how biological evolution generates robust, novel, creative outcomes, unlike those seen in current artificial evolution. This has led to a renewed interest in understanding the principles behind biological evolution so that artificial systems can capture some of those admirable properties; the difference now is that the transference is sought at the level of concepts and principles rather than in the historical inspiration of specific biological elements and structures, many of which are contingent and perhaps even arbitrary; certainly complex.

This section now expands upon previous work that is directly relevant to our specific problem, the onset of evolution in artificial systems, in four areas:

\begin{itemize}
	\item Principles behind biological evolution
	\item Generalisations from biological and artificial evolution
	\item General evolutionary models
	\item Specific work in artificial systems
\end{itemize}

\section{Principles behind biological evolution}

\autocite{Watson2012}:

\begin{itemize}
	\item
	
	``We take it as given that biology instantiates ENS'' - but that
	doesn't mean that the algorithm of biology is ENS. Biological
	evolution may be superset of ENS, and ENS may not be sufficient for
	biological evolution. Hampered by circular reasoning that ENS
	described in terms of biological evolution
	
	\item
	
	{[}fundamentally addresses issues of algorithmic equivalence between
	biological evolution and ENS, without concerning the problem of ENS
	and evolution in general{]}
	
\end{itemize}


\hypertarget{watson2015---evolutionary-connectionism-algorithmic-principles-underlying-the-evolution-of-biological-organisation-in-dvo-devo-evo-eco-and-evolutionary-transitions}{\subsection{Watson2015
		- ``Evolutionary connectionism: algorithmic principles underlying the
		evolution of biological organisation in dvo-devo, evo-eco and
		evolutionary
		transitions''}\label{watson2015---evolutionary-connectionism-algorithmic-principles-underlying-the-evolution-of-biological-organisation-in-dvo-devo-evo-eco-and-evolutionary-transitions}}

\begin{itemize}
	\item
	
	Darwinian machine is fundamentally self-referential - products of
	evolution affect process of evolution - lots of examples
	
	
	\begin{itemize}
		\item
		
		Major transitions in evolution not possible without
		self-referentiality - unit of evolution must change between levels
		e.g., from molecules to cells. This was/is accomplished by changing
		the way that individuals interact, from competition to cooperation
		(fitness change) to form the next level. How does selection at one
		level get suppressed, and introduced at the next?
		
		\item
		
		Maynard-Smith/Szathmary
		
		\item
		
		Independent replication before transition, must replicate as part of
		a larger whole afterwards
		
		\item
		
		Fundamental for complexity - complexity associated with levels
		
		\item
		
		Draws analogy with Hinton 2008 Deep Learning
		
		\item
		
		At one level, supervised learning improving adaptation; at next
		higher, unsupervised improving robustness
		
		\item
		
		Modes of inheritance (for groups)
		
		
		\begin{itemize}
			\item
			
			``migrant pools'' - particles disperse horizontally and reform
			into new pools - type 1 group selection - no group inheritance
			
			\item
			
			Group fissioning - vertical inheritance - group inheritance
			(differences are heritable) - type 2
			
		\end{itemize}
	\end{itemize}
\end{itemize}




\section{Generalisations from biological and artificial evolution}
\hypertarget{griesemer2005---the-informational-gene-and-the-substantial-body-on-the-generalization-of-evolutionary-theory-by-abstraction}{\subsection{Griesemer2005
		- ``The Informational Gene and the Substantial Body: On the
		Generalization of Evolutionary Theory by
		Abstraction''}\label{griesemer2005---the-informational-gene-and-the-substantial-body-on-the-generalization-of-evolutionary-theory-by-abstraction}}

\begin{itemize}
	\item
	
	``Darwin's theory of evolution by natural selection is restricted in
	scope. One sense in which it is restricted is that it refers to
	organisms.''
	
	
	\begin{itemize}
		\item
		
		``a theory of descent with modification''
		
		\item
		
		natural selection as important mechanism of modification
		
		\item
		
		``Owing to this struggle for life, any variation, however slight and
		from whatever cause proceeding, if it be in any degree profitable to
		an individual of any species, in its infinitely complex relations to
		other organic beings and to external nature, will tend to the
		preservation of that individual, and will generally be inherited by
		its offspring. The offspring, also, will thus have a better chance
		of surviving, for, of the many individuals of any species which are
		periodically born, but a small number can survive. I have called
		this principle, by which each slight variation, if useful, is
		preserved, by the term of Natural Selection, in order to mark its
		relation to man's power of selection (Darwin 1859, p. 61).''
		
		\item
		
		Restricted by:
		
		
		\begin{itemize}
			\item
			
			natural selection, not drift or Lamarckian (biased) inheritance
			
			\item
			
			organisms not defined - so scope vague. Many treat vertebrates as
			paradigm, despite rarity
			
			\item
			
			organismal level - but many other levels
			
		\end{itemize}
		\item
		
		``Darwin's theory is about a complex causal process, not merely a
		description of effects or patterns''
		
	\end{itemize}
	\item
	
	Weismannism - central dogma! Causal relationships between genes and
	soma (body)
	
	
	\begin{itemize}
		\item
		
		EB Wilson (1896) version (restricted version of Weismann) - genes
		inherited, soma transitory, genes causally responsible for soma, but
		no causal relationship back from soma to gene
		
		\item
		
		Crick's central dogma - molecular interpretation DNA/RNA/protein -
		isomorphic to Wilson
		
		\item
		
		Maynard Smith compared Wilson and Crick explicitly showing
		isomorphism
		
		\item
		
		Weismann own view showed gene-plasma (``molecular protoplasm of the
		nucleus'') also continuous despite gene-cells not being so. Heredity
		is a function of development - somatic differentiation causes
		continuity of germ-plasm and so soma affects soma - causal link
		added to Wilson (Weismannism) from soma to soma
		
	\end{itemize}
	\item
	
	Strategies for abstraction
	
	
	\begin{itemize}
		\item
		
		``An important ``Platonic'' conception is that abstraction is a
		mental process by which properties are thought of as entities
		distinct from the concrete objects in which they are instantiated.
		On an alternative, Aristotelian conception, abstraction is the
		mental process of subtracting certain accidental properties from
		concrete objects so as to regard objects in a manner closer to their
		essential natures.''
		
		\item
		
		Cartwright and Mendell use Aristotle's four kinds of cause as
		ordering criteria for relative abstractness - material, efficient,
		formal, final
		
		\item
		
		``Levels abstraction''
		
		
		\begin{itemize}
			\item
			
			Lewontin argued that can abstract form of individual to more
			general entity (at any level) - ``any entities in nature that have
			variation, reproduction and heritability may evolve'' Lewontin1970
			as necessary and sufficient conditions (despite ``may''?)
			
			\item
			
			assumes an a priori ``biological hierarchy'', and evolution
			operates on any level of that hierarchy given certain properties
			for units at that level
			
			\item
			
			Problem is that no criteria for recognizing elements (variation,
			reproduction, and heritability) at any level other than
			organismal. Complicated further by clonal, asexual etc...How to
			recognize individual - e.g., ``individuation criteria''
			
		\end{itemize}
		\item
		
		``the second strategy abstracts from kinds of explanatory properties
		of significant entities at the canonical hierarchical levels of
		neo-Darwinian explanation'' - functional roles
		
		
		\begin{itemize}
			\item
			
			Replicators (e.g., genes) and vehicles (e.g., organisms)
			
			\item
			
			abstracts roles of organisms and genes to two level hierarchy of
			``interactors'' and ``replicators'' (Hull)
			
			\item
			
			``The problem is how to square the information analysis at the
			molecular level with the belief that evolution may occur at a
			variety of levels: the more seriously one takes information, the
			more likely one is to deny that evolution happens at any level
			other than that of the gene.'' Leads to idea that genes are
			information, and that must be ``genes'' at each level of
			hierarchy.
			
			\item
			
			Key problem is that other information other than gene equivalent
			(e.g., development) may be needed at different levels (when don't
			have genes or equivalents) and yet
			
			\item
			
			In an Estimation Distribution Algorithm (EDA), the algorithm tries
			to determine the distribution of the solution features, e.g.
			probability of having a 1-bit at a particular position, at the
			optimum. Some EDAs can be regarded abstractions of evolutionary
			processes: instead of generating new solutions through variation
			and then selecting from these, EDAs use a more direct approach to
			refine the underlying probability distribution. The perspective of
			updating a probability distribution is similar to the
			Wright--Fisher model.
			
			\item
			
			doesn't allow soma-soma (although Weismann does, but in practice
			developmental mechanisms preclude it - single zygote stage makes
			it hard for soma changes to be inherited)
			
		\end{itemize}
	\end{itemize}
	\item
	
	``One must clearly distinguish between heredity (a relation),
	heritability (a capacity), and inheritance (a process)''
	
\end{itemize}

\autocite{Paixao2015}:

\begin{itemize}
	\item
	
	attempt to unify evolutionary computation and population genetics -
	based on common conception of evolutionary process
	
	\item
	
	Related work in Population Genetics
	
	
	\begin{itemize}
		\item
		
		PG evolution described by dynamics of allele or genotype frequencies
		
		\item
		
		Lewontin 1964 equation for canonical evolutionary process - genotype
		frequencies, based on common ``genetic operator'' for rate of
		generating a genotype x from parents y and z (combines mutation,
		recombination etc)
		
		\item
		
		This and other models biologically focussed - where ``selection
		assumes a particular form''
		
		\item
		
		Another approach to PG quantitative genetics - phenotypic view of
		trait evolution. Useful in animal breeding. Doesn't address
		mechanism (genetics)
		
		\item
		
		Price equation - very general, useful for that reason in comparing
		models
		
	\end{itemize}
	\item
	
	Any evolutionary process describes ``population undergoing changes
	over time based on some set of transformations''
	
	\item
	
	A transformation can be decomposed into a collection of (stochastic)
	operators. Operator can be described as a probability distribution of
	potential outcomes; and evolutionary process as a trajectory through a
	space of distributions.
	
	\item
	
	Process described both as a sequence of population transformations,
	and as distribution transformations
	
	\item
	
	Individuals described with both genotype and phenotype; gp mapping
	between them. Some operators act on genotype others on phenotype
	
	\item
	
	Various operators defined - for selection (uniform, proportional,
	tournament, truncation, cut, replace); variation (mutation) (uniform,
	single-point), variation (recombination) (one-point crossover, k-point
	crossover, uniform crossover, unbiased variation)
	
	\item
	
	Most models satisfy five given mathematical properties
	
	
	\begin{itemize}
		\item
		
		V1 - uniformity preserving - no change to distribution if uniformly
		distributed
		
		\item
		
		M1 - mutation acts on individuals
		
		\item
		
		M2 - mutation can generate whole search space ie an ergodic operator
		(defining characteristic of mutations)
		
		\item
		
		R1 - recombination preserves allele frequencies (in expectation)
		
		\item
		
		S1 - selection doesn't change individuals
		
	\end{itemize}
	\item
	
	Demonstrates how existing ``classical models in theoretical population
	genetics and in the theory of evolutionary computation'' can be mapped
	into this framework. Most of PG models can be represented
	(unsurprising as most variants of classical models that have been
	demonstrated in framework); some topic-specific EC models could not be
	- but not ones that have a relationship with PG and so of little
	interest to biologists. GP models omitted for reasons of balance
	between simplicity and inclusiveness
	
	\item
	
	Leaves MOEAs for later work
	
	\item
	
	``In an Estimation Distribution Algorithm (EDA), the algorithm tries
	to determine the distribution of the solution features, e.g.
	probability of having a 1-bit at a particular position, at the
	optimum. Some EDAs can be regarded abstractions of evolutionary
	processes: instead of generating new solutions through variation and
	then selecting from these, EDAs use a more direct approach to refine
	the underlying probability distribution. The perspective of updating a
	probability distribution is similar to the Wright--Fisher model.''
	
	
	\begin{itemize}
		\item
		
		Close similarity between simplest EDA (the Univariate Marginal
		Distribution Algorithm) and Linkage equilibrium models in population
		genetics as pointed out in Chastain2014 (``We demonstrate that in
		the regime of weak selection, the standard equations of population
		genetics describing natural selection in the presence of sex become
		identical to those of a repeated game between genes played according
		to multiplicative weight updates (MWUA), an algorithm known in
		computer science to be surprisingly powerful and versatile. MWUA
		maximizes a tradeoff between cumulative performance and entropy,
		which suggests a new view on the maintenance of diversity in
		evolution.'')
		
		\item
		
		AdaBoost is a form of MWUA
		(\href{https://www.cs.princeton.edu/~arora/pubs/MWsurvey.pdf}{\emph{https://www.cs.princeton.edu/\textasciitilde{}arora/pubs/MWsurvey.pdf}})
		
	\end{itemize}
\end{itemize}

\autocite{Barton2014}

\begin{itemize}
	\item
	
	Experts in MWUA equivalent to alleles in NS
	
	\item
	
	In selection, ``frequency of each type {[}allele{]} is simply
	multiplied by its relative fitness; which corresponds precisely to
	MWUA''
	
	\item
	
	MWUA ``maximizes the sum of two quantities: the expected total fitness
	of the chosen allele, summed over past generations, plus the entropy,
	which is a measure of the allelic diversity.''
	
	\item
	
	Fitness differences increase over time, and so population converges
	towards type with best performance; entropy term acts to slow down
	this convergence
	
	\item
	
	Problems
	
	
	\begin{itemize}
		\item
		
		because MWUA is deterministic, and some interesting problems e.g.,
		recombination as a mechanism for reintroducing gene combinations
		lost through drift in finite populations, are essentially
		stochastic, MWUA cannot address them
		
		\item
		
		Value of sex - MWUA applies equally to sexual and asexual
		populations so hard to gain insights as to role of sex
		
	\end{itemize}
\end{itemize}

\autocite{Chastain2014}:

\begin{itemize}
	\item
	
	Assumes weak selection - from paper: differences in fitness between
	genotypes are small relative to the recombination rate and so
	evolution proceeds near linkage equilibrium - probability of
	occurrence of a certain genotype is the product of the probabilities
	of each of its alleles
	
	
	\begin{itemize}
		\item
		
		{[}Weak selection where two phenotypes have similar fitness, and so
		one only slightly preferred. Only relevant (claim elsewhere -
		wikipedia) that only relevant in Moran process (fixed population as
		births-deaths paired - no birth without a death) as in growing
		population both can proliferate and weak selection results in
		effectively no selection{]}
		
		\item
		
		{[}Also assumes sexual reproduction - ``in the presence of sexual
		reproduction'', and does not address mutation{]}
		
	\end{itemize}
	\item
	
	Shows that ``equations of population genetic dynamics are
	mathematically equivalent to positing that each locus selects a
	probability distribution on alleles according to a particular
	rule...known as the multiplicative weight updates algorithm (MWUA)''
	
	
	\begin{itemize}
		\item
		
		Uses Nagylaki's theorem for allele frequencies given weak selection
		in presence of sex
		
		\item
		
		NS is ``tantamount to each locus choosing at each generation its
		allele frequencies in the population so as to maximize the sum of
		the expected cumulative differential fitness over the alleles, plus
		the distribution's entropy.''
		
	\end{itemize}
	\item
	
	Hints that MWUA enhances entropy of alleles' distribution, so helping
	to maintain genetic diversity under NS - a ``tradeoff between
	increasing entropy and increasing (cumulative) fitness.''
	
\end{itemize}

\section{General evolutionary models}

\autocite{Watson2010}:

\begin{itemize}
	\item
	
	Extended abstract
	
	\item
	
	Southampton and Sussex (inc Bullock and Noble)
	
	\item
	
	Three non-related
	\href{https://scholar.google.com/scholar?cites=10707786264093825040\&as_sdt=2005\&sciodt=0,5\&hl=en}{\emph{citations}},
	including
	
	
	\begin{itemize}
		\item
		
		\href{http://onlinelibrary.wiley.com/doi/10.1111/bij.12124/abstract?userIsAuthenticated=false\&deniedAccessCustomisedMessage=}{\emph{Adaptive
				evolution without natural selection}}
		
		\item
		
		Davies, Adam (2014) On the interaction of function, constraint and
		complexity in evolutionary systems. University of Southampton,
		Physical Sciences and Engineering,
		\href{http://eprints.soton.ac.uk/374145/}{\emph{Doctoral Thesis}} ,
		204pp.
		
	\end{itemize}
	\item
	
	Adaptation in biology appears to precede Natural Selection, so
	adaptation is possible without NS
	
	\item
	
	Fundamental open question - mechanisms for adaptation
	
	\item
	
	Explored through variety of different projects, self-organized
	adaptive systems without NS - ``We present an abstract model and
	simulation of this process and discuss how it relates to a number of
	different domains: the evolution of evolvability in gene regulation
	networks {[}12{]}, the evolution of new units of selection {[}10{]}
	via symbiosis {[}15{]} and 'social niche construction' {[}8,9{]},
	games on adaptive networks {[}2{]}, distributed optimisation in
	multi-agent complex adaptive systems {[}13,14{]} and multi-scale
	optimisation algorithms {[}6,7{]}.``
	
\end{itemize}


\autocite{Saunders1994}

\begin{itemize}
	\item
	
	Lack of proof that NS was mechanism of natural evolution {[}still
	discussion e.g., Masatoshi Nei stressing mutation as driver;
	acceptance e.g, Mayr, that Darwin couldn't prove NS as mechanism for
	adaptation{]}
	
	\item
	
	Lovelock proposed Daisyworld as alternative explanation (rather than
	selection) for regulation as seen in Gaia hypothesis, and in organisms
	
	\item
	
	Two feedback loops lead to regulation, ``As a result, regulation, one
	of the most fundamental and necessary properties of organisms, appears
	without being selected for. What is more, it appears as a property not
	of the daisies, on which natural selection may have acted, but of the
	planet, on which, as Dawkins rightly points out, it could not.''
	
	\item
	
	{[}individuals modify environment{]}
	
	\item
	
	Daisies adapt planet (temperature for maximum growth) to suit
	themselves, rather than themselves to planet
	
	\item
	
	Little benefit to adaptation by daisies to planet. In fact, ``the
	ability to withstand a greater variability is not the result of
	Darwinian adaptation. On the contrary, it exists because of the
	absence of Darwinian adaptation.''
	
	\item
	
	``What is especially interesting for biology is that the problems
	arise when we try to optimize simultaneously two connected features of
	a structure. They will therefore not be revealed by a research
	strategy which seeks to account for organisms by decomposing them into
	individual traits which, so it is assumed, are acted on separately by
	natural selection.'' -- an argument for a holistic rather than reductionist view.
	
\end{itemize}

\section{Specific work in artificial systems}

Minimal conditions for evolutionary system capable of open-ended (but not necessarily interesting behaviour)

\begin{itemize}
	\item
	      Elements that allow for ongoing evolution--necessary, and starting
	      point for novelty
	\item
	      Open-ended evolution can be seen as evolution in an open-ended system
	      (\eg Chemistry), where an open-ended system has effectively
	      unrestricted representation: the number of possible types must be much
	      larger than the number of individuals (ideally without any
	      restriction). Without this property all possible types can be
	      generated in a finite time, and the system will either reach stasis or
	      begin to repeat. Not all open-ended systems necessarily support
	      evolution, but in those that do, our intuition suggests that
	      open-ended evolution produces increasing complexity, increasing
	      diversity, accumulation of novelty and continual adaptation
	      \autocite{Lehman2012}.
\end{itemize}

\quote{
	by open-ended evolutionary capacity we understand the potential of a
	system to reproduce its basic functional-constitutive dynamics, bringing
	about an un-limited variety of equivalent systems, of ways of expressing
	that dynamics, which are not subject to any predetermined upper bound of
	organizational complexity (even if they are, indeed, to the
	energetic-material restrictions imposed by a finite environment and by
the universal physico-chemical laws.}
{\autocite{Ruiz-Mirazo2004}}

\begin{itemize}
	\item
	      Open-ended from \autocite{MaynardSmith1999} definition--\TODO{ size of search space vs population}
	\item
	      Heritability a challenge--biological organisms employ digital
	      heredity; sophisticated mechanism with controlled error rates, but
	      exceedingly unlikely to arise spontaneously
	\item
	      Multiplication/heredity for maintenance of population
	\item
	      Analog methods possible--\eg:
	      	
	      \begin{itemize}
	      	\item
	      	      compositional (where new entity contains some elements of original)
	      	      (as seen in ACS ``core'' inheritance e.g., \autocite{Vasas2015, Watson2012}?)
	      	\item
	      	      \quote{migrant pools}{\autocite{Watson2015}}
	      	\item
	      	      Group fissioning \autocite{Watson2015}
	      	\item
	      	      Attractor based \autocite{Szathmary2000}
	      \end{itemize}
	\item
	      Heredity seen as method to maintain low entropy over much longer time
	      than possible with non-''biological'' systems \autocite{Adami2015}
	\item
	      Argument that heredity may in fact be a product of evolution rather than a precursor \autocite{Bourrat2015}
	\item
	      \autocite{Kauffman:1993kk} argued that self-organization (RAPN) can replace the genome
\end{itemize}

\autocite{Bourrat2015} showed that inheritance bias (correlation in trait between parent and offspring) increases over time

\begin{itemize}
	\item
	      Model 4--proportion of high fitness entities rapidly increases to
	      near 1.0
	\item
	      Model 5--both the proportion of offspring that are procreators
	      (rather than persistors) and the heritability of ability to procreate
	      increases over time to 1.0
\end{itemize}


\begin{enumerate}
	\item
	      Fitness independent of inheritance potential--as explored in
	      \autocite{Bourrat2015}--bias applied only to bias value of offspring, not fitness
	\item
	      fitness dependent on inheritance--more likely. A mechanism that doesn't copy well unlikely to preserve information leading to high
	      fitness\ldots{}--the parent's fidelity influences the offspring's fidelity, and to offspring's fitness
\end{enumerate}

\autocite{Bourrat2015} introduces a check-for-overcrowding step--is this necessary? Under endogenous selection shouldn't overcrowding also be endogenous?

General difficulties with earlier work:
\begin{itemize}
	\item Somewhat arbitrary choices of elements of description
	\item Genotype/Phenotype, Selection,\ldots{} often based on goal of rationalizing existing descriptions, so not a re-examination
	\item Lack causality--so hard to use as mechanism
	\item Leave options and alternatives for implementer
	\item Sheer number of EA algorithms
\end{itemize}

Proofs of effectiveness:
\begin{itemize}
	\item Base equations from Malthus and population genetics
	\item \autocite{Vose:1999di} for EAs
	\item Underlying assumptions should be maintained in our models
\end{itemize}

\autocite{Godfrey-Smith2007}
\begin{itemize}
	\item Biological summaries
	\item Purpose of summaries as opposed to Formal models
	\item Identify the major elements in biology
\end{itemize}

Previous work coming to consensus on conditions--e.g., rich generative mechanism, unlimited heredity, inexhaustible fitness landscape, emergence \autocite{Vasas2015}, and good genetic representation, ``sufficiently large world for every individual to be evaluated'', and a seed or starting point, (plus four specific conditions \autocite{Soros2014})







\autocite{VonNeumann1966} as reviewed in \autocite{Taylor:1999sc} (Lack of environmental emphasis)
\autocite{Waddington2008} as reviewed in \autocite{Taylor:1999sc}--Originally published in ``Towards a Theoretical Biology, Vol. 2'' in 1969


\autocite{Taylor2001}:

\begin{itemize}
	\item
	      	
	      46 citations - Hutton, Huneman, Lucht, Taylor, plus others in
	      Evolutionary Art, Robotics
	      	
	\item
	      	
	      Derived from Thesis (Taylor:1999sc)
	      	
	\item
	      	
	      Suggestions for further work or approaches without reaching a
	      conclusion or specific proposal
	      	
	\item
	      	
	      Classification of systems on embeddedness, degree of interactions
	      	
	\item
	      	
	      Competition between individuals for resources - VanValen1973 Red Queen
	      hypothesis - primary source of intrinsic selection pressure.
	      Individuals and environment mutually affect each other
	      
	      \begin{itemize}
	      	\item
	      	      		
	      	      Stated without justification
	      	      		
	      	\item
	      	      		
	      	      Resources must be ``(a) a vital commodity to individuals in the
	      	      population; (b) of limited availability; and (c) that individuals
	      	      can compete for (at either a global or local level). This resource
	      	      can usually be interpreted as energy, space, matter, or a
	      	      combination of these.''
	      	      		
	      	\item
	      	      		
	      	      Given as point of distinction between EAs and this work: ``The
	      	      difference is that we require a system with the potential for a
	      	      large degree of intrinsic adaptation for open-ended evolution,
	      	      rather than a system where the selection of individuals is
	      	      determined by an externally-defined fitness function''
	      	      		
	      	      		
	      	      \begin{itemize}
	      	      	\item
	      	      	      			
	      	      	      Similar arguments in favour of interactions with other individuals
	      	      	      (rather than isolated as in EA) by Ray
	      	      	      			
	      	      \end{itemize}
	      \end{itemize}
	\item
	      	
	      Distinction between OEE and ``kinds of evolutionary innovation'' -
	      e.g., in Ray1977 innovations quite limited - e.g., parasitism emerged
	      because of system design and initial seeding conditions.
	      	
	      	
	      \begin{itemize}
	      	\item
	      	      		
	      	      `fundamentally new' (labelled `creative') means new ways of sensing
	      	      environment and interacting with it
	      	      		
	      	\item
	      	      		
	      	      ``From the point of view of the evolvability of individuals, the
	      	      more embedded they are, and the less restricted the interactions
	      	      are, then the more potential there is for the very structure of the
	      	      individual to be modified. Recall that this is one aspect of my
	      	      definition of creative evolution. Sections of the individual which
	      	      are not embedded in the arena of competition are `hard-wired' and
	      	      likely to remain unchanged unless specific mechanisms are included
	      	      to allow them to change (and the very fact that specific mechanisms
	      	      are required suggests that they would still only be able to change
	      	      in certain restricted ways).``
	      	      		
	      	\item
	      	      		
	      	      Similar to Pattee's semantic closure - ``organisms should be
	      	      constructed `with the parts and laws of an artificial physical
	      	      world'''
	      	      		
	      \end{itemize}
	\item
	      	
	      Von Neumann's architecture for how ``complicated machines could evolve
	      from simple machines''
	      	
	      	
	      \begin{itemize}
	      	\item
	      	      		
	      	      Fundamental distinction between a description of a machine and the
	      	      machine itself.
	      	      		
	      	\item
	      	      		
	      	      A - Constructive machine takes description and builds instance of
	      	      machine
	      	      		
	      	\item
	      	      		
	      	      B - Copying machine makes a copy of a description
	      	      		
	      	\item
	      	      		
	      	      C - Control machine to sequence other two machines - copy first,
	      	      then construct, then link resulting machine to description
	      	      		
	      	\item
	      	      		
	      	      Taylor states ``I would suggest that the reproducing programs in
	      	      Tierra and similar systems can also sensibly be analysed in terms of
	      	      von Neumann's architecture.''
	      	      		
	      	      		
	      	      \begin{itemize}
	      	      	\item
	      	      	      			
	      	      	      Although some of A, B, C are implicit in world rather than in
	      	      	      organism
	      	      	      			
	      	      \end{itemize}
	      \end{itemize}
	\item
	      	
	      Waddington a process for open-ended evolution
	      	
	      	
	      \begin{itemize}
	      	\item
	      	      		
	      	      Perhaps a starting point for further work
	      	      		
	      	\item
	      	      		
	      	      Also Genotype (G) and Phenotype (Q*) based, where Q* associated with
	      	      an environment (E from Ej)
	      	      		
	      	\item
	      	      		
	      	      For OEE need: Ej infinite numbered set, and sufficient Qs for Q*s
	      	      for all those Ejs
	      	      		
	      	      		
	      	      \begin{itemize}
	      	      	\item
	      	      	      			
	      	      	      Q*s are part of Ej satisfies condition one {[}recursive?{]}
	      	      	      			
	      	      	\item
	      	      	      			
	      	      	      Second is an emergent one - ``it is not sufficient to create new
	      	      	      mutations which merely insert new parameters into existing
	      	      	      programmes; they must actually be able to rewrite the programme''
	      	      	      - key distinction between OEE and creative evolution
	      	      	      			
	      	      \end{itemize}
	      \end{itemize}
	\item
	      	
	      Criticism of Tierra
	      	
	      	
	      \begin{itemize}
	      	\item
	      	      		
	      	      Not built around any particular theory - ``This weakness is not
	      	      specific to Tierra, but is shared by most, if not all, of the other
	      	      Tierra-like systems which have emerged over the last
	      	      decade\ldots{}''
	      	      		
	      	\item
	      	      		
	      	      In Ray's words, ``...this approach involves engineering over the
	      	      early history of life to design complex evolvable organisms, and
	      	      then attempting to create conditions that will set off a spontaneous
	      	      evolutionary process of increasing diversity and complexity of
	      	      organisms''
	      	      		
	      	      		
	      	      \begin{itemize}
	      	      	\item
	      	      	      			
	      	      	      Problem with `engineering over' is we don't understand the natural
	      	      	      examples well enough to engineer them
	      	      	      			
	      	      	\item
	      	      	      			
	      	      	      Similar criticism by Pattee1988 - ``simulations that are dependent
	      	      	      on ad hoc and special-purpose rules and constraints for their
	      	      	      mimicry cannot be used to support theories of life''
	      	      	      			
	      	      \end{itemize}
	      \end{itemize}
	\item
	      	
	      General points
	      	
	      	
	      \begin{itemize}
	      	\item
	      	      		
	      	      Von Neumann architecture includes genotype-phenotype distinction
	      	      (machine and description)
	      	      		
	      	      		
	      	      \begin{itemize}
	      	      	\item
	      	      	      			
	      	      	      Advantages of G/P distinction discussed by many, including
	      	      	      Taylor:1999sc section 7.2.3
	      	      	      			
	      	      \end{itemize}
	      	\item
	      	      		
	      	      Phenotype fundamentally ``involves interaction with the environment
	      	      (and that this is the essential distinction between the notions of
	      	      phenotype and genotype - the latter being an informational
	      	      concept)''
	      	      		
	      	\item
	      	      		
	      	      Seed (proto-DNA) must itself be an indefinite heredity replicator
	      	      {[}assumes that this is minimal starting point, rather than that
	      	      this itself may evolve{]}
	      	      		
	      	\item
	      	      		
	      	      Assume that early stages see A+B implicitly encoded in the
	      	      environment {[}essentially because simpler than explicit mechanism,
	      	      but little justification{]} ``At the early stages of an evolutionary
	      	      process, however, we would not expect there to be mechanisms for
	      	      explicitly decoding the proto-DNA\ldots{}''
	      	      		
	      \end{itemize}
	\item
	      	
	      Nils Barricelli: ````It may appear that the properties one would have
	      to assign to a population of self-reproducing elements in order to
	      obtain Darwinian evolution are of a spectacular simplicity. The
	      elements would only have to: (1) Be self-reproducing and (2) Undergo
	      hereditary changes (mutations) in order to permit evolution by a
	      process based on the survival of the fittest'' (Barricelli, 1962,
	      pp.70--71)
	      	
	\item
	      	
	      Materiality - better than formal system as get some features `for
	      free' {[}more generally, minimal constraint on interactions as only
	      one class of elements rather than separate non-interacting ones{]}
	      	
\end{itemize}


``Synthesis and simulation of living systems'' or contemporary artificial life as \quote{an interdisciplinary study of life and life-like processes, whose two most important qualities are that it focuses on the essential rather than the contingent features of living systems and that it attempts to understand living systems by artificially synthesizing simple forms of them.}{\autocite{Bedau:2007ga}}

``life-as-it-could-be'' rather than ``life-as-we-know-it'' \autocite{Langton1989}

Implicit rather than explicit fitness

\begin{itemize}
	\item
	      The mapping between representation and fitness must be implicit
	\item
	      a property that arises from the representation itself rather than from
	      an external measure
	\item
	      difficult to imagine how to pre-specify a mapping that remains
	      relevant in an open-ended system
\end{itemize}

Life a subset of Alife

Life is the only \emph{provided} example that we have

Life has non-trivial emergence

Alife doesn't have to, but it is a guide

Hard vs Soft vs Wet forms of Alife

Purposes

\begin{itemize}
	\item
	      Insights into life
	\item
	      Swimming--\autocite{Terzopoulos1994}
	\item
	      Foundations for other fields e.g., AI
	\item
	      In own right--life ``de novo''
	\item
	      Philosophy--what is living?
\end{itemize}

Themes \autocite{Aguilar2014}

\begin{itemize}
	\item
	      Properties of living systems--Origins of life, autonomy, self-organization, adaptation (evolution, development, and learning)
	\item
	      Life at different scales--Ecology, artificial societies, behaviour, computational biology, artificial chemistries
	\item
	      Understanding, uses and descriptions of the living--information, living technology, art and philosophy
\end{itemize}

History

\begin{itemize}
	\item Prehistory
	\item Various automata
	\item
	      Philosophy about automata
	\item
	      1818 Mary Shelley ``Frankenstein; or, The Modern Prometheus''
	\item
	      Uptick in mentions of ``Artificial Life'' as quoted in \autocite{Aguilar2014}
	\item
	      Modern field
	\item
	      1951 Von Neumann--first formal model
	\item
	      1984 Christopher Langton
	\item
	      1987 ``Official'' birth of field--first ``Workshop on the Synthesis
	      and Simulation of Living Systems'' in Sante Fe, NM, by Langton
	\item
	      Conway Game of Life
	\item
	      Cellular Automata
	\item
	      Tierra, Avida
	\item
	      Dawkin's Biomorphs
	\item
	      Bedau Challenges
	\item
	      Overlap with Brooks's robotics, AI, etc
\end{itemize}

Some systems claim capable of OEE (e.g., Channon, Avida) but not necessarily creative

EvoEvo project taking a similar approach, but from a higher level biological starting point (genotype-phenotype mappings)
http://evoevo.liris.cnrs.fr/about-evoevo-project/

\begin{itemize}
	\item Presupposes microbial evolution, ``at the level of genomes, biological networks and populations.''
	\item Focus on four specific properties of a genotype-phenotype mapping - Variability, Robustness, Evolvability, Open-endedness
	\item Later work to remove biological specificity to provide framework for applying EvoEvo to ICT problems
\end{itemize}

Finally, in Artificial Life, Artificial Chemistries have been used in the exploration of open-ended or creative evolution. Squirm3 \parencite{Hutton2002,Hutton2009,Lucht2012} adopts fixed molecule types, and pre-defined reactions for replication and gene-sequence transcription, and so although capable of interesting behaviour is not capable of unlimited extension. Stringmol \parencite{Hickinbotham2011} - a bacterial inspired microprogram chemistry - though does demonstrate a rich heredity for open-ended evolution using string-matching to model binding between sequences, and RBN-World \parencite{Faulconbridge2011} shows that a form of Random Boolean Network, with the addition of a bonding mechanisms to allow for composition and decomposition of RBNs, can be used to build a chemistry capable of almost limitless extension out of non-traditional components.

Open-ended evolution can be seen as evolution in an open-ended system (\eg Chemistry), where an open-ended system has effectively unrestricted representation: the number of possible types must be much larger than the number of individuals (ideally without any restriction). Without this property all possible types can be generated in a finite time, and the system will either reach stasis or begin to repeat. Not all open-ended systems necessarily support evolution, but in those that do, our intuition suggests that open-ended evolution produces increasing complexity, increasing diversity, accumulation of novelty and continual adaptation \cite{Lehman2012}.

\quote{by open-ended evolutionary capacity we understand the potential of a system to reproduce its basic functional-constitutive dynamics, bringing about an un-limited variety of equivalent systems, of ways of expressing that dynamics, which are not subject to any predetermined upper bound of organizational complexity (even if they are, indeed, to the energetic-material restrictions imposed by a finite environment and by the universal physico-chemical laws)}{\cite{Ruiz-Mirazo2004}}

\begin{itemize}
	\item An open-ended evolutionary system must demonstrate unbounded diversity during its growth phase.
	\item An open-ended evolutionary system must embody selection.
	\item An open-ended evolutionary system must exhibit continuing (``positive'') new adaptive activity.
	\item An open-ended evolutionary system must have an endogenous implementation of niches.
\end{itemize} \cite{Maley:1999bs} (considered ``rather abstract'' by Hutton \parencite[p.341]{Hutton2002}).

\Textcite{Taylor2001,Taylor:1999sc} discuss creativity in \gls{oee} in depth and argues that, for it to be possible, the replicators must \parencite{Hutton2004}:\begin{enumerate}[label=\roman*] \item Be fully embedded in their arena of competition \item Have rich, unlimited interactions between each other and with their environment \item Initially replicate implicitly, rather than using some encoding of the replication process, and \item Be constructed entirely of `material' components, allowing the possibility of different encodings of information. (\quote{the very stuff from which they are constructed is a valuable resource of matter and energy}{\cite[s3.6]{Taylor2001}})
\end{enumerate}

Bottom-up models for open-ended evolution leverage richness of underlying environment - less information in entity definition, more in environment definition. Similar to biology, where physics and chemistry underpin living organisms, where definition of minimal cell many orders of magnitude simpler than the working out of the chemical and physical rules that it relies upon.

Top-down models assume a knowledge of the necessary elements.

Limited heredity replicators vs unlimited - the first where the number of possible types is less than the number of individuals; the second where it far exceeds\cite{Szathmary:2006ty}

\begin{table}
	\scriptsize
	\caption{A sample of Artificial Chemistries for open-ended evolution}
	\label{tab1}
	\begin{tabular}{@{}p{4cm}p{4.5cm}p{4.5cm}@{}}
		\hline\noalign{\smallskip}
		Chemistry                                                          & Energy Model?                                                      & Constructive?                            \\ 
		\\ \noalign{\smallskip}
		\hline
		\noalign{\smallskip}
		\cite{Ducharme2012}                                                & Yes                                                                & Yes                                      \\
		StringMol \parencite{Hickinbotham2012}                             & No, global energy only, conservation of mass as proposed extension & Yes                                      \\
		Squirm3 \parencite{Hutton2002,Lucht2012}                           & No                                                                 & No                                       \\
		RBN-World \parencite{Faulconbridge2011}                            & Unknown                                                            & Yes                                      \\
		\cite{Lenaerts2009}                                                & No                                                                 & Yes - molecular interactions             \\
		ZChem \parencite{Tominaga2009}                                     & Conservation of mass                                               & No - reactions are atomic with wildcards \\
		Substrate-Catalyst-Link (SCL) \parencite{Varela:1974qd,Suzuki2008} & No                                                                 & Unknown                                  \\
		\cite{Fernando:2008xy,Fernando:2007pf}                             & Yes, and thermodynamics govern reactions                           & No - atomic reactions                    \\
		\cite{Gardiner2007}                                                & No                                                                 & No - atomic reactions                    \\
		NAC \parencite{Suzuki2006}                                         & Unknown                                                            & Yes                                      \\						
		GGL/ToyChem \parencite{Benko2005}                                  & Mass conservation only                                             & Yes                                      \\
		Lattice Artificial Chemistry \parencite{Ono2000,Madina2003}        & No                                                                 & No                                       \\
		GGL/ToyChem \parencite{Benko2003}                                  & Mass conservation only                                             & No - pre-defined reactions only          \\
		\hline
	\end{tabular}
\end{table}



%Geb \cite{Channon:iw,Channon:2001ly} artificial organisms controlled by neural networks created by a developmental process from a bit-string genotype. Individuals interact with the world through five predefined types of interaction generated by the neural network - reproduce (crossover and mutation of production rules), fight, turn anti-clockwise or clockwise, move forward.

%RBN-World \cite{Faulconbridge2011} chemistry where the entities are described as a form of Random Boolean Network, with the addition of a bonding mechanisms to allow for composition and decomposition of RBNs. A number of parameters affect the behaviour of the chemistry, and so a series of experiments sampled from the parameter-space, and then used a GA, to search for interesting variants  as measured by non-catalysed €˜loops€™ (ideal measures of auto-catalytic sets and Hypercycles too rare for use as a measure) (\cite[§8]{Faulconbridge2011}). 

Ducharme et al \parencite{Ducharme2012}. The approach taken is to model the energy changes associated with reactions. The chemistry is spatial; atoms are arranged on a 2-dimensional grid and have velocity. When two atoms pass within a particular distance, they interact. The possible types of interactions are prespecified, with the type chosen being driven by the atomic composition and energies of the interacting atoms. Reactions are therefore between atoms rather than molecules; a molecule in this chemistry is a combination of atoms arranged in a particular structure, re-examined after each reaction to form a stable configuration based on expectations from real-world chemistry. Although computational costs are not reported, it seems plausible that the calculation of intersections on a 2-dimensional grid will be expensive for large molecular populations. Another cost comes from the re-arrangement of molecules into energy-efficient configurations. This spatial structuring enables the model to restrict atomic interactions to those atoms that are accessible on a molecule, but at the cost of additional modelling complexity. 

%\cite{Tominaga2009,Tominaga2007,Tominaga2004} - string-based pattern matching applied to examination of the feasibility of biochemical pathways.
%Lenaerts2009 - modelling the topological evolution of chemical networks

\begin{center}
	\scriptsize
	\begin{longtable}{@{}
			p{\dimexpr 0.15\textwidth-2\tabcolsep}
			p{\dimexpr 0.2\textwidth-2\tabcolsep}
			p{\dimexpr 0.25\textwidth-2\tabcolsep}
			p{\dimexpr 0.25\textwidth-2\tabcolsep}
			p{\dimexpr 0.15\textwidth-2\tabcolsep}
			@{}}
			\toprule
			Reference & Goals & Theoretical argument & Description & Summary \\
			\midrule\endhead
					
			\cite{Wrobel2012} & 
			``\ldots steps towards realizing a biologically realistic system'', motivational autonomy in searching for food and water while needing to perform work &
			Extension of previous work to three-resource problem; biological inspirations without explicit grounding;  appeal to theories of minimal cognition, ecological grounding and motivational autonomy &
			GRN driven by inputs from environmental sensors and driving output motor actuators. 
			Abstractions of food, water and work (called ``beauty'') - marked by scents that can be detected by animat sensors - food and water are required to generate motive energy (through a simulation of a microbial fuel cell). 
			Isolated mobile animats, synchronous generational evolutionary model with fixed population size, explicit fitness function and binary tournament selection&
			\\
					
			\cite{Prieto2010,Trueba2011,Trueba2012} &
			Evolution guided by energy-regulation towards a fixed objective; objective of collective gathering of blocks requires coordinated behaviours from agents that develop specializations &
			Situated and embodied (from ``living'' in environment), no biological significance claimed &
			Agents are unchanging physical robots; active life-span determined by internal energy-analogue.
			Static morphology; agent energy-analogue depleted by activity and increased by performing specific tasks.
			Asynchronous generations, local interactions determine reproduction, no global competition, no fitness function; evolution (only of behaviour) happens in real-time rather than offline before deployment &
			\\
					
			\cite{Lobo2010} & 
			Emergence of novelty and diversity; artificial genome model similar to that in \textcite{Reil:1999rp}, development of organisms modelled by springs to path-follow; three cell-types; results classified by hand (exploratory method) &
			``\ldots evolution of genetic regulation could be a sufficient condition for the emergence of novelty and diversity''. No previous study of regulation has succeeded in showing both novelty and diversity under evolution. &
			Embedding purely through physics modelling.
			No resource model - goal is path following.
			Standard EA population model - individuals are assessed by the EA in isolation & 
			Four behaviours considered evidence of "novelty" and "diversity"; no predefined measure (so concerns of seeing what wish to see); critical assumption without evidence that results will extend ad infinitum\\
					
			\cite{Taylor1999,Taylor:1999sc}&
			Creation of \gls{alife}&
			Extension of Tierra \cite{Ray1991} by cell regulation, parallel processes, energy modelling \cite[p.4]{Taylor:1999sc}.
			Cells are explicitly modelled as bitstrings which run as programs&
			Adhoc theoretical model (essentially Tierra plus some likely improvements)\\
					
			\cite{Hutton2007,Hutton2002} &
			Artificial system capable of life-like \gls{oee} (creativity) &
			Based on hypothesis (materiality, interactions, embedding) of \textcite{Taylor2001}, better expressed in \textcite[p.341]{Hutton2002}; membrane to allow individuals to benefit from innovations by protecting internal reactions from others&
			\Gls{achem} from \textcite{Hutton2002} used to construct all elements in world - material and embedded.
			Atoms in \gls{achem} of different types and states; reaction rules; otherwise no energy modelling; only physics modelling is the concept of location (either integer-coordinates or real number-coordinates) and impossibility of colocation. Floods are used to recycle raw materials.
			Individuals (each bounded by a membrane) with the capability for division and mutation. Raw materials (atoms) are only required for division&
			Interactions between individuals are limited to effects on shared environment (niche construction without direct interaction), so one element of hypothesis untested; restrictions on open-endedness: evolution of new enzymes unobserved as extremely unlikely, and genotype-phenotype mapping hard-coded (unevolvable)\\
					
			%\cite{Stepney2011} & & & & \\
			%\cite{Korb2009} & & & & \\
			%\cite{Dorin2008} & & &  & \\
			%\cite{Spector:2007qf} & & & & \\
			%\cite{Channon:2006st,Channon:2001ys} & & & & \\
			%\cite{Dorin:2006fk} & & & & \\
			%\cite{Nowostawski2005} & & & & \\
			%\cite{Channon:2001ys} & & & & \\
			%\cite{Maley:1999bs} & & & & \\
			%\cite{Jakobi1996} & & & & \\
			%\cite{Chaumont2010} & & & & \\
			%\cite{Faulconbridge2010} & & & & \\
			%\cite{Hickinbotham2010} & & & & \\
			%\cite{Hickinbotham2010a} & Details of evolutionary behaviour stemming from self-copying replicase molecule
			% 
			% \cite{Hickinbotham2010a}& Goal is not specifically about creativity or novelty instead being driven primarily by bacterial modelling (Plazzmid)
			% No membrane model - just evolution; implication is that membranes are too computationally expensive for novelty investigation \cite[p.88]{Hickinbotham2010}\\
					
			% \cite{Fernando:2008xy,Fernando:2007pf} & 
			% Driven by origin-of-life objectives (``the evolution of chemical networks that lead to autonomous systems'') so attempt to establish a phylogeny that includes first autonomous system & 
			% Simulation of laboratory experiment of lipid aggregates in a reactor. Strong theoretical justification. Hypothesis supported by inspection - single entities that are examples; Novelty is implicit in overall goals with sole quantitative measure (``fitness``) of integral of quantity and size - more and bigger is better&
			% Molecules and food molecules share same representation and chemistry&
			% Molecules, light-energy and thermodynamics&
			% Hill-climbing algorithm replaces parent with first 10\% better-fitness child; explicit fitness calculation&
			% Experiment reported in \textcite{Fernando:2008xy} generates phylogeny which is then examined and interpreted - danger of arguing from specific to general. Shows only possible, no more.\\
					
			%\cite{Suzuki2008} & Self-movement & & & & \\
					
			% \cite{Dorin:2006fk} &
			% Demonstrate that \gls{achem} can support a simple ecology - autotrophs and heterotrophs, with interactions between organisms and between organisms and the abiotic environment&
			% Loosely based on simplified model of generic terrestrial ecosystem &
			% All elements emerge from \gls{achem}&
			% Atoms and energy&
			% No evolutionary model; proposed ecosystem made up of a number of component organisms&
			% Description of model only - no results; adhoc theoretical justification and choice of ecosystem elements\\
					
			%\cite{Tominaga2005,Tominaga2009,Tominaga2007} & Not about novelty/creativity, but biological modelling including replication
			% Ono, N., & Ikegami, T. (2000). Self-maintenance and self-reproduction in an abstract cell model. Journal of Theoretical Biology, 206, 243Ð253
			% \\
					
					
			\bottomrule
			\caption{Previous Work on Open-Ended Evolution in Artificial Evolution\label{tbl:previousworkcreativity}}
		\end{longtable}
	\end{center}
	
	\begin{center}
		\scriptsize
		\begin{longtable}{@{}
				p{\dimexpr 0.15\textwidth-2\tabcolsep}
				p{\dimexpr 0.2\textwidth-2\tabcolsep}
				p{\dimexpr 0.25\textwidth-2\tabcolsep}
				p{\dimexpr 0.25\textwidth-2\tabcolsep}
				p{\dimexpr 0.15\textwidth-2\tabcolsep}
				@{}}
				\toprule
				Reference                            & Goals & Theoretical argument & Description & Summary \\
				\midrule\endhead
				%Wrobel2012		&Motivational autonomy in searching for food and water while needing to perform work. GRN driven by inputs from environmental sensors and driving output motor actuators. Abstractions of food, water and work (called ``beauty'') - marked by scents that can be detected by animat sensors - food and water are required to generate motive energy (through a simulation of a microbial fuel cell).  Isolated mobile animats, synchronous generational evolutionary model with fixed population size, explicit fitness function and binary tournament selection\\
				%Miconi:2007xh	& \\
				%Taylor:1999sc	&Extension of Tierra (Ray1991) by cell regulation, parallel processes, energy modelling (Taylor:1999sc, p4).Cells are explicitly modelled as bitstrings which run as programs. Adhoc theoretical model (essentially Tierra+)\\
				%Korb2009,Dorin2008,Dorin:2006fk& \\
				%Spector:2007qf& \\
				%Channon:2006st,Channon:2001ys& \\
				%Nowostawski2005& \\
				%Komosinski2000,Komosinski1999& \\
				%Maley:1999bs& \\
				%Jakobi1996& \\
				\cite{Wrobel2012} & 
				``\ldots steps towards realizing a biologically realistic system'', motivational autonomy in searching for food and water while needing to perform work &
				Extension of previous work to three-resource problem; biological inspirations without explicit grounding;  appeal to theories of minimal cognition, ecological grounding and motivational autonomy &
				GRN driven by inputs from environmental sensors and driving output motor actuators. 
				Abstractions of food, water and work (called ``beauty'') - marked by scents that can be detected by animat sensors - food and water are required to generate motive energy (through a simulation of a microbial fuel cell). 
				Isolated mobile animats, synchronous generational evolutionary model with fixed population size, explicit fitness function and binary tournament selection&
				\\
						
				\cite{Prieto2010,Trueba2011,Trueba2012} &
				Evolution guided by energy-regulation towards a fixed objective; objective of collective gathering of blocks requires coordinated behaviours from agents that develop specializations &
				Situated and embodied (from ``living'' in environment), no biological significance claimed &
				Agents are unchanging physical robots; active life-span determined by internal energy-analogue.
				Static morphology; agent energy-analogue depleted by activity and increased by performing specific tasks.
				Asynchronous generations, local interactions determine reproduction, no global competition, no fitness function; evolution (only of behaviour) happens in real-time rather than offline before deployment &
				\\
						
				\cite{Lobo2010} & 
				Emergence of novelty and diversity; artificial genome model similar to that in \textcite{Reil:1999rp}, development of organisms modelled by springs to path-follow; three cell-types; results classified by hand (exploratory method) &
				``\ldots evolution of genetic regulation could be a sufficient condition for the emergence of novelty and diversity''. No previous study of regulation has succeeded in showing both novelty and diversity under evolution. &
				Embedding purely through physics modelling.
				No resource model - goal is path following.
				Standard EA population model - individuals are assessed by the EA in isolation & 
				Four behaviours considered evidence of "novelty" and "diversity"; no predefined measure (so concerns of seeing what wish to see); critical assumption without evidence that results will extend ad infinitum\\
						
				\cite{Taylor1999,Taylor:1999sc}&
				Creation of \gls{alife}&
				Extension of Tierra \cite{Ray1991} by cell regulation, parallel processes, energy modelling \cite[p.4]{Taylor:1999sc}.
				Cells are explicitly modelled as bitstrings which run as programs&
				Adhoc theoretical model (essentially Tierra plus some likely improvements)\\
						
				\cite{Hutton2007,Hutton2002} &
				Artificial system capable of life-like \gls{oee} (creativity) &
				Based on hypothesis (materiality, interactions, embedding) of \textcite{Taylor2001}, better expressed in \textcite[p.341]{Hutton2002}; membrane to allow individuals to benefit from innovations by protecting internal reactions from others&
				\Gls{achem} from \textcite{Hutton2002} used to construct all elements in world - material and embedded.
				Atoms in \gls{achem} of different types and states; reaction rules; otherwise no energy modelling; only physics modelling is the concept of location (either integer-coordinates or real number-coordinates) and impossibility of colocation. Floods are used to recycle raw materials.
				Individuals (each bounded by a membrane) with the capability for division and mutation. Raw materials (atoms) are only required for division&
				Interactions between individuals are limited to effects on shared environment (niche construction without direct interaction), so one element of hypothesis untested; restrictions on open-endedness: evolution of new enzymes unobserved as extremely unlikely, and genotype-phenotype mapping hard-coded (unevolvable)\\
						
				\cite{Stepney2011}                   &       &                      &             &         \\
				\cite{Korb2009}                      &       &                      &             &         \\
				\cite{Dorin2008}                     &       &                      &             &         \\
				\cite{Spector:2007qf}                &       &                      &             &         \\
				\cite{Channon:2006st,Channon:2001ys} &       &                      &             &         \\
				\cite{Dorin:2006fk}                  &       &                      &             &         \\
				\cite{Nowostawski2005}               &       &                      &             &         \\
				\cite{Channon:2001ys}                &       &                      &             &         \\
				\cite{Maley:1999bs}                  &       &                      &             &         \\
				\cite{Jakobi1996}                    &       &                      &             &         \\
				\cite{Chaumont2010}                  &       &                      &             &         \\
				\cite{Faulconbridge2010}             &       &                      &             &         \\
				\cite{Hickinbotham2010}              &       &                      &             &         \\
						
				% \cite{Conforth2009} & Not explicitly about creativity, but top-down model incorporating stoichiometry (material transformations), energy flow and evolution. Shows improvements in stability and diversity of agents. Goal is eventual embedding of agents into robots.
						
				% \cite{Komosinski2009,Komosinski2000} & 
				% FramSticks - an experimental platform for investigating open-ended evolution; Genetic encoding (both direct and developmental) of Brain (neural network), Receptors (sensors), Effectors (muscles) and Body with multiple alternative physics simulators; channels for signals between actors&
				%  & & & \\
						
				% \cite{Fernandez2012} &
				% Evolution of ecological communities, particularly diversification (genotype and phenotype variation) &
				% \Gls{ibm}, selection of biological inspirations (gene duplication and development as mechanism for enabling diversification), little justification for choices, little support for claims of biological relevance beyond similarity of two patterns
				% Fitness based on light captured by each plant in competition with others.
				% Effectively none.
				% Generational model with fixed number of synchronous generations, variable population with complete replacement at each generation, explicit fitness determines reproduction rate&
				% \\
						
				% \cite{Sayama2011} & Swarm Chemistry & & & & \\
				%\cite{Sayama2011} explores ability of Swarm Chemistry - an integration of Artificial Chemistry with Swarm Robotics - to achieve Open-Ended Evolution. The models are rather non-biological, and interpretation as biology is difficult. One specific difficulty with the results as reported, noted by the author, is that all results were obtained by visual inspection, and that measures need to be developed based on some standard works.
						
				%\cite{Hickinbotham2011a} & Proposal for novelty - not a model yet
				%\cite{Hickinbotham2010a} & Details of evolutionary behaviour stemming from self-copying replicase molecule
				% 
				% \cite{Hickinbotham2010a}& Goal is not specifically about creativity or novelty instead being driven primarily by bacterial modelling (Plazzmid)
				% &
				% &
				% &
				% &
				% No membrane model - just evolution; implication is that membranes are too computationally expensive for novelty investigation \cite[p.88]{Hickinbotham2010}\\
						
				% \cite{Fernando:2008xy,Fernando:2007pf} & 
				% Driven by origin-of-life objectives (``the evolution of chemical networks that lead to autonomous systems'') so attempt to establish a phylogeny that includes first autonomous system & 
				% Simulation of laboratory experiment of lipid aggregates in a reactor. Strong theoretical justification. Hypothesis supported by inspection - single entities that are examples; Novelty is implicit in overall goals with sole quantitative measure (``fitness``) of integral of quantity and size - more and bigger is better&
				% Molecules and food molecules share same representation and chemistry&
				% Molecules, light-energy and thermodynamics&
				% Hill-climbing algorithm replaces parent with first 10\% better-fitness child; explicit fitness calculation&
				% Experiment reported in \textcite{Fernando:2008xy} generates phylogeny which is then examined and interpreted - danger of arguing from specific to general. Shows only possible, no more.\\
						
				%\cite{Suzuki2008} & Self-movement & & & & \\
						
						
				% \cite{Dorin:2006fk} &
				% Demonstrate that \gls{achem} can support a simple ecology - autotrophs and heterotrophs, with interactions between organisms and between organisms and the abiotic environment&
				% Loosely based on simplified model of generic terrestrial ecosystem &
				% All elements emerge from \gls{achem}&
				% Atoms and energy&
				% No evolutionary model; proposed ecosystem made up of a number of component organisms&
				% Description of model only - no results; adhoc theoretical justification and choice of ecosystem elements\\
						
				%\cite{Tominaga2005,Tominaga2009,Tominaga2007} & Not about novelty/creativity, but biological modelling including replication
				% Ono, N., & Ikegami, T. (2000). Self-maintenance and self-reproduction in an abstract cell model. Journal of Theoretical Biology, 206, 243Ð253
				% &
				% &
				% &
				% &
				% &
				% \\
						
						
				\bottomrule
				\caption{Previous Work on Open-Ended Evolution in Artificial Evolution\label{tbl:previousworkcreativity}}
			\end{longtable}
		\end{center}
		
		\hypertarget{rewriting-or-string-based}{\section{Rewriting or
				String-based}\label{rewriting-or-string-based}}
		
		\hypertarget{antonakopoulos2011th---a-common-genetic-representation}{\subsection{Antonakopoulos:2011th
				- ``A common genetic
				representation\ldots{}''}\label{antonakopoulos2011th---a-common-genetic-representation}}
		
		\begin{itemize}
			\item
			
			One developmental mapping for several structures or species e.g., CAs,
			and boolean networks - sparsely connected networks
			
			\item
			
			Using L-systems - one for cells/nodes, and the other for connectivity
			rules
			
			\item
			
			Developmental process ``capable of expressing developmental actions
			e.g., growth and differentiation'' and ``able to express a large
			variety of topologies within each architecture\ldots{}''
			
		\end{itemize}
		
		\hypertarget{lenski2003---the-evolutionary-origin-of-complex-features}{\subsection{Lenski2003
				- ``The evolutionary origin of complex
				features''}\label{lenski2003---the-evolutionary-origin-of-complex-features}}
		
		\begin{itemize}
			\item
			
			Modelled in Avida
			
			\item
			
			``Our experiments demonstrate the validity of the hypothesis, first
			articulated by Darwin and supported today by comparative and
			experimental evidence, that complex features generally evolve by
			modifying existing structures and functions.''
			
			\item
			
			``Some simpler functions were accessible from the ancestor by
			relatively few mutations, and these served as a foundation on which
			more complex features were built. The foundational role of simpler
			functions in the origin of more complex ones was evident in the
			overlap of the genetic networks underlying their expression, and the
			frequent loss of simpler functions as side-effects of mutations
			yielding more complex function''
			
		\end{itemize}
		
		\hypertarget{suzuki-and-tanaka-in-alife-viip54}{\subsection{Suzuki and
				Tanaka in Alife VIIp54}\label{suzuki-and-tanaka-in-alife-viip54}}
		
		\begin{itemize}
			\item
			
			``Abstract Rewriting System on Multisets'' for computing with a
			membrane (unlike other referenced works which concern membrane
			formation)
			
			
			\begin{itemize}
				\item
				
				\href{https://books.google.co.nz/books?id=0J8kQEjXe38C\&pg=PA521\&lpg=PA521\&dq=suzuki+Abstract+Rewriting+System+on+Multisets\&source=bl\&ots=nmUxuxeEW2\&sig=5F8aDFQyFP17NMNNNBviRJjsL2E\&hl=en\&sa=X\&redir_esc=y\#v=onepage\&q=suzuki\%20Abstract\%20Rewriting\%20System\%20on\%20Multisets\&f=false}{\emph{ARMS
						introduced}} in Alife V, example of an abstract rewriting system
				(defined in paper) (common example rewriting in calculus)
				
				
				\begin{itemize}
					\item
					
					Chemical reactions can be seen as reaction rules
					
					\item
					
					unassociated with any specific architecture such as turing-system
					or CA
					
					\item
					
					can leverage previous work on rewriting systems
					
				\end{itemize}
				\item
				
				Similar to Formal Grammar (as in L-systems, although there used for
				pattern formation, and here for temporal dynamics) and Fontana's
				Lambda-Calculus
				
				\item
				
				Normal form = death or steady-state = no rules can be applied and no
				inputs available
				
			\end{itemize}
			\item
			
			Within tradition of work on membrane computing and P-systems (1998)
			
			\item
			
			Not a string rewriting system
			
			\item
			
			Shown can simulate ``real bio-chemical reactions such as the
			"Belouzov- Zhabotinsky reaction(BZ\_reaction)(Suzuki1998) the
			BZ-reaction is a spontaneous chemical oscillation, and is considered
			as the basic mechanism of bio-chemical systems.''
			
			\item
			
			Solution a finite multiset of molecules; heating (a -\textgreater{}
			b,c,...) and cooling (a,b,c\ldots{}-\textgreater{}e) rules/reactions
			
			\item
			
			Appears that rules are fixed, and exact matching (no templates)
			
			\item
			
			``if we could extract computation in terms of living systems then we
			might obtain a new computational world\ldots{}.In a living system,
			every cell and every organ is acting in parallel and they are
			consisted of membranes\ldots{}.We expect that these studies lead us to
			a new computing paradigm which goes beyond the Turing Machine based
			computing paradigm.''
			
		\end{itemize}
		
		\hypertarget{dittrich1998---self-evolution-in-a-constructive-binary-string-system}{\subsection{Dittrich1998
				- ``Self-Evolution in a Constructive Binary String
				System''}\label{dittrich1998---self-evolution-in-a-constructive-binary-string-system}}
		
		\begin{itemize}
			\item
			
			``study the phenomena of life, not by simulating life as it is (weak
			AL) but by instantiating life as it could be (strong AL)'' {[}Langton,
			C. G. (1989). Artificial life{]}
			
			\item
			
			``An important property of most strong AL systems is that they contain
			the ability for self-reference. For instance, Ray's Tierra organisms
			are able to read, copy, and modify their own code {[}27{]}. In
			Fontana's algorithmic chemistry every object is a character string
			able to process other objects by using the lambda-calculus that maps
			the character string into an (active) function {[}13{]}. The dualism
			inherent in those systems can be traced back to Godel {[}15{]} who
			defined a mapping of mathematical statements into natural numbers ``
			that allowed self-reference, to Turing's universal machine {[}31{]},
			and to von Neumann's stored program computer {[}7{]}.''
			
			\item
			
			``Here, we will concentrate on dynamic phenomena, especially on the
			emergence of prebiotic evolution''
			
			\item
			
			``The term self-evolution should refer to an evolutionary process
			within a population system where the components responsible for the
			evolutionary behavior are (only) the individuals of the population
			system itself. Every variation is carried out by the individuals.
			Selection pressure is generated implicitly through interaction among
			the individuals and not by external agents. The system must not
			contain a component that can be identified as a fitness function or
			global operators performing selection or variation (e.g.,
			crossover).''
			
			\item
			
			Introduces S,R,A classification scheme for artificial chemistry -
			here, S=`` binary strings with a constant length of 32 bits'',
			R=s1+s2-\textgreater{}s3, and A=''simulates a well-stirred tank
			reactor with mass-action kinetics, which assures that the probability
			of a collision is proportional to the product of the concentration of
			the colliding objects'' from earlier work (Fontana, Kauffman etc)
			
			\item
			
			A=``1. Select two objects s1,s2 from the soup randomly, without
			removing them. 2. If there exists a reaction s1 + s2 to s3 and the
			filter condition f (s1,s2,s3) holds, replace a randomly selected
			object of the soup by s3.'', s1 and s2 are not consumed, rather they
			act as catalysts. Chosen as this shown capable of hypercyclic
			organisation
			
			\item
			
			AND reaction and an automata reaction to generate s3 from s1, s2.
			Automata is a deterministic FSA, running s1 (4-bit fixed instructions)
			on s2.
			
			\item
			
			Measures (all macroscopic)
			
			
			\begin{itemize}
				\item
				
				Diversity
				
				\item
				
				Distance Distribution Complexity (from Kim)
				
				\item
				
				Productivity
				
				\item
				
				Innovativity (innovation = new string)
				
			\end{itemize}
			\item
			
			Experiments - initialised always with random (32-bit) strings
			
			\item
			
			Compared to ODE approach in Bagley1992 (AlifeV version)
			
		\end{itemize}
		
		\hypertarget{fontana1992}{\subsection{Fontana1992}\label{fontana1992}}
		
		\begin{itemize}
			\item
			
			AlChemy
			
		\end{itemize}
		
		\hypertarget{fenizio2000}{\subsection{Fenizio2000}\label{fenizio2000}}
		
		\begin{itemize}
			\item
			
			Original AlChemy reactions of form A+B-\textgreater{}C where C
			replaces an existing element (X)
			
			\item
			
			This system generates A+B-\textgreater{}C1+C2...CN where C is a
			multiset of size N. Done by modifying the original K rule to detach x2
			and eliminate both original elements (like reactants in chemistry)
			
			\item
			
			Uses combinators rather than lambdas
			
			\item
			
			To prevent from stopping (out of elements) added modification where
			randomly add/remove some elements
			
			\item
			
			Combinator first combines (appends) elements, each element other than
			first bracketed. Then each 1-term combinator applied to string, where
			it makes specific changes e.g., K x1x2s0-\textgreater{}x1s0 (s0 is
			remaining substring, may be null). Apply until no further reductions
			possible (that is, in normal form). Two combinators are equivalent if
			can be reduced to same combinator (and previously noted that order is
			not important - same results regardless of order).
			
			\item
			
			Free pool of atoms for conservation of ``mass''
			
		\end{itemize}
		
		\hypertarget{fenizio2001}{\subsection{Fenizio2001}\label{fenizio2001}}
		
		\begin{itemize}
			\item
			
			Follow-on to Fenizio2000
			
			\item
			
			Was at COGS Sussex
			
			\item
			
			Focus on ``identity as an entity separated from its environment'',
			that is, membrane formation. Graph used to model spatial structures
			
			\item
			
			``an artificial chemistry (AC) is embedded in a graph, with each
			molecule being a vertex of the graph and possible interactions being
			allowed only along the edges of the graph''
			
			\item
			
			``Molecules are built from a substrate of elements called atoms. There
			are seven types of atoms (I, K, W, R, B, C, S), each with a different
			function. The total number of atoms in the reactor is kept constant
			during a run. Free atoms (not bounded in molecules) are separately
			stored and form a global pool.''
			
			\item
			
			Provides definition of autopoeisis in terms of graphs
			
			\item
			
			Experiment to show spontaneous formation of autopoietic cells
			
			\item
			
			Rules for combinations of two molecules are predetermined - reaction
			mechanisms are described in Fenizio2000
			
		\end{itemize}
		
		\hypertarget{maley1999---four-steps-towards-open-ended-evolution}{\subsection{Maley1999
				- ``Four Steps towards Open-Ended
				Evolution''}\label{maley1999---four-steps-towards-open-ended-evolution}}
		
		\begin{itemize}
			\item
			
			``we would like the evolutionary system, like life, to continue to
			produce individuals of increasing complexity and diversity.'' -
			although note, following McShea, that much of life is single-celled
			and hasn't become much more complex in billions of years.
			
			\item
			
			Focus on diversity (to make progress)
			
			\item
			
			Some suggestions previously that diversity is bounded (at minimum, by
			number of molecules available for biosphere, also by energy and
			minimal populations and probably other things), and plateaus
			(punctuated equilibrium). Indicates two time constants
			
			
			\begin{itemize}
				\item
				
				fast - expansion to use available resources
				
				\item
				
				slow - innovations to open new adaptive space
				
			\end{itemize}
			\item
			
			Urmodel1 - neutral landscape, mutation, early stop before all niches
			filled (to prevent edge effects)
			
			
			\begin{itemize}
				\item
				
				32-bit genotype - mutation flips one-bit
				
				\item
				
				Get unbounded diversity, but no selection or heritable effect on
				fitness
				
			\end{itemize}
			\item
			
			Posits that need selection : ``Requirement 2 An open-ended
			evolutionary system must embody selection'' - because ``fails to meet
			one of the basic criteria of natural selection: the heritable
			variation has no effect on fertility'' {[}ignoring use of fertility as
			a fitness-analogue{]}, and from Bedau ``Requirement 3 ...continuing
			(`positive') new adaptive activity'' {[}ignore neutral theory, and
			accepts Bedau - perhaps to allow use of Anew as measure?{]}
			
			\item
			
			Urmodel2 - natural selection: mutation, ``dissimilarity'' for
			competitive advantage (justified by biological example of niche
			overlap theory (Levins, 1968)) - no increase in Anew
			
			\item
			
			Urmodel3 - selective sweep (hypothesis): parasites (mutations) and
			hosts (fixed genotypes). Fitness on degree of match between parasite
			and host bit patterns.
			
			
			\begin{itemize}
				\item
				
				Claim shows unbounded activity - ``the first known artificial
				evolutionary system demonstrating unbounded evolutionary activity''
				
				\item
				
				{[}Restricted by 32-bit genomes, no death{]}
				
				\item
				
				But probably not a unique or even significant result - ``The only
				trick is to defer the point when the model hits its true asymptotic
				behaviour for long enough that the growth dynamics of the model are
				themselves asymptotic in some sense''
				
			\end{itemize}
			\item
			
			Urmodel4 - ``the most important aspect of an organism's environment
			are the other organisms with which it interacts'' - add coevolution to
			Urmodel3 by letting hosts mutate
			
			\item
			
			Two ``distasteful aspects of Urmodel3'' leading to belief that metrics
			aren't right
			
			
			\begin{itemize}
				\item
				
				niches are imposed from outside, not endogenous - this becomes
				Requirement 4
				
				\item
				
				no surprise - claim is because not complex - ``A puddle of inert,
				multicoloured and diverse algae would not be nearly so inspirational
				as the rain forest.'' Again, a biological metaphor.
				
			\end{itemize}
		\end{itemize}
		
		\hypertarget{ofria2004---avida-a-software-platform-for-research-in-computational-evolutionary-biology}{\subsection{Ofria2004
				- ``Avida: A Software Platform for Research in Computational
				Evolutionary
				Biology''}\label{ofria2004---avida-a-software-platform-for-research-in-computational-evolutionary-biology}}
		
		\begin{itemize}
			\item
			
			``An approach to studying evolution...''
			
			\item
			
			``According to Daniel Dennett, ``...evolution will occur whenever and
			wherever three conditions are met: replication, variation (mutation),
			and differential fitness (competition)''''
			
			\item
			
			``(However, as Barton and Zuidema {[}3{]} note, general acceptance
			will ultimately hinge on whether artificial life researchers embrace
			or ignore the large body of population genetics literature.)''
			
			\item
			
			Difference with GAs - natural organisms must replicate themselves to
			pass on genetic information - ``final arbiter of fitness'', and
			interaction with other organisms and with environment
			
			\item
			
			Steen Rasmussen inspired by computer game core war - competing
			segments of simplified assembly code in core memory. With change to
			copy command to introduce mutations and hence evolutionary potential,
			core world created. But system ``collapsed into a non-living state''
			{[}non-living?{]} One possible reason - copying over existing
			organisms
			
			\item
			
			Tierra next year (relationship not stated) organisms had to allocate
			memory first before using. Initial selective pressure only from rate
			of replication. Sequential execution of organism code
			
			\item
			
			Avida summer of 1993 - better metering and measuring, and parallel
			code execution
			
			
			\begin{itemize}
				\item
				
				``In principle, the only assumption made about these
				self-replicating automata in the core Avida software is that their
				initial state can be described by a string of symbols (their genome)
				and that they autonomously produce offspring organisms. However, in
				practice our work has focused on automata with a simple von Neumann
				architecture that operate on an assembly-like language inspired by
				the Tierra system.''
				
				\item
				
				Instruction, read, write, and flow control heads for relative rather
				than absolute addressing - bit like a Turing tape machine
				
				\item
				
				Many instructions grouped into instruction sets. Default set has 26
				instructions
				
				\item
				
				Every program is valid
				
			\end{itemize}
			\item
			
			Phenotypes - ``The primary mode of environmental interaction is by
			inputting numbers from the environment, performing computations on
			those numbers, and outputting the results. The organisms receive a
			benefit for performing specific computations associated with
			resources''
			
			\item
			
			All very configurable, and complicated, but why? What rationale behind
			choices? More of a testbed for experiments, e.g., `` in one experiment
			we wanted to study a population that could not adapt, but that would
			nevertheless accumulate deleterious or neutral mutations through
			drift''
			
			\item
			
			``The quest to halt adaptation is only one example of a special
			feature in Avida; many more have been explored, and are continuously
			being added to the source code. The most successful features are all
			fully described in the documentation that comes with the software.''
			
		\end{itemize}
		
		\hypertarget{gardiner2007---a-framework-for-the-co-evolution-of-genes-proteins-and-a-genetic-code-within-an-artificial-chemistry-reaction-set}{\subsection{Gardiner2007
				- ``A framework for the co-evolution of genes, proteins and a genetic
				code within an artificial chemistry reaction
				set''}\label{gardiner2007---a-framework-for-the-co-evolution-of-genes-proteins-and-a-genetic-code-within-an-artificial-chemistry-reaction-set}}
		
		\begin{itemize}
			\item
			
			String based chemistry to investigate protein metabolism evolution
			under genetic control.
			
			\item
			
			Three types of molecule - protein, gene and service molecule - react
			in particular ways associated with the types of interacting molecules.
			Type and pattern of molecules defines type of interaction.
			
		\end{itemize}
		
		\hypertarget{boolean-networks-and-graphs}{\section{Boolean Networks and
				Graphs}\label{boolean-networks-and-graphs}}
		
		\hypertarget{huning2000---a-search-for-multiple-autocatalytic-sets-in-artificial-chemistries-based-on-boolean-networks}{\subsection{Huning2000
				- ``A search for multiple autocatalytic sets in artificial chemistries
				based on boolean
				networks''}\label{huning2000---a-search-for-multiple-autocatalytic-sets-in-artificial-chemistries-based-on-boolean-networks}}
		
		\begin{itemize}
			\item
			
			No significant further
			\href{https://scholar.google.com/scholar?cites=6514200166149032694\&as_sdt=2005\&sciodt=0,5\&hl=en}{\emph{citations}}
			
			\item
			
			Extension of chemistry in Dittrich98 for shorter-length strings
			
			\item
			
			Uses \textbf{boolean networks} instead of algorithms as can encode a
			BN in short space; one string specifies BN, other the input. Output as
			in Dittrich filtered - a selection function - probabilistic based on
			desired goal e.g., all 1's. Slightly unclear how BN is encoded in
			string - ``RAM''s...
			
			\item
			
			Otherwise same algorithm as Dittrich98
			
			\item
			
			Initialised with random strings
			
		\end{itemize}
		
		\hypertarget{nellis2012---towards-meta-evolution-via-embodiment-in-artificial-chemistries}{\subsection{Nellis2012
				- ``Towards meta-evolution via embodiment in artificial
				chemistries''}\label{nellis2012---towards-meta-evolution-via-embodiment-in-artificial-chemistries}}
		
		\begin{itemize}
			\item
			
			No significant citations in Google Scholar
			
			\item
			
			University of York, CompSci, supervisor Stepney, part of Plazzmid
			project (Clark, Hickinbotham, Young, Clarke, Pay)
			
			\item
			
			``Our aim is to improve novelty-generation algorithms by making their
			biological models richer''
			
			\item
			
			Novelty-generation as goal/theme for meta-evolution
			
			\item
			
			No measure for novelty - not even sure is possible. But informal
			definition that says more novelty as result of more embodiment (this
			seems circular?) p87
			
			\item
			
			Embodiment as mechanism. Provides overview in 3.2
			
			\item
			
			AChems categorized as world/chemistry/constraints (p132), with
			constraints being energy model/binding model. RBN-World mentioned as a
			world where elements are RBN networks
			
			\item
			
			States that binding model needs to be rich - empirical evidence
			presented (StringMol) only
			
			\item
			
			Copying is a phenomenon, expressed through mechanisms at different
			levels of the AChem
			
			\item
			
			Explored through embodied copying mechanism built in GraphMol, with
			antecedent in StringMol (Hickingbotham2011)
			
			\item
			
			StringMol includes an embodied copying mechanism -
			start/at-end/char-copy/next. Some elements (at-end) use Smith-Waterman
			matching algorithm, which opens ability for evolution to modify the
			function of at-end (Nellis2012p143). But others, e.g., next, are not
			embodied. GraphMol makes next embodied
			
			\item
			
			Design is graph based (obviously..). ``The world defined by GraphMol
			contains chemicals (represented as graphs) that bind to each other via
			multiple binding sites, and then run simple computer programs (encoded
			in the graphs) that modify the binding of these chemicals.''. Why? No
			explicit rationale presented. Presumably StringMol starting point
			meant programs, copying, then graphs?
			
			\item
			
			Operates (like StringMol) at level of proteins/enzymes e.g.,
			replicases, DNA
			
			\item
			
			Runtimes in weeks to months
			
		\end{itemize}
		
		\hypertarget{nellis2014---computational-novelty-phenomena-mechanisms-worlds}{\subsection{Nellis2014
				- ``Computational novelty: Phenomena, mechanisms,
				worlds''}\label{nellis2014---computational-novelty-phenomena-mechanisms-worlds}}
		
		\begin{itemize}
			\item
			
			Summary of PhD thesis - computational novelty through embodiment
			
			\item
			
			Comparing StringMol and GraphMol
			
		\end{itemize}
		
		\begin{itemize}
			\item
			
			Emergent systems easy to code but produce surprising results. Cannot
			predict results from rules (or in fact easily predict rules from
			desired results. One-way function, akin to encryption)
			
			\item
			
			Standard GAs search through a fixed space - cannot surprise, limited
			range
			
			\item
			
			Need something emergent
			
			\item
			
			Mechanism of evolution must be itself evolvable; individuals and
			environment interact to give new ways of producing new individuals.
			Dynamics required for novelty-generation.
			
			\item
			
			Quick defined embodiment in terms of two dynamical systems mutually
			affecting each other - no need for a physical world. System modifies
			environment. Doesn't address if system is constructed from
			environment. Autopoiesis does say though that system built from
			environment. But autopoiesis talks about maintenance not evolution
			
			\item
			
			Functions such as template copying must be embodied mechanisms in the
			world - so can be affected and evolved
			
			\item
			
			Stringmol and Graphmol have embodied template copying, done in
			different ways. Different computational models result in different
			properties - ``Stringmol exhibits macro-mutation and two chemical
			copying; GraphMol exhibits two types of quasispecies, cooperative and
			parasitic. These two systems use the same domain (emergent evolution)
			and metamodel (machines copying strings), but different computational
			models.''
			
		\end{itemize}
		
		\hypertarget{faulconbridge2010-and-2011---rbn-world-sub-symbolic-artificial-chemistry-for-artificial-life}{\subsection{Faulconbridge2010
				and 2011 - ``RBN-World: sub-symbolic artificial chemistry for artificial
				life''}\label{faulconbridge2010-and-2011---rbn-world-sub-symbolic-artificial-chemistry-for-artificial-life}}
		
		\begin{itemize}
			\item
			
			No significant citations in Google Scholar
			
			\item
			
			University of York, Department of Biology, co-authored paper with
			Hickinbotham and Nellis, supervisors Susan Stepney, Leo Caves and
			Julian Miller
			
			\item
			
			RBNWorld
			
			
			\begin{itemize}
				\item
				
				Entities are described as a form of Random Boolean Network, with the
				addition of a bonding mechanisms to allow for composition and
				decomposition of RBNs. A number of parameters affect the behaviour
				of the chemistry, and so a series of experiments sampled from the
				parameter-space, and then used a GA, to search for interesting
				variants as measured by non-catalysed loops (ideal measures of
				auto-catalytic sets and Hypercycles too rare for use as a measure)
				(chap 8)
				
				\item
				
				Gillespie-like" - random reaction, random time - not correlated in
				any way with reaction energies or rates (Faulconbridge2011§8.4.3.1)
				
				\item
				
				bRBNs used as atomic elements, properties determined by the network
				
				\item
				
				Larger structures are formed by ``bonding'' two independent bRBNs at
				each bRBNs bonding node. ``All reactions are between two reactants;
				it is assumed that more complicated reactions can be expressed as a
				series of two-reactant reactions with intermediate structures.``
				Record kept of composition so that decomposition can be easily done.
				
				\item
				
				Many design choices - bonding mechanism etc. Examination of
				alternatives done by searching with EA
				
				\item
				
				Each bRBN is a RBN, made up of a number of nodes, each with an
				initial state (true/false) assigned randomly and with a input/output
				matrix assigned randomly. Finally k(=2) inputs are established per
				node. Synchronous state update. All based on Kauffman1969
				(interestingly, although noted as ``original'' so later work known)
				
				\item
				
				bonding method uses ''cycle length as the bonding property and
				equality as the bonding criterion....bonds only exist between bRBNs
				that have the same cycle length.'' in initial examples at least n=5
				and b(k?)=2, and alternatives examined using EA. After initial bond
				formation recalculate cycle lengths, and check again for equality -
				might result in decomposition.
				
			\end{itemize}
			\item
			
			Three types of chemistries
			
			
			\begin{itemize}
				\item
				
				Symbolic - symbols/molecules have no inherent meaning, so no
				``implicit reactions''
				
				\item
				
				Structured - one or more atoms arranged in a structure (string,
				tree, graph..) with bonds within molecules - unlimited, but
				consequently computationally expensive
				
				\item
				
				Sub-symbolic - emergence of properties from lower-levels (ala real
				chemistry, also neural networks). Symbols (atoms) have internal
				structure which gives properties.
				
			\end{itemize}
			\item
			
			Describes two generic strategies for the selection of reactants,
			spatial and aspatial, where the primary difference is whether
			molecular position is a factor in reactant selection
			
			\item
			
			Measures are Synthesis, Self-Synthesis, Decomposition, Substitution,
			and Catalysis (Faulconbridge2011§7), and non-catalysed `loops' (ideal
			measures of auto-catalytic sets and Hypercycles too rare for use as a
			measure) (Faulconbridge2011§8)
			
			\item
			
			Random sampling + EA (Fitness function (§8.4.3.2) is based on
			non-catalysed loops (§4.3.9.4))
			
			\item
			
			"identifies three types of \textbackslash{}emph\{mixing method\} or
			reactor algorithm
			
			\item
			
			Contains an interesting discussion mapping these desirable properties
			onto the emergent properties that are then required of an
			\textbackslash{}gls\{achem\}
			
			\item
			
			Some advantages claimed over Hutton (emergence and computational
			intractability p190)
			
			\item
			
			``As the choice to use RBNs as the sub- symbolic representation in
			RBN-World was based on limited information. As a discrete dynamical
			system that is computationally tractable yet also spans a wide range
			of behaviours, RBNs met the appropriate criteria. It is not expected
			that RBNs are the best representation however; others may be more
			suitable for particular emergent properties.``
			
		\end{itemize}
		
		\hypertarget{squirm3}{\section{Squirm3}\label{squirm3}}
		
		\hypertarget{lucht2012---size-selection-and-adaptive-evolution-in-an-artificial-chemistry}{\subsection{Lucht2012
				- ``Size Selection and Adaptive Evolution in an Artificial
				Chemistry``}\label{lucht2012---size-selection-and-adaptive-evolution-in-an-artificial-chemistry}}
		
		\begin{itemize}
			\item
			
			http://www.cis.utas.edu.au/users/mwlucht/BTL.html.
			
			\item
			
			Challenge to community - "to develop a system in which, starting with
			a soup of free atoms and a simple ``bootstrap'' chemistry, a cell-like
			creature similar to the one in H-41 evolves."
			
			\item
			
			Based on Hutton2007 Squirm3 chemistry, using Hutton's floods
			
			\item
			
			Squirm3 - fixed molecule types, and pre-defined reactions for
			replication and gene-sequence transcription, and so although capable
			of interesting behaviour is not capable of unlimited extension
			
			\item
			
			Added reaction types to address Squirm3's ``global-extinction problem
			and showing how quasi-universal enzymes can evolve
			
		\end{itemize}
		
		\hypertarget{hutton2007---evolvable-self-reproducing-cells-in-a-two-dimensional-artificial-chemistry}{\subsection{Hutton2007
				- ``Evolvable Self-Reproducing Cells in a Two-Dimensional Artificial
				Chemistry ``
			}\label{hutton2007---evolvable-self-reproducing-cells-in-a-two-dimensional-artificial-chemistry}}
		
		\begin{itemize}
			\item
			
			http://www.sq3.org.uk
			
			\item
			
			Enzymes can act as catalysts so G affects P; but enzyme evolution too
			improbable for OEE
			
			\item
			
			2D grid of squares (lattice), spring force for membrane
			
			\item
			
			Enzymes can now affect all reactions except enzyme production; in
			practice too slow
			
			\item
			
			Production rules hardcoded into AChem
			
			\item
			
			Only with raw materials in environment, niche construction through
			adaptation to availability of raw materials
			
			\item
			
			AChem hand-built with reactions instead of enzymes
			
			\item
			
			Prespecified reactions - see Faulconbridge2011 p.49
			
			\item
			
			Hutton2007 and Hutton2002
			
		\end{itemize}
		
		\begin{itemize}
			\item
			
			Artificial system capable of life-like \textbackslash{}gls\{oee\}
			(creativity)
			
			\item
			
			Based on hypothesis (materiality, interactions, embedding) of
			\textbackslash{}textcite\{Taylor2001\}, better expressed in
			\textbackslash{}textcite{[}p.341{]}\{Hutton2002\}; membrane to allow
			individuals to benefit from innovations by protecting internal
			reactions from others
			
			\item
			
			\textbackslash{}Gls\{achem\} from
			\textbackslash{}textcite\{Hutton2002\} used to construct all elements
			in world - material and embedded.
			
			\item
			
			Atoms in \textbackslash{}gls\{achem\} of different types and states;
			reaction rules; otherwise no energy modelling; only physics modelling
			is the concept of location (either integer-coordinates or real
			number-coordinates) and impossibility of colocation. Floods are used
			to recycle raw materials.
			
			\item
			
			Individuals (each bounded by a membrane) with the capability for
			division and mutation. Raw materials (atoms) are only required for
			division
			
			\item
			
			Interactions between individuals are limited to effects on shared
			environment (niche construction without direct interaction), so one
			element of hypothesis untested; restrictions on open-endedness:
			evolution of new enzymes unobserved as extremely unlikely, and
			genotype-phenotype mapping hard-coded (unevolvable)
			
		\end{itemize}
		
		\hypertarget{atomsmolecules-in-ool}{\section{Atoms/Molecules in
				OOL}\label{atomsmolecules-in-ool}}
		
		\hypertarget{vasas2015---primordial-evolvability-impasses-and-challenges}{\subsection{Vasas2015
				- ``Primordial evolvability: Impasses and
				challenges''}\label{vasas2015---primordial-evolvability-impasses-and-challenges}}
		
		\begin{itemize}
			\item
			
			Lewontin variation, fitness differences, heritability of fitness
			
			\item
			
			Maynard-Smith variation, multiplication, heritability
			
			\item
			
			Greisemer 2000 states these are fundamentally different
			
			\item
			
			But if inheritance is statistical only then ok; but literature
			stresses digitally encoded although might just be cognitive bias
			
			\item
			
			Unit of selection very important to define now in abiogenesis
			
			\item
			
			Can evolution happen when information transfer is non-digital?
			Specifically where there is a parent-offspring correlation in
			molecular composition?
			
			\item
			
			Based on GARD Segre et al 1998
			
			
			\begin{itemize}
				\item
				
				Compositional inheritance
				
				\item
				
				claim made that GARD is capable of darwinian evolution, but
				population analysis showed not in response to directional selection
				(Vasas2010)
				
				\item
				
				Eigen threshold applies - mutation rates (see issue of replication -
				big differences between parent and child) overwhelm selection
				
			\end{itemize}
			\item
			
			Ganti and Eigen showed that distinct, organizationally different
			alternative autocatalytic networks in same environment might compete
			and fittest would prevail (e.g., autocatalytic networks as units of
			selection)
			
			
			\begin{itemize}
				\item
				
				Check number of network components (=autocatalytic networks). In
				GARD, end up with just one big component
				
			\end{itemize}
			\item
			
			GARD does not result in selectable replicating entities - there is no
			replication as certain highly catalytic molecules determine the
			properties of the compotype, and these are not inherited equally -
			instead a child may or may not inherit one of these molecules and so
			its properties may be similar to or very different from its parent
			
			\item
			
			Kauffman 1986 reflexively autocatalytic polymer networks however are
			evolvable. First, likelihood of such networks higher than expected
			(Hordijk and Steel) and second, in Vasas2012 putting these networks
			into compartments (so not well-stirred) then can do directional
			selection
			
			\item
			
			Differences between GARD and RAPN:
			
			
			\begin{itemize}
				\item
				
				kinetics of growth - RAPN has ligation, GARD does not. Ligation
				allows new components to be formed with new properties
				
				\item
				
				search for adaptations - GARD has fixed catalysis, Kauffman does not
				- mutant polymers can arise, be incorporated into the set, and
				influence its fitness
				
			\end{itemize}
			\item
			
			Minimal conditions for OEE
			
			
			\begin{itemize}
				\item
				
				very rich combinatorial generative mechanism e.g., organic
				chemistry. Underlies the evolvability of niches (Dorin and Korb
				2011)
				
				\item
				
				unlimited heredity - number of possible heritable types should
				astronomically exceed individuals in population
				(Maynard-Smith:1995lw)
				
				\item
				
				inexhaustible fitness landscape - implies rich, dynamical
				environment
				
				\item
				
				Cannot state in advance possible preadaptations. (not clear why this
				is a condition\ldots{}seems to be saying that evolution is
				algorithmic but results are not - are emergent) Richness part of
				real chemistry, not from representations of chemistry which are
				limited - necessary requirement for OEE in material systems. (But
				real chemistry is also limited...just not as much)
				
			\end{itemize}
		\end{itemize}
		
		\hypertarget{hurndall2014---pre-template-metabolic-replicators-genotype-phenotype-decoupling-as-a-route-to-evolvability}{\subsection{Hurndall2014
				- ``Pre-template Metabolic Replicators: Genotype-Phenotype Decoupling as
				a Route to
				Evolvability''}\label{hurndall2014---pre-template-metabolic-replicators-genotype-phenotype-decoupling-as-a-route-to-evolvability}}
		
		\href{https://mitpress.mit.edu/sites/default/files/titles/content/alife14/978-0-262-32621-6-ch141.pdf}{\emph{https://mitpress.mit.edu/sites/default/files/titles/content/alife14/978-0-262-32621-6-ch141.pdf}}
		
		\begin{itemize}
			\item
			
			``The aim of this model is to identify a minimal set of physical and
			chemical requisites to heritable variation in a metabolic model of
			compartmentalised autocatalytic sets''
			
			\item
			
			Modelled by ODEs
			
			\item
			
			Tested under artificial selection where protocell concentration
			profile similarity to a target profile used to adjust reaction rates
			for that step
			
		\end{itemize}
		
		\hypertarget{soros2014---identifying-necessary-conditions-for-open-ended-evolution-through-the-artificial-life-world-of-chromaria}{\subsection{Soros2014
				- ``Identifying Necessary Conditions for Open-Ended Evolution through
				the Artificial Life World of
				Chromaria''}\label{soros2014---identifying-necessary-conditions-for-open-ended-evolution-through-the-artificial-life-world-of-chromaria}}
		
		\href{https://mitpress.mit.edu/sites/default/files/titles/content/alife14/978-0-262-32621-6-ch128.pdf}{\emph{https://mitpress.mit.edu/sites/default/files/titles/content/alife14/978-0-262-32621-6-ch128.pdf}}
		
		\begin{itemize}
			\item
			
			``Standish (2003), which is that openendedness depends fundamentally
			on the continual production of novelty.''
			
			\item
			
			``Holland (1994) similarly investigates the necessary conditions for
			emergent phenomena in complex adaptive systems.''
			
			\item
			
			General prereqs: good genetic representation, ``sufficiently large
			world for every individual to be evaluated'', and a seed or starting
			point
			
			\item
			
			Hypothesis - four necessary conditions for OEE (left open if
			sufficient)
			
			
			\begin{itemize}
				\item
				
				A rule should be enforced that individuals must meet some minimal
				criterion (MC) before they can reproduce, and that criterion must be
				nontrivial.
				
				\item
				
				The evolution of new individuals should create novel opportunities
				for satisfying the MC
				
				\item
				
				Decisions about how and where individuals interact with the world
				should be made by the individuals themselves.
				
				\item
				
				The potential size and complexity of the individuals' phenotypes
				should be (in principle) unbounded.
				
			\end{itemize}
			\item
			
			Experimental test by breaking condition 2
			
		\end{itemize}
		
		\hypertarget{pascal2015---starting-life-requires-more-than-organic-matter}{\subsection{Pascal2015
				- ``Starting life requires more than organic
				matter''}\label{pascal2015---starting-life-requires-more-than-organic-matter}}
		
		\begin{itemize}
			\item
			
			far-from-equilibrium conditions implies that energy is fed into the
			system in kinetically irreversible way
			
			\item
			
			irreversible through barriers on reverse reaction - so process is not
			only exergonic (energy flows from system to environment), but with a
			kinetic barrier to reverse reaction (see Pascal 2013)
			
		\end{itemize}
		
		\hypertarget{virgo2014---self-organising-autocatalysis}{\subsection{Virgo2014
				- ``Self-Organising
				Autocatalysis''}\label{virgo2014---self-organising-autocatalysis}}
		
		\begin{itemize}
			\item
			
			Thermodynamics at heart of far-from-equilibrium structures
			
			\item
			
			Achem (Virgo and Ikegami 2013) that is thermodynamically reasonable,
			and that doesn't include catalysis, but catalysis emerged as property
			of reaction network. Generated ``moderately large and complex
			autocatalytic cycles''
			
			\item
			
			Direction of reactions chosen by thermodynamic principles, and that
			enabled it to generate catalysis and autocatalytic cycles
			
			\item
			
			Question is under which conditions do these cycles emerge
			
			\item
			
			Basic cause is flow of energy from more to less constrained states, or
			low to high entropy. Blocking simple direct pathways is needed for
			autocatalysis to emerge
			
			\item
			
			Results are robust; only one parameter requires tuning and that only
			for one result (oscillatory behaviour)
			
			\item
			
			Previous work, such as Fontana and Buss 1994 assumed reactions driven
			by external energy source, not modelled. But dynamics quite different
			from model where energetics modelled
			
			\item
			
			Energy modelled by reactions occurring with probability determined by
			rate constant of the reaction.
			
		\end{itemize}
		
		\hypertarget{virgo2013---autocatalysis-before-enzymes-the-emergence-of-prebiotic-chain-reactions}{\subsection{Virgo2013
				- ``Autocatalysis Before Enzymes: The Emergence of Prebiotic Chain
				Reactions''}\label{virgo2013---autocatalysis-before-enzymes-the-emergence-of-prebiotic-chain-reactions}}
		
		http://mitpress.mit.edu/sites/default/files/titles/content/ecal13/ch036.html
		
		\hypertarget{pereto2012---out-of-fuzzy-chemistry-from-prebiotic-chemistry-to-metabolic-networks}{\subsection{Pereto2012
				- ``Out of fuzzy chemistry: from prebiotic chemistry to metabolic
				networks''}\label{pereto2012---out-of-fuzzy-chemistry-from-prebiotic-chemistry-to-metabolic-networks}}
		
		\begin{itemize}
			\item
			
			Review of origin of metabolic networks
			
			\item
			
			Foundations in Ganti Chemoton, Autopoiesis, and Rosen (M,R) systems
			where ``metabolism at the core of understanding life''
			
			\item
			
			Centre around metabolic closure
			
			\item
			
			autocatalysis = ``catalysis of a reaction (or sequence of reactions)
			by one or more of its products'' (encyclopedia of astrobiology). Only
			one known example of an abiotic autocatalytic cycle (formose reaction)
			
			\item
			
			In life, template chemistry is autocatalytic - generation of copies of
			double-stranded polymers from activated monomers and generating waste
			
			\item
			
			Orgel 2008 PLoS metabolic cycles are either simple or autocatalytic.
			Simple has stoichiometric regeneration of one of the reactants (the
			feeder), so x-\textgreater{}y requires A which is consumed, and
			replaced, by the reaction. In constrast, autocatalytic produces
			additional A
			
			\item
			
			metabolism first from a view of life as an emergent properties of
			complex, self-sustained systems; genetics first from life as outcome
			of evolutionary process based on heredity
			
			\item
			
			Oparin 1924 scientists explore the origin of life ``like two parties
			of workers boring from the two opposite ends of a tunnel'' - chemical
			bottom-up and biological top-down approaches
			(\href{http://www.valencia.edu/~orilife/textos/The\%20Origin\%20of\%20Life.pdf}{\emph{http://www.valencia.edu/\textasciitilde{}orilife/textos/The\%20Origin\%20of\%20Life.pdf}})
			Full quote - ``What we do not know today we shall know tomorrow. A
			whole army of biologists is studying the structure and organization of
			living matter, while a no less number of physicists and chemists are
			daily revealing to us new properties of dead things. Like two parties
			of workers boring from the two opposite ends of a tunnel, they are
			working towards the same goal. The work has already gone a long way
			and very, very soon the last barriers between the living and the dead
			will crumble under the attack of patient work and powerful scientific
			thought.''
			
		\end{itemize}
		
		\hypertarget{pascal2013---towards-an-evolutionary-theory-of-the-origin-of-life-based-on-kinetics-and-thermodynamics}{\subsection{Pascal2013
				- ``Towards an evolutionary theory of the origin of life based on
				kinetics and
				thermodynamics''}\label{pascal2013---towards-an-evolutionary-theory-of-the-origin-of-life-based-on-kinetics-and-thermodynamics}}
		
		\begin{itemize}
			\item
			
			Woese early life prior to crossing some Darwinian threshold, early
			life probably communal, swapping components (cf HGT)
			
			\item
			
			Highly unlikely ever know what this early life looked like as fossil
			remains unlikely
			
			\item
			
			metabolism = how ``living matter evades the decay to equilibrium''
			(Schrodinger 1944)
			
			\item
			
			Details may be unknown and unknowable, but principles possible
			
			\item
			
			Transition from non-living to living
			
			
			\begin{itemize}
				\item
				
				Sudden transition to life like today astronomically unlikely -
				single RNA strand capable of ribozyme activity probability of
				10E-60\ldots{} (for 100 monomers)
				
				\item
				
				Alternative is that there is a driving force that moved through
				series of intermediates, of increasing (presumably) degrees of
				``aliveness''
				
				\item
				
				Corollary is that no clear cut transition between non-living and
				living
				
			\end{itemize}
			\item
			
			Definition of Life
			
			
			\begin{itemize}
				\item
				
				Elusive, probably not useful
				
				\item
				
				References: Bruylants 2010, Popa R, 2004, Tirard 2010
				
			\end{itemize}
			\item
			
			Assuming series of contingent states, each state must be stable, long
			enough for further transitions. Implies some alternative means of
			stability to allow further improbable changes
			
			
			\begin{itemize}
				\item
				
				Dynamic Kinetic Stability proposed as mechanism. An autocatalytic
				stability (reproduction) rather than thermodynamic stability
				
				\item
				
				far-from-equilibrium state rather than thermodynamic stability
				
				\item
				
				Driving force is then mechanism that directs (not drives) towards
				systems of increasing stability (associated with persistent
				replicators)
				
			\end{itemize}
			\item
			
			Autocatalytic cycles, competing for same resource, result in
			extinction of less efficient one (Lifson). An exponential effect,
			rather than a linear one (would result in less efficient reduced in
			number not extinguished?)
			
			
			\begin{itemize}
				\item
				
				Autocatalytic cycles near equilibrium don't behave this way, as
				catalysis is reversible
				
				\item
				
				Therefore ACS must be kept away from equilibrium to show DKS
				
				\item
				
				Compatible with self-organization - Prigogine ``the distance from
				equilibrium and the nonlinearity may both be source of order capable
				of driving the system to an ordered configuration''
				
				\item
				
				ACS one type of replicator; others covered in Zachar2010
				
				\item
				
				Need parameters to prevent drop of ACS back to equilibrium - height
				of kinetic barrier; absolute temperature; turnover timescale of
				chemical network (Pascal 2012)
				
				
				\begin{itemize}
					\item
					
					See Scheme 4 for illustration of barrier - requires the waste of
					an amount of energy equal to the kinetic barrier of the reverse
					reaction
					
					\item
					
					(implies any model for continuous ACS needs energy model capable
					of this)
					
				\end{itemize}
				\item
				
				DKS cannot be used to predict outcomes - as context-dependent (see
				discussion of interacting reaction cycles)
				
			\end{itemize}
		\end{itemize}
		
		\hypertarget{vasas2012---evolution-before-genes}{\subsection{Vasas2012 -
				``Evolution before genes''}\label{vasas2012---evolution-before-genes}}
		
		\begin{itemize}
			\item
			
			Darwin ``warm little pond'' Life and letters vol3 1887
			(``\emph{\textbf{But if (and Oh! What a big if!) we could conceive in
					some warm little pond, with all sorts of ammonia and phosphoric salts,
					light, heat, electricity etc, present, that a protein compound was
					chemically formed ready to undergo still more complex
					changes\ldots{}'')}}
			
			\item
			
			Thesis - Template reproduction not required
			
			
			\begin{itemize}
				\item
				
				Autocatalytic replication - early ideas on how in Dyson 1982, 1985
				and Kauffman 1993 and 1986
				
			\end{itemize}
			\item
			
			Herbert Spencer - sense of evolution as change rather than
			accumulation of adaptations
			
			\item
			
			Autocatalytic sets as originally defined in Kauffman (and formalized
			by Hordijk and Steel as RAF sets) not capable of evolution
			
			
			\begin{itemize}
				\item
				
				GARD, conceptually similar, fails test (vasas)
				
				\item
				
				RAF/ACS do not address heredity or selectability
				
			\end{itemize}
			\item
			
			With general requirements though, chemical networks can evolve
			
			
			\begin{itemize}
				\item
				
				e.g., Bimolecular rearrangements - a similar evolutionary mechanism
				- Rowe 2007,2008
				
				\item
				
				Show that multiple autocatalytic cores in compartments can evolve
				(``sustain selectable heritable variation'')
				
				\item
				
				Each core is a unit of selection (loops within core are not)
				
			\end{itemize}
			\item
			
			Autocatalytic core (one or more linked autocatalytic loops) as a
			(chemical network) genotype, and periphery (molecular species
			catalysed by core) as phenotype. Any molecule of a core - but not
			periphery - species is capable of acting as a seed (to produce all
			core and periphery species, in sustained reaction)
			
			\item
			
			Suicidal autocatalysis - as reactions are reversible, if A + X
			\textless{}-\textgreater{} 2A, if X not present then
			auto-decomposition (to produce A and X from 2A)
			
			\item
			
			Viable autocatalytic loops necessary but not sufficient for ENS of
			autocatalytic networks
			
			\item
			
			Original ACS model (Kauffman and Farmer 1986)
			
			
			\begin{itemize}
				\item
				
				Food set of low complexity ( up to some length M of B types of
				monomers) always-present molecules
				
				\item
				
				Each molecule has probability (fixed and constant) P of catalyzing
				each ligation/cleavage reaction (no change in efficiency) without
				specificity to any particular reaction (on average, therefore, P
				fraction of possible reactions)
				
				\item
				
				Above a certain level P, chain reaction and ACS appear
				
				\item
				
				H+S showed percolation produced fully-connected ACS above Pc
				
				\item
				
				Farmer showed logistic growth of chemical network above same value
				Pc
				
			\end{itemize}
			\item
			
			Objections and fixes to original model
			
			
			\begin{itemize}
				\item
				
				Issues with P - as length M of polymers increases, number of
				reactions possible increases faster than number of molecules
				possible, and so proportion of catalysed reactions rises - not
				realistic. Modification to model following Lifson to base P on joint
				probability of molecule is a catalyst, and molecule catalyses a
				particular reaction. Alternative seems more likely - that P is a per
				reaction probability.
				
				\item
				
				``Paradox of specificity'' - large number of molecules needed for
				ACS, but associated harmful side-reactions rises with size. Solution
				appears to be competitive inhibition - low levels remove
				side-reactions while only slightly slowing growth
				
			\end{itemize}
			\item
			
			Evolvability
			
			
			\begin{itemize}
				\item
				
				Ganti and Wachtershauser - ``if distinct, organizationally
				different, alternative autocatalytic networks can coexist in the
				same environment then they could compete with each other and the
				`fittest' would eventually prevail''
				
				
				\begin{itemize}
					\item
					
					Narrow view of unit of selection (why?)
					
					\item
					
					Reaction networks must posses multiple attractors and transitions
					between attractors possible
					
					\item
					
					Attractor as essence of self-organization (Wesson)
					
				\end{itemize}
				\item
				
				Catalytic network must be
				
				
				\begin{itemize}
					\item
					
					compartmentalized
					
					
					\begin{itemize}
						\item
						
						to filter out harmful modifications, demonstrated in Fernando
						2007
						
						\item
						
						modelled as flow reactors
						
					\end{itemize}
					\item
					
					platonic reaction network must have multiple attractors
					
					
					\begin{itemize}
						\item
						
						multiple pathways of autocatalysis
						
					\end{itemize}
					\item
					
					some of these attractors must be selectable
					
					
					\begin{itemize}
						\item
						
						transit between attractors according to fitness value
						
						\item
						
						artificial selection used in experiment to test evolvability
						
						\item
						
						Moran process used to test evolvability when subject to natural
						selection
						
					\end{itemize}
					\item
					
					Original ACS only one attractor (and only one autocatalytic core)
					and therefore no selection possible. Kauffmans ACS will eventually
					stablize into attracting network and never be Darwinian
					
					\item
					
					One core = one attractor
					
					\item
					
					Inhibition suggested to form multiple dynamic attractors
					
					
					\begin{itemize}
						\item
						
						But transitions between them rare - either periodically or
						chaotically
						
						\item
						
						Could not be stably selected - so not Darwinian
						
						\item
						
						Multiple attractors not sufficient
						
					\end{itemize}
					\item
					
					Formation of new molecular species (without necessarily
					inhibition) showed multiple cores
					
					
					\begin{itemize}
						\item
						
						Mix of large, stable attractors, and less stable ones
						
						\item
						
						Could be stably selected
						
						\item
						
						New species can lead to new core and it is selectable
						
					\end{itemize}
					\item
					
					Heredity
					
					
					\begin{itemize}
						\item
						
						Not addressed, but not necessary - core-periphery dichotomy
						suggested as gp-mapping, and therefore cores are units of
						inheritance and create periphery.
						
						\item
						
						But a core is one-bit of heritable information, and so not
						immediately capable of OEE
						
						\item
						
						But could form a pathway
						
					\end{itemize}
				\end{itemize}
			\end{itemize}
		\end{itemize}
		
		\hypertarget{vasas2012a---evolution-before-life}{\subsection{Vasas2012a
				- ``Evolution before life''}\label{vasas2012a---evolution-before-life}}
		
		\begin{itemize}
			\item
			
			Advantages of template replication
			
			
			\begin{itemize}
				\item
				
				Potential - n\^{}L sequences, effectively unlimited
				
				\item
				
				Mutations are hereditary
				
				\item
				
				Problem of correct copying - drops with length (Eigen's paradox -
				good copier longer than could form)
				
				\item
				
				hypercycles (Eigen 1971) (polynucleotide strings produce protein
				which assists template reproduction of next string etc) destroyed by
				shortcuts and parasites (Szathmary 1987) which benefit but don't
				contribute - need compartmentalization
				
				\item
				
				template replication is autocatalytic; any mutation is automatically
				autocatalytic
				
			\end{itemize}
			\item
			
			Non template replication
			
			
			\begin{itemize}
				\item
				
				Autocatalytic cycles
				
				
				\begin{itemize}
					\item
					
					Any element of cycle can form a seed, others generated from food
					set
					
					\item
					
					More reactions in the cycle, less likely to remain functional -
					problem of side reactions
					
					\item
					
					p reactions, si specificity (this reaction rather than another
					side reaction), turnover t, grows only if t* product all si's
					\textgreater{} 1
					
					\item
					
					Specificity limited for uncatalysed reactions (uncatalysed as
					assume before formation of enzymes, as enzymes are products of
					more complicated chemistry), so size of cycle limited
					
					\item
					
					Explains why know of only one or two (condensation reaction in
					Patzke and von Kiedrowski 2007) natural examples other than
					Formose, although some hypothetical archaic ones suggested
					
					\item
					
					Variations in cycle may not automatically be autocatalytic (why?
					because mutant copies generally not functional), and so not
					heritable variation
					
					\item
					
					Chemical avalanche model shows alternative - changes by
					adding/inventing new autocatalytic cycles (from products of rare
					side reactions) rather than by modifying existing ones
					
					
					\begin{itemize}
						\item
						
						Attractor based heredity, rather than storage (RNA/DNA) based
						(Hogeweg 1998)
						
						\item
						
						Random sample of molecules from a compartment ``inherits'' its
						state - same/similar molecules so same/similar autocatalytic
						cycles/states (particulate model of heredity?) (cycles created
						from seed molecules which are heritable)
						
						\item
						
						Obviously need multiple attractors/states for this to be
						meaningful - otherwise
						
						\item
						
						No real world support for this - although Formose cycle not
						fully understood and so still might resemble this Chemical
						Avalanche model
						
					\end{itemize}
				\end{itemize}
				\item
				
				Autocatalytic sets
				
				
				\begin{itemize}
					\item
					
					formed from autocatalytic loops - ``each molecule in the loop
					depends on the previous one for it production, and at least one of
					the steps is a catalytic dependency'' - just need one step where
					product is not consumed but acts as catalyst in next step for loop
					to maintain autocatalytic properties
					
					\item
					
					Loops grow because of catalysis, not because of stoichiometry (as
					in cycles). So much less subject to side reactions (why?)
					
					\item
					
					Inhibition in network permits more than one core/attracting
					network (Kauffman speculation). Same mechanism as for Chemical
					Avalanches, but here cycles are same as one-member autocatalytic
					loops.
					
				\end{itemize}
				\item
				
				Autocatalysis in Life
				
				
				\begin{itemize}
					\item
					
					Present in all three subsystems (of Ganti)
					
					
					\begin{itemize}
						\item
						
						DNA replicates with enzymatic help
						
						\item
						
						some metabolites, such as ATP, exclusively autocatalytic
						
						\item
						
						lipids in a membrane enhance addition of other lipids
						
					\end{itemize}
					\item
					
					Lifson 1997 argues that evolution proceeds life
					
					\item
					
					``addressing selectability in a model is equivalent to discussing
					whether it can evolve or not''
					
					\item
					
					Properties of evolution, from Maynard-Smith transferred to
					chemical systems
					
					
					\begin{itemize}
						\item
						
						Individuals - either single molecules or compartmentalized sets
						of molecules (why do they have to compartmentalized?)
						
						\item
						
						Multiplication - from direct or indirect autocatalysis
						
						\item
						
						Heredity - state inheritance
						
					\end{itemize}
				\end{itemize}
			\end{itemize}
		\end{itemize}
		
		\hypertarget{fernando2008xyfernando2007pf}{\subsection{Fernando:2008xy,Fernando:2007pf
			}\label{fernando2008xyfernando2007pf}}
		
		\begin{itemize}
			\item
			
			\href{http://www.sussex.ac.uk/Users/ctf20/dphil_2005/Software/Chemistry30thAprilCyclicMassCheck.zip}{\emph{http://www.sussex.ac.uk/Users/ctf20/dphil\_2005/Software/Chemistry30thAprilCyclicMassCheck.zip}}
			
			\item
			
			flow-reactor for evolution of metabolism in lipid aggregates based on
			predefined types and reactions
			
			\item
			
			replication by external agitation, variation by chemical avalanches
			
		\end{itemize}
		
		\hypertarget{fernando2007pf---natural-selection-in-chemical-evolution}{\subsection{Fernando:2007pf
				- ``Natural selection in chemical
				evolution''}\label{fernando2007pf---natural-selection-in-chemical-evolution}}
		
		\begin{itemize}
			\item
			
			prebiotic evolution ``process producing increasingly organized
			chemical supersystems'' - Ganti 2003b, Hogeweg and Takeuchi 2003
			
			\item
			
			Avoids definition of life - instead presents goal as any one of
			
			
			\begin{itemize}
				\item
				
				OEE - Bedau 2000
				
				\item
				
				basic autonomy - Ruiz-Mirazo 2004
				
				\item
				
				production of modular replicators with unlimited heredity potential
				Maynard-Smith and Szathmary 1995/Szathmary 2000
				
				\item
				
				course of evolution by which\ldots{} Oparin 1964
				
				\item
				
				coupled cycling of bioelements Morowitz 1968 1971
				
				\item
				
				maximization of energy production by a biosphere Kleidon 2004
				
				\item
				
				minimal unit of life Ganti 2003a
				
				\item
				
				autopoetic unit Maturana and Varela 1992
				
			\end{itemize}
			\item
			
			Assume that initial stage in prebiotic evolution was same as mechanism
			for variation in chemical evolution - stochastic reaction avalanches
			Wachtershauser 1992
			
			\item
			
			Alternatives
			
			
			\begin{itemize}
				\item
				
				Kauffman 1986 fatally flawed by ignoring side reactions which cause
				removal of species - Szathmary 2005 could not produce reflexive
				autocatalytic set under Kauffmans conditions
				
				\item
				
				Fontana and Buss - random removal of lambda-objects provides
				selective pressure (in which direction though?), but Decker and
				Szathmary didn't see selection or complexification (?)
				
			\end{itemize}
			\item
			
			Discussion of composomes and self-replication
			
			\item
			
			Model uses external agitation to divide liposomes and hence
			replication
			
			\item
			
			Holistic or attractor based heredity (Szathmary 2000)
			
		\end{itemize}
		
		\hypertarget{other}{\section{Other}\label{other}}
		
		\hypertarget{wang2014---a-hybrid-offon-lattice-model-of-emergence-and-maintenance-autopoiesis}{\subsection{Wang2014
				- ``A Hybrid Off/On-Lattice Model of Emergence and Maintenance
				Autopoiesis''}\label{wang2014---a-hybrid-offon-lattice-model-of-emergence-and-maintenance-autopoiesis}}
		
		\href{https://mitpress.mit.edu/sites/default/files/titles/content/alife14/978-0-262-32621-6-ch085.pdf}{\emph{https://mitpress.mit.edu/sites/default/files/titles/content/alife14/978-0-262-32621-6-ch085.pdf}}
		
		\begin{itemize}
			\item
			
			Bourgine and Stewart 2004 ``An autopoietic system is a network of
			processes that produces the components that reproduce the network, and
			that also regulates (from inside) the membrane conditions necessary
			for its ongoing existence as a network.''
			
			\item
			
			Continuous 2D environment
			
			\item
			
			Four types of particle - substrate, monomer, component; monomers can
			decay into waste particles. Types behave differently, e.g., monomers
			don't impede substrates
			
			\item
			
			Monomers however bind to a grid, and cannot move once bound (except
			vibrate) - so a rigid membrane
			
			\item
			
			Three key interactions between particle types - synthesis
			(S+S-\textgreater{}C), decay (M-\textgreater{}W), repair
			(C+M-\textgreater{}M+M)
			
			\item
			
			No physics engine - just NetLogo
			
			\item
			
			Can not only maintain existing autopoeitic cell, but build new one
			(``emergence autopoiesis'') from seed of M. Emergence form shows
			morphogenetic engineering - form arises from properties of model
			
			\item
			
			Three main modeling approaches/perspectives to life:
			
			
			\begin{itemize}
				\item
				
				``origins'' (or protocells which don't necessarily recapitulate
				life's actual origins)- Transitions from Nonliving to Living Matter
				workshop reports (Rasmussen et al., 2004)
				https://www.sciencemag.org/content/303/5660/963.short
				
				\item
				
				``autopoiesis''
				
				\item
				
				``self-replication'' (Langton, 1984; Sayama, 2000).
				
			\end{itemize}
		\end{itemize}
		
		\hypertarget{ono2002}{\subsection{Ono2002}\label{ono2002}}
		
		\begin{itemize}
			\item
			
			Boid-like rules in Lattice Artificial Chemistry leads to membrane
			formation
			
			\item
			
			Fixed types of particles with associated
			hydrophobic/hydrophilic/neutral class (with orientation). Fixed
			reaction paths to form an autocatalytic set of reactions
			
		\end{itemize}
		
		\hypertarget{sayama2011---seeking-open-ended-evolution-in-swarm-chemistry}{\subsection{Sayama2011
				- ``Seeking open-ended evolution in Swarm
				Chemistry''}\label{sayama2011---seeking-open-ended-evolution-in-swarm-chemistry}}
		
		\begin{itemize}
			\item
			
			New focus for Swarm Chemistry on OEE and evolutionary dynamics
			
			\item
			
			``self-organization of dynamic patterns of kinetically interacting
			heterogeneous particles. A swarm population in Swarm Chemistry
			consists of a number of simple self-propelled particles moving in a
			two-dimensional continuous space. Each particle can perceive average
			positions and velocities of other particles within its local
			perception range, and change its velocity in discrete time steps
			according to kinetic rules similar to those of Reynolds' Boids ``
			
			\item
			
			``The key components of model revisions were the local transmission of
			recipe information from active to passive particles, as well as
			between two active particles, and the possibility of mutation and
			competition among different recipes. These model assumptions realized
			the three key ingredients of evolution---inheritance, variation and
			selection---all arising from local processes occurring at microscopic
			levels. Yet our simulation results successfully demonstrated that
			ecological and evolutionary dynamics of macroscopic structures can
			emerge in such systems.``
			
			\item
			
			Speculation from earlier work that dynamic environments promote
			evolutionary changes
			
			\item
			
			Mostly qualitative arguments\ldots{}
			
		\end{itemize}
		
		\hypertarget{nowostawski2005---an-architecture-for-self-organising-evolvable-virtual-machines}{\subsection{Nowostawski2005
				- ``An Architecture for Self-Organising Evolvable Virtual
				Machines''}\label{nowostawski2005---an-architecture-for-self-organising-evolvable-virtual-machines}}
		
		\begin{itemize}
			\item
			
			Initial exploration of evolution interesting
			
			\item
			
			Specifics of model less so as takes particular VM approach and purpose
			
		\end{itemize}
		
		Standish2008 - ``Open-ended Artificial Evolution''
		
		Stout - ``Validation of Evolutionary Activity Metrics for Long-Term
		Evolutionary Dynamics''
		
		Hotz2003 - ``Genome-Physics Interaction as a New Concept\ldots{}''
		
		Corriea - ``Computational evolution: taking liberties''
		
		RuizMirazo - ``A Universal Definition of Life''
		
		\hypertarget{ruiz-mirazo2008nb---enabling-conditions-for-open-ended-evolution}{\subsection{Ruiz-Mirazo:2008nb
				- ``Enabling conditions for `open-ended
				evolution'''}\label{ruiz-mirazo2008nb---enabling-conditions-for-open-ended-evolution}}
		
		\begin{itemize}
			\item
			
			Context stated as ``in biological theory''
			
			\item
			
			``To avoid confusion or fruitless discussion, we will here refer to
			open-ended evolution (rather than Darwinian evolution) as a process in
			which there is the possibility for an indefinite increase in
			complexity''
			
			\item
			
			Comments on McMullin's review of von Neumann where von Neumann doesn't
			require an exact definition of complexity, but merely ``the crudest of
			qualitative rankings''
			
			\item
			
			Interesting philosophical and historical discussion on OOL
			
		\end{itemize}
		
		Watson - ``Towards more Relevant Evolutionary Models''
		
		Whitacre - ``Degeneracy''
		
		Defaweux - ``Evolutionary Transitions..''
		
		McMullin - ``30 Years of Computational Autopoiesis''
		
		Miconi - ``Evolution and Complexity''
		
		Korb - ``Evolution unbound''
		
		Pigliucci2008 - ``Is evolvability evolvable?''
		
		\hypertarget{hogeweg1998---on-searching-generic-properties}{\subsection{Hogeweg1998
				- ``On searching generic
				properties\ldots{}''}\label{hogeweg1998---on-searching-generic-properties}}
		
		\begin{itemize}
			\item
			
			``The simplest way of defining an evolutionary process is to define
			some set of predefined interactions between replicators and subject
			one (or a few) of the parameters of the system to mutations (selection
			automatically ensues from the dynamics of the system).''
			
			\item
			
			``\ldots{}the processes associated with the major transitions are an
			automatic consequence of mutation and selection, due to the generation
			of higher levels of selection due to spatial self-organization.''
			
			
			\begin{itemize}
				\item
				
				iff the interactions between the replicators are defined locally
				(meaning spatially) (from earlier work)
				
				\item
				
				because local interactions will form ``higher-level structures
				(e.g., spiral waves, turbulence, path like structures of different
				sizes) which constitute different levels of selection.''
				
				\item
				
				stress is on new levels of selection, rather than other elements of
				transitions
				
				\item
				
				But still ``they do not give us `novel' entities, as biotic
				evolution undoubtedly has''
				
				
				\begin{itemize}
					\item
					
					Perhaps because lack one of the elements in Maynard-Smith:1995lw -
					``transition from limited inheritance to universal inheritance''
					
					\item
					
					Claim see other three in earlier work:
					
					
					\begin{itemize}
						\item
						
						Symbiogenesis - ``properties of local interacting, evolutionary
						systems ...embody a process reminiscent of `Symbiogenesis' in
						that self-sufficiency is (partly) given up in favor of the
						larger scale entities.''
						
						\item
						
						Conflicts among levels of selection - claims interactions
						between meso-scale and micro-scale entities are inherent (from
						observation of spiral wave experiments)
						
						\item
						
						Division of labour - different elements of spatial structure
						reproduce differently, hence germ-like and soma-like...
						
					\end{itemize}
				\end{itemize}
			\end{itemize}
			\item
			
			Necessary/essential for OEE - ability to redefine interactions,
			genetic representations, and fitness of the replicators
			
		\end{itemize}
		
		\hypertarget{dorin2006fk}{\subsection{Dorin:2006fk}\label{dorin2006fk}}
		
		\begin{itemize}
			\item
			
			\textbackslash{}gls\{achem\} to explore virtual ecosystems
			
		\end{itemize}
		
		\hypertarget{ducharme2012}{\subsection{Ducharme2012}\label{ducharme2012}}
		
		\begin{itemize}
			\item
			
			models the energy changes associated with reactions. The chemistry is
			spatial; atoms are arranged on a 2-dimensional grid and have velocity.
			When two atoms pass within a particular distance, they interact. The
			possible types of interactions are prespecified, with the type chosen
			being driven by the atomic composition and energies of the interacting
			atoms.
			
			\item
			
			Reactions are therefore between atoms rather than molecules; a
			molecule in this chemistry is a combination of atoms arranged in a
			particular structure, re-examined after each reaction to form a stable
			configuration based on expectations from real-world chemistry.
			
			\item
			
			Although computational costs are not reported, it seems plausible that
			the calculation of intersections on a 2-dimensional grid will be
			expensive for large molecular populations. Another cost comes from the
			re-arrangement of molecules into energy-efficient configurations. This
			spatial structuring enables the model to restrict atomic interactions
			to those atoms that are accessible on a molecule, but at the cost of
			additional modelling complexity
			
		\end{itemize}
		
		\hypertarget{wrobel2012}{\subsection{Wrobel2012 }\label{wrobel2012}}
		
		\begin{itemize}
			\item
			
			``\textbackslash{}ldots steps towards realizing a biologically
			realistic system'', motivational autonomy in searching for food and
			water while needing to perform work
			
			\item
			
			Extension of previous work to three-resource problem; biological
			inspirations without explicit grounding; appeal to theories of minimal
			cognition, ecological grounding and motivational autonomy
			
			\item
			
			GRN driven by inputs from environmental sensors and driving output
			motor actuators.
			
			\item
			
			Abstractions of food, water and work (called ``beauty'') - marked by
			scents that can be detected by animat sensors - food and water are
			required to generate motive energy (through a simulation of a
			microbial fuel cell).
			
			\item
			
			Isolated mobile animats, synchronous generational evolutionary model
			with fixed population size, explicit fitness function and binary
			tournament selection
			
		\end{itemize}
		
		\hypertarget{prieto2010trueba2011trueba2012}{\subsection{Prieto2010,Trueba2011,Trueba2012}\label{prieto2010trueba2011trueba2012}}
		
		\begin{itemize}
			\item
			
			Evolution guided by energy-regulation towards a fixed objective;
			objective of collective gathering of blocks requires coordinated
			behaviours from agents that develop specializations
			
			\item
			
			Situated and embodied (from ``living'' in environment), no biological
			significance claimed
			
			\item
			
			Agents are unchanging physical robots; active life-span determined by
			internal energy-analogue
			
			\item
			
			Static morphology; agent energy-analogue depleted by activity and
			increased by performing specific tasks
			
			\item
			
			Asynchronous generations, local interactions determine reproduction,
			no global competition, no fitness function; evolution (only of
			behaviour) happens in real-time rather than offline before deployment
			
		\end{itemize}
		
		\hypertarget{lobo2010}{\subsection{Lobo2010}\label{lobo2010}}
		
		\begin{itemize}
			\item
			
			Emergence of novelty and diversity; artificial genome model similar to
			that in \textbackslash{}textcite\{Reil:1999rp\}, development of
			organisms modelled by springs to path-follow; three cell-types;
			results classified by hand (exploratory method)
			
			\item
			
			``\textbackslash{}ldots evolution of genetic regulation could be a
			sufficient condition for the emergence of novelty and diversity''. No
			previous study of regulation has succeeded in showing both novelty and
			diversity under evolution.
			
			\item
			
			Embedding purely through physics modelling.
			
			\item
			
			No resource model - goal is path following.
			
			\item
			
			Standard EA population model - individuals are assessed by the EA in
			isolation
			
			\item
			
			Four behaviours considered evidence of "novelty" and "diversity"; no
			predefined measure (so concerns of seeing what wish to see); critical
			assumption without evidence that results will extend ad infinitum
			
		\end{itemize}
		
		\hypertarget{taylor1999taylor1999a-phd-thesis}{\subsection{Taylor1999,Taylor1999a
				(PhD thesis)}\label{taylor1999taylor1999a-phd-thesis}}
		
		\begin{itemize}
			\item
			
			Creation of \textbackslash{}gls\{alife\}
			
			\item
			
			Extension of Tierra \textbackslash{}cite\{Ray1991\} by cell
			regulation, parallel processes, energy modelling
			\textbackslash{}cite{[}p.4{]}\{Taylor1999a\}
			
			\item
			
			Cells are explicitly modelled as bitstrings which run as programs
			
			\item
			
			Adhoc theoretical model (essentially Tierra plus some likely
			improvements)
			
		\end{itemize}